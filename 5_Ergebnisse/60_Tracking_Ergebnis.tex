\section{FPGA Tracking Loop}
Da derzeit noch keine Möglichkeit besteht die Acquisition Ergebnisse an den Tracking Loop im FPGA weiterzugeben wurde für diesen Versuch die gesamte \lstinline$gps_top$ Komponente mit \emph{ghdl} simuliert. Da es sich um eine \gls{RTL} Simulation mit dem selben Code der auch auf dem FPGA ausgeführt wird handelt, unterscheidet sich das Ergebnis nicht von der tatsächlichen Ausführung im FPGA.

Dazu wurde eine Testbench \lstinline$gps_top_tb$ erstellt, die die \lstinline$gps_top$ Komponente enthält und die Test-Stimuli erzeugt. Der Rohdaten Stimulus wird durch eine Unterkomponente \lstinline$samples_from_file$ erzeugt. Diese Komponente wurde geschrieben um die Rohdaten aus einer Datei zu lesen. Die \lstinline$samples_from_file$ Komponente gibt die Rohdaten über das MAX2769 Interface an die \lstinline$gps_top$ Komponente weiter, wirkt also quasi als MAX2769 \emph{Dummy}. Die Ausgaben der Korrelatoren und einiger anderer Kennwerte (z.B. Code/Carrier FCW) werden von der \lstinline$gps_top$ Komponente selbst in Dateien geschrieben. Einen Überblick über den Versuchsaufbau gibt \FR{Versuch_Tracking_Testbench.pdf}.
%UML_comp_structure_gps_top.pdf  Section_gps_top_entity

Um zu prüfen ob die durch den Tracking Loop demodulierten Daten auch den gesendeten NAV Daten entsprechen wurden für diesen Versuch künstlich erzeugte Rohdaten mit bekannter PRN, SNR, Codephase und Dopplerverschiebung erzeugt. 


