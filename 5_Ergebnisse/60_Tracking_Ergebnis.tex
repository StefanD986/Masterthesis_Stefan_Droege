\section{FPGA Tracking Loop}
Da derzeit noch keine Möglichkeit besteht die Acquisition Ergebnisse an den Tracking Loop im FPGA weiterzugeben wurde für diesen Versuch die gesamte \lstinline$gps_top$ Komponente mittels \emph{ghdl} simuliert. Da es sich um eine \gls{RTL} Simulation mit dem selben Code der auch auf dem FPGA ausgeführt wird handelt, unterscheidet sich das Ergebnis nicht von der tatsächlichen Ausführung im FPGA.

Dazu wurde eine Testbench \lstinline$gps_top_tb$ erstellt, die die \lstinline$gps_top$ Komponente enthält und die Test-Stimuli erzeugt. Der Rohdaten Stimulus wird durch eine Unterkomponente \lstinline$samples_from_file$ erzeugt. Diese Komponente wurde geschrieben um die Rohdaten aus einer Datei zu lesen. Die \lstinline$samples_from_file$ Komponente gibt die Rohdaten über das MAX2769 Interface an die \lstinline$gps_top$ Komponente weiter, wirkt also quasi als MAX2769 \emph{Dummy}. Die Ausgaben der Korrelatoren und einiger anderer Kennwerte (z.B. Code/Carrier FCW) werden von der \lstinline$gps_top$ Komponente selbst in Dateien geschrieben. Einen Überblick über den Versuchsaufbau gibt \FR{Versuch_Tracking_Testbench.pdf}.

\FGimg[Versuchsaufbau Tracking Loop]{Versuch_Tracking_Testbench.pdf}{Aufbau der Testbench mit der die Simulation des Tracking Loops durchgeführt wurde. Die Rohdaten werden der \lstinline$gps_top$ Komponente über das MAX2769 Interface zugeführt (siehe \FR{UML_comp_structure_gps_top.pdf}).}{12cm}

Um zu prüfen ob die durch den Tracking Loop demodulierten Daten auch den gesendeten NAV Daten entsprechen, wurden für diesen Versuch künstlich erzeugte Rohdaten mit bekannter PRN, SNR, Codephase und Dopplerverschiebung erzeugt. 

\subsection{Versuchsvorbereitung}
\emph{ghdl} ist ein Compiler für VHDL, der aus dem VHDL Code eine ausführbare Datei erstellt. Es wird im folgenden vorausgesetzt, dass \emph{ghdl} installiert ist. Hier wurde die Version 0.33 verwendet.

Beim kompilieren der Dateien muss auf die Reihenfolge und zuordnung der Quelldateien zu den richtigen Libraries geachtet werden. Um dies zu vereinfachen wurde ein \emph{Makefile} geschrieben, was in \lstinline$gps_fpga/sim$ zu finden ist. Das Makefile kompiliert außerdem automatisch die Firmware für den MBlite Softcore Prozessor. Mit dem Befehl \lstinline$make$ wird der VHDL Code kompiliert und die ausführbare Datei \lstinline$gps_top_tb$ erstellt. 


\subsection{Versuchsdurchführung}
