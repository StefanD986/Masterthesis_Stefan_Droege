% this file is jbba_titelseite.tex

\thispagestyle{empty}
 
\makeatletter
\begin{titlepage}

\begin{tabularx}{1.1\textwidth}{Xrr}
  & \includegraphics[height=1.5cm]{figures/ihflogoblau2001.jpg} \hspace{0.5cm}
%\includegraphics[height=2.5cm]{figures/Milwaukee_logo.pdf} & 
\includegraphics[height=1.5cm]{figures/RWTHAachenUniversity.jpg}	 \\ 
\end{tabularx}

\begin{flushleft}
Diese Arbeit wurde vorgelegt am Institut für Hochfrequenztechnik
\end{flushleft}
\vspace{2cm}

\begin{center}

\textbf{\Huge
\@title
\vfill
\LARGE Masterarbeit\\
\vspace{0,5cm}
\LARGE von\\
\vspace{1cm}
\LARGE Stefan Dröge}
\vfill
\large
\vfill
 \begin{tabbing}
  \textbf{Datum der Abgabe:} \hspace*{1cm}\= \kill      % erste Zeile der Tabelle bestimmt Abstand der Spalten --> längstes Wort reinschreiben, als Referenzwert, durch kill wird nicht angezeigt
  \textbf{Matrikelnummer:}			\> 342108\\
  \textbf{Fakultät:}				\> Elektrotechnik und Informationstechnik\\
  \textbf{Studiengang:}				\> Elektrotechnik, Informationstechnik \\
                                    \> und Technische Informatik M.Sc.\\
  \textbf{Schwerpunkt:}             \> Informations- und Kommunikationstechnik\\
  \textbf{Betreuer:}  	  			\> Prof. Dr.-Ing. Dirk Heberling\\
                                    \> Dipl. Ing. Ralf Wilke\\
  \textbf{Datum der Abgabe:}		\> 4. Juni 2016
 \end{tabbing}
 
\end{center}
\end{titlepage}
\makeatother

%\clearpage			%setzt alle floating-elemente bevor nächste seite anfängt
\endinput

\maketitle