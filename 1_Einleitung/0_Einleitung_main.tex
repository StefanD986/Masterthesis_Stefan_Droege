\chapter{Einleitung}

Diese Arbeit beschreibt die Entwicklung eines Software Defined GPS Empfängers für den \dscubesat.

Der Hardware-Teil des GPS Empfängers wurde bereits in vergangen Arbeiten entwickelt. Die Entwicklung des Software-Teils und die Integration auf das COM Board des \dscubesat standen jedoch noch aus. Die Software des GPS Empfängers teilt sich auf den FPGA und den DSP des COM Board auf. Im DSP werden vor allem weniger zeitkritische, aber speicher- und rechenintensive Teile implementiert. Im FPGA werden zeitkritische, gut parallelisierbare Teile implementiert. In der programmierbaren Hardware des FPGA wurde außerdem ein Softcore Prozessor integriert, der Aufgaben erledigt, deren Implementierung in reiner programmierbarer Hardware sehr umständlich wäre.

Die Arbeit gliedert sich in drei Teile. Nach einer kurzen Beschreibung zur Motivation des Projekts und einem Blick auf ähnliche Projekte werden im ersten Teil einige Grundlagen zu GPS Signal und GPS Empfängern besprochen. Ein wichtiger Parameter bei der Demodulation ist die Dopplerverschiebung des Signals, der durch die hohe Geschwindigkeit des \dscubesat im Orbit noch verstärkt wird. Deshalb wird eine Formel hergeleitet, mit der sich näherungsweise die maximale Dopplerverschiebung berechnen lässt. Ein weiterer Abschnitt gibt eine Übersicht über die für diese Arbeit relevanten Teile des COM Subsystems, auf dem der GPS Empfänger implementiert wird.

Im zweiten Teil wird zuerst der Entwurf des Empfängers besprochen. Danach werden die Teile des Entwurfs besprochen, die bisher realisiert wurden. Abschließend werden relevante Versuchsergebnisse präsentiert.

Ein wichtiger Teil der Entwurfsphase für die Software war die Umsetzung der Berechnungen in Festkomma Arithmetik. Im Appendix wird deshalb eine Einführung zu der hier verwendeten Notation und dem Festkommaentwurf gegeben.

\section{Motivation}
Der \dscubesat ist ein kleiner \gls{LEO} Satellit, der im Rahmen mehrerer Abschlussarbeiten an der FH Aachen und der RWTH Aachen entstanden ist\footnote{Einige der vorangegangenen Arbeiten zum Kommunikations-Subsystem sind z.B. \cite{DragsailKaiMA}, \cite{DragsailRalfMA}, \cite{DragsailAndrejMA},
\cite{DragsailUweMAGroundDataHandling}, \cite{DragsailVolkerMACommstandard}, \cite{DragsailGeorgMAGroundSDR}, \cite{DragsailNeelamMAGroundDSP} und weitere.}.

Um \enquote{Weltraumschrott} zu vermeiden ist das Ziel, dass Cubesats nicht länger als 25 Jahre im Orbit verbleiben sollten. Die Verweildauer wird unter anderem durch die Höhe des Orbits festgelegt in dem die Satelliten ausgesetzt werden. Die Orbithöhe in der ein  Satellit von der Größe eines Cubesat maximal 25 Jahre bleibt wird von der NASA als etwa \SI{600}{\kilo\meter} angegeben \cite{NASAOrbitalDebris}.

Cubesats werden üblicherweise als Beiladung beim Start kommerzieller Satelliten in den Weltraum geschickt. Daher ist der Orbit nicht beliebig frei wählbar, sondern hängt davon wo der kommerzielle Satellit ausgesetzt werden soll.

Deshalb ist eines der Experimente an Board des Satelliten das Entfalten eines Segels am Ende der Lebenszeit des Satelliten um den Wiedereintritt zu beschleunigen und so die Verweildauer auch bei höheren Orbits zu begrenzen. Für das Experiment sind die Orbitdaten, bzw. die Änderung der Orbitdaten durch das Entfalten des Segels von Interesse. Die Orbitdaten werde durch Organisationen wie NORAD veröffentlicht. Die Genauigkeit dieser Orbitdaten für kleine Satelliten wurde in anderen Arbeiten untersucht, beispielsweise in \cite{TLEAccuracyKahr} oder \cite{TLEAccuracyDoyle}. Bedingt durch die geringe Größe ist sind diese Daten nicht immer sehr genau und weichen je nach Parameter um bis zu \SI{1.29}{\percent} oder \SI{370}{\percent} ab \cite{TLEAccuracyDoyle}.

Da GPS Signale auch von \gls{LEO} Satelliten empfangen werden können, bietet es sich an, dass der Satellit selbst seine Position und damit seinen Orbit berechnet. Aufgrund von Waffenexportbeschränkungen funktionieren jedoch die wenigsten GPS Empfänger Chips im Weltraum. Für den \dscubesat musste deshalb ein alternativer Weg gegangen werden. Die Rechenleistung auf heutigen Embedded Systems wie dem COM Board des \dscubesat reicht aus, um GPS Empfänger vollständig in Software zu implementieren. Der benötigte Hardware Teil beschränkt sich damit auf ein HF Frontend, was ein digitalisiertes Basisband Signal liefert.

Auf dem COM Board des \dscubesat wurde deshalb ein GPS Frontend Chip eingebaut, der ein digitalisiertes Rohdatensignal liefert. Die Signalverarbeitung bis zur Positionsbestimmung kann somit in Software auf dem FPGA und DSP geschehen.

\section{Ähnliche Arbeiten}
Software Defined Radio GPS Empfänger sind nicht neu. Eines der ersten Projekte entstand schon 1994 \cite{vidmar1994diy}. Das diskret aufgebaute HF Frontend ist jedoch noch vergleichsweise aufwändig und wäre schwer in einem kleinen Cubesat unterzubringen. Ein sehr ähnliches Projekt aus dem Jahre 2013 \cite{A_holme_2013} verwendet bereits ein etwas vereinfachtes HF Frontend, welches jedoch immer noch vergleichsweise viel Platz beansprucht.

Seit einigen Jahren sind jedoch relativ günstige GPS Frontend Chips verfügbar, die den kompletten HF Teil, inklusive Mischer und Lokaloszillator und HF/ZF Filtern mit einem AD Wandler vereinen. Damit bietet sich die Möglichkeit, ohne aufwändiges HF Schaltungsdesign, ein GPS Frontend in Schaltungen zu integrieren. Ein solcher GPS Frontend Chip wurde auch im Design des COM Board für den \dscubesat integriert. 

Eine weitere interessante Arbeit ist \cite{Birklykke2010}, die das Ergebnis eines mehrsemestrigen Projekts ist, welches sich speziell mit der Implementierung eines GPS Empfängers für kleine \gls{LEO} Satelliten beschäftigt hat. Der Quellcode des in der Arbeit entwickelten GPS Empfängers ist leider nicht verfügbar.

Es gibt auch mehrere kommerzielle Software Defined GPS Receiver, von denen das SwiftNav Piksi \cite{SwiftNavPiksi} dadurch hervorsteht, dass ein großer Teil der Software unter GPL Lizenz frei verfügbar ist. Ein elementarer Bestandteil, die FPGA Firmware ist allerdings nicht veröffentlicht. Für zukünftige Weiterentwicklung der hier vorgestellten Arbeit sollte jedoch untersucht werden ob die Open Source Teile des SwiftNav Piksi Projekts hilfreich sein können.

Abseits von Realisierungen für Embedded Systeme ist auch das GNSS-SDR Projekt \cite{gnss-sdr} interessant. GNSS-SDR ist ein Open Source Framework mit dem sich SDR GNSS Receiver realisieren lassen. Der Schwerpunkt liegt dabei aber im Bildungs- und Forschungsbereich um Algorithmen zu testen und zu vergleichen. Für eine Echtzeit-Positionsbestimmung in einem Embedded System hat GNSS-SDR zu hohe Leistungsforderungen. Mit einem leistungsfähigen PC ist jedoch auch mit dem GNSS-SDR Receiver Echtzeit-Positionsbestimmung möglich.

Ein ähnliches Tool, was für Bildung und Forschung gedacht ist, ist der SoftGNSS Receiver der als Teil von \cite{borre2007software} entwickelt wurde. Der SoftGNSS Receiver ist vollständig in Matlab implementiert und nicht auf Echtzeitverarbeitung ausgelegt. Im Rahmen dieser Arbeit wurde jedoch von dem SoftGNSS Receiver Gebrauch gemacht um die Funktionsfähigkeit des Signalpfads zu testen.