\section{Ähnliche Arbeiten}
Software Defined Radio GPS Empfänger sind nicht neu. Eines der ersten Projekte entstand schon 1994 \cite{vidmar1994diy}. Das diskret aufgebaute HF Frontend ist jedoch noch vergleichsweise aufwändig und wäre schwer in einem kleinen Cubesat unterzubringen. Ein sehr ähnliches Projekt aus dem Jahre 2013 \cite{A_holme_2013} verwendet bereits ein etwas vereinfachtes HF Frontend, welches jedoch immer noch vergleichsweise viel Platz beansprucht.

Seit einigen Jahren sind jedoch relativ günstige GPS Frontend Chips verfügbar, die den kompletten HF Teil, inklusive Mischer und Lokaloszillator und HF/ZF Filtern mit einem AD Wandler vereinen. Damit bietet sich die Möglichkeit, ohne aufwändiges HF Schaltungsdesign, ein GPS Frontend in Schaltungen zu integrieren. Ein solcher GPS Frontend Chip wurde auch im Design des COM Board für den \dscubesat integriert. 

Eine weitere interessante Arbeit ist \cite{Birklykke2010}, die das Ergebnis eines mehrsemestrigen Projekts ist, welches sich speziell mit der Implementierung eines GPS Empfängers für kleine \gls{LEO} Satelliten beschäftigt hat. Der Quellcode des in der Arbeit entwickelten GPS Empfängers ist leider nicht verfügbar.

Es gibt auch mehrere kommerzielle Software Defined GPS Receiver, von denen das SwiftNav Piksi \cite{SwiftNavPiksi} dadurch hervorsteht, dass ein großer Teil der Software unter GPL Lizenz frei verfügbar ist. Ein elementarer Bestandteil, die FPGA Firmware ist allerdings nicht veröffentlicht. Für zukünftige Weiterentwicklung der hier vorgestellten Arbeit sollte jedoch untersucht werden, ob die Open Source Teile des SwiftNav Piksi Projekts hilfreich sein können.

Abseits von Realisierungen für Embedded Systeme ist auch das GNSS-SDR Projekt \cite{gnss-sdr} interessant. GNSS-SDR ist ein Open Source Framework mit dem sich SDR GNSS Receiver realisieren lassen. Der Schwerpunkt liegt dabei aber im Bildungs- und Forschungsbereich um Algorithmen zu testen und zu vergleichen. Für eine Echtzeit-Positionsbestimmung in einem Embedded System hat GNSS-SDR zu hohe Leistungsforderungen. Mit einem leistungsfähigen PC ist jedoch auch mit dem GNSS-SDR Receiver Echtzeit-Positionsbestimmung möglich.

Ein ähnliches Tool, was für Bildung und Forschung gedacht ist, ist der SoftGNSS Receiver der als Teil von \cite{borre2007software} entwickelt wurde. Der SoftGNSS Receiver ist vollständig in Matlab implementiert und nicht auf Echtzeitverarbeitung ausgelegt. Im Rahmen dieser Arbeit wurde jedoch von dem SoftGNSS Receiver Gebrauch gemacht um die Funktionsfähigkeit des Signalpfads zu testen.