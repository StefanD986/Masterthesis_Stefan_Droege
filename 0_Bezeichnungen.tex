Es wurde versucht in dieser Arbeit durchgehend die selben Symbole für Signale mit der selben Bedeutung zu nehmen. Die selben Symbole wurden auch in den Blockschaltbildern verwendet.

Einige generelle Konventionen denen in dieser Arbeit gefolgt wird: 
\begin{itemize}
\item Symbole mit einer Tilde wie z.B. $\gpsin$ stellen Signale auf der Empfängerseite dar. 
\item Großgeschriebene Symbole stellen die Fouriertransformierte des entsprechenden Signals mit Kleinbuchstaben dar.
\item Wenn von einem Signal mehrere Versionen vorliegen wie z.B. im Falle von $\gpslo_I$ und $\gpslo_Q$ wird in einigen Gleichungen das Subscript weggelassen, wenn die Gleichung für alle Versionen gilt.
\end{itemize}
