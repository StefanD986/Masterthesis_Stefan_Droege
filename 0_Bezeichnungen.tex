% Some shortcuts so that everyhwere the same nomenclature is used for signals in this document
Dieser Abschnitt ist eine Übersicht über die verwendeten Symbole in dieser Arbeit. Bei vielen Symbolen kann es außerdem hilfreich sein, die Blockschaltbilder zur Hilfe zu nehmen. 

Einige generelle Konventionen denen in dieser Arbeit gefolgt wird: 
\begin{itemize}
\item Symbole mit einer Tilde wie z.B. $\gpsin$ stellen Signale auf der Empfängerseite dar. 
\item Großgeschriebene Symbole stellen die Fouriertransformierte des entsprechenden Signals mit Kleinbuchstaben dar.
\item Wenn von einem Signal mehrere Versionen vorliegen wie z.B. im Falle von $\gpslo_I$ und $\gpslo_Q$ wird in einigen Gleichungen das Subscript weggelassen, wenn die Gleichung für alle Versionen gilt.
\end{itemize}

\begin{scriptsize}

\begin{description}
\item [$\ifft{\cdot}$] Operator für die Fourier Transformation
\item [$\fft{\cdot}$] Operator für die inverse Fourier Transformation % inverse FFT{}

\item [$\gpsnav$] % output of acquisiton
\item [$\gpsin$] Eingangssignal des GPS Empfängers (bereits heruntergemischt auf eine ZF).% Input signal to GPS receiver
\item [$\gpslo$] Signal des \gls{LO}.% LO signal
\item [$\gpsca$] Signal des Codegenerators. Der index $i$ bezeichnet die Code PRN. % CA code in the receiver
\item [$\gpsmx$] Ausgangssignal des Mixers der das Eingangssignal $\gpsin$ mit $\gpslo$ mischt. % output of carrier mixer
\item [$\gpsxc$] Ausgang des Mischers der $\gpsmx$ mit $\gpsca$ mischt.% output of code correlator
\item [$\gpsaout$] % output of acquisiton
\item [$\gpsdat$] % output of acquisiton
\item [$\gpspllout$] % output of acquisiton

\item [$\gpsTcode$] Dauer eines C/A Codewortes (\SI{1}{\ms} bei GPS)% (time)length of one codeword
\item [$\gpsTchip$] Dauer eines Chips des C/A Codewortes (\SI{\approx977}{\ms} bei GPS)% (time)length of one codeword

\end{description}

\end{scriptsize}