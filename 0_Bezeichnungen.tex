Es wurde versucht in dieser Arbeit für Signale mit der selben Bedeutung im Text und in den Blockschaltbildern durchgehend die selben Symbole zu nehmen.

Einige generelle Konventionen denen in dieser Arbeit gefolgt wird: 
\begin{itemize}
\item Symbole mit einer Tilde wie z.B. $\gpsin$ stellen Signale auf der Empfängerseite dar. 
\item Großgeschriebene Symbole stellen die Fouriertransformierte des entsprechenden Signals mit Kleinbuchstaben dar.
\item Wenn von einem Signal mehrere Versionen vorliegen wie z.B. im Falle von $\gpslo_I$ und $\gpslo_Q$ wird in einigen Gleichungen das Subscript weggelassen, wenn die Gleichung für alle Versionen gilt.
\end{itemize}
