\chapter{Entwurf}
Einleitung zum Entwurf

\section{Anforderungen}
In diesem Abschnitt sollen zunächst die Anforderungen genannt werden, anhand derer sich die Arbeitspakete identifizieren lassen. Das Ziel ist: Der \dscubesat soll in der Lage sein seine eigene Position mittels GPS zu bestimmen.  

\subsection{Anzahl der Kanäle} Wie in Abschnitt \ref{positionsbestimmung} erläutert werden dazu die Daten von mindesten 4 GPS Satelliten benötigt. Daraus ergibt sich die Anforderung, dass der Empfänger mindestens 4 Kanäle zum Tracking der Signale haben muss. Weiterhin soll bei einem Verbindungsabbruch zu einem der 4 Satelliten die Positionsbestimmung nicht abbrechen, weshalb >4 Kanäle erforderlich sind. Außerdem steigt die Genauigkeit der Positionsbestimmung mit mehr Satelliten. Deshalb soll der hier entwickelte GPS Empfänger mindestens 6 Kanäle umfassen.

\subsection{Frontend Parameter}
Die wichtigsten Parameter des Frontend sind die ZF Frequenz - und damit die PLL Parameter - die Einstellung des ZF Filters, und die Parameter des AD Wandlers. An dieser Stelle sei gesagt, dass die Dokumentation des MAX2769 Frontend ICs an vielen Stellen leider sehr lückenhaft ist. Daher mussten die optimalen Einstellungen teils durch Ausprobieren ermittelt werden.

\paragraph{ZF Frequenz und PLL Parameter}
Bei der Erzeugung des LO Signals stehen im Frontend eine Integer PLL oder eine Fractional PLL zur Auswahl. Experimentell wurde bestimmt, dass die Integer PLL weniger Interferenzen verursacht, und die besten Ergebnisse bei einer ZF von etwa \SI{2}{\MHz} erreicht werden können. 

\paragraph{ZF Filter}
Als Konfiguration für das ZF Filter im Frontend wurde ein komplexes Bandpassfilter mit $B=\SI{2.5}{\MHz}$ gewählt. Das komplexe Bandpassfilter ist bei Anwendungen mit einer ZF $\neq 0$ zu bevorzugen, da damit die Spiegelfrequenzen unterdrückt werden.\footnote{Falls die ZF zu \SI{0}{\Hz} gewählt wird sollte als ZF Filter das reelle Tiefpassfilter gewählt werden, um Satellitensignale mit negativer Doppleverschiebung nicht zu unterdrücken.}

\SI{2.5}{MHz} gewählt, da die Nyquist Bandbreite des \gls{CA} kodierten Signals $1/\gpsTchip=\SI{1.023}{\MHz}$ beträgt. Der Einfluss der Filterbandbreite auf die Verschlechterung im Vergleich zu einem Filter mit unendlicher Bandbreite wurde in \cite{itc1982chang} untersucht. Bandbreiten sorgen zwar für eine geringere Verschlechterung, aber wie in  zeigt ist der Einfluss für $B \gpsTchip>2$ gering.

Die Mittenfrequenz des ZF Filters lässt sich auf die gewählte ZF über ein \SI{6}{\bit} Konfigurationswort ($\textrm{FCEN}$) anpassen. Die Berechnung des Konfigurationsworts ist leider nicht im MAX2769 Datenblatt dokumentiert, es konnte aber herausgefunden werden, dass das korrekte Konfigurationswort mit folgender Gleichung ermittelt werden kann: 
\begin{equation}
    f_{cen,2.5} = (64 - \textrm{FCEN'}) \cdot \SI{0.195}{MHz}
\end{equation}
wobei $\textrm{FCEN'}$ das gespiegelte Konfigurationswort $\textrm{FCEN}$ ist (wenn z.B. $\textrm{FCEN}=101100$, dann $\textrm{FCEN}=001101$)

\paragraph{Quantisierung und Samplingrate}
Quantisierung, Abtastung und Bandpassfilterung des Signals im Frontend verschlechtert sich das SNR leicht. Der genaue Zusammenhang wurde in \cite{itc1982chang} untersucht. Daraus lässt sich entnehmen, dass die Verschlechterung vor allem von der Anzahl der Bits $m$ des Quantisierers, schwach von dem Produkt $B\cdot T$ und nur sehr schwach vom $E_b/N_0$ abhängt. dass die Bandbreite des Frontend Filters $B\geq 2/\gpsTchip$ gewählt werden sollte. Weiterhin schreibt \cite{itc1982chang}, dass der Einfluss der Samplingrate abnimmt, 

% 1 bit reduzierte komplexität
% höhere samplingrate als 1/TChip verbessert SNR nur marginal (Chang 2.2)
Bei der Wahl der Frontend Für einen kleinen Pseudoranging Fehler sind hohe Samplingraten des GPS Frontends von Vorteil (siehe \ref{positionsbestimmung}). Allerdings steigt mit höherer Samplingrate auch der Rechenaufwand bei der Acquisition, und der Rohdatenpuffer benötigt mehr Speicher. 

% Tabelle 
% FCEN
% PLL M, N
% Samplingrate
%

LNA Strom
LO Strom
Mixerstrom
Mixpole??
LNA1/2 Auswahl
Antenna bias
Mixer enable
FCEN 27
Filter Bandbreite 2.5 MHz
Filter order 5
Filter bandpass
filter gain high

IQ enable off
gainref 
agcmode_indep
format unsigned
bits_1
drvcfg_cmos
LO enable

Gainin (ignoriert bei agc)
fslowen??
hiloaden off wegen interferenz
adc enable
drven??
fofsten?
filten?
fhipen? off
pgaien
pgaqen

pll vco enable
ivco normal
refout enable
refdiv none
ixtal buff normal
xtalcap 0b10000
ldmux(0)
icp 0.5ma
cptest(0)
intpll

ndiv 1538
rdiv 16

fdiv nicht benutzt

lcnt, mcnt nicht benutzt
clk_serclk

\subsection{Rohdatenpuffer}
Voraussetzung für das Tracking ist die erfolgreiche Acquisition, wofür eine gewisse Menge Rohdaten benötigt werden. Aufgrund der Bitwechsel Problemetik (\ref{DatenmengeAcq}) beträgt die maximal sinnvolle Pufferlänge $2\times\SI{10}{\ms}$. Für den Entwurf wurde sich an der Implementierung in der Simulation in \cite{borre2007software} orientiert, wo ein \SI{10}{\ms} Segment in der zweiten Phase der Acquisition verwendet wird.




 
% Die Voraussetzungen sind dafür: NAV Daten von >= 4 Satelliten
% Also müssen mindestens 4 Satelliten getrackt werden können. (plus Reserve falls Empfang verloren geht)
% Voraussetzung für das Tracking ist die Acquisition. Für die Acquisition werden 2ms Rohdaten benötigt. --> Speicheranforderung. Berechnung durchführen.
% Aus Experimenten mit verschieden Konfigurationen hat sich gezeigt, dass die besten Ergebnisse mit 1 bit sampling und einer ZF von xx Hz erzielt werden können, bei einer Samplingrate von xyz Hz. Das Frontend bietet zwar höhere Quantisierungen, jedoch führt dies offenbar i

% # CHannels
% Samplingrate
% Sample länge

\section{Planung}
Um die Arbeitspakete und Anforderungen zu identifizieren ist es zunächst wichtig zu verstehen wie die Demodulation des GPS Signals prinzipiell abläuft. Dies wird deshalb im ersten Abschnitt beschrieben. Im zweiten Abschnitt werden die Arbeitspakete dann den einzelnen Hardware Komponenten zugeordnet. Da die COM Hardware mit dem DSP und dem FPGA zwei komplexe Datenverarbeitende Komponenten besitzt, die Daten miteinander austauschen wird im dritten Abschnitt der Entwurf der Software Schnittstelle zwischen DSP und FPGA beschrieben. Der letzte Abschnitt beschreibt die Simulation, von der während des Entwurfs und Implementierung viel Gebrauch gemacht wurde, um Konzepte und Ideen zu prüfen.

% Meilensteine:
% Daten von Frontend nehmen, puffern und an DSP Übertragen
%% Kerneltreiber schreiben
%%% SPI, DMA
%% FIFO im FPGA implementieren
%%% EBIU Schnittstelle/State Machine
% Damit Acquisition in Matlab durchführen
% Trackingloop in Matlab simulieren
%% High Level (LUT, sin, cos)
%% Dann Low Level (LFSR, NCO)
% Implementierung Tracking FPGA
% Softwareschnittstelle FPGA<-> DSP
% Implementierung Acquisiton DSP


\paragraph{Entwurf der Parameter für die Acquisition}
Die wichtigen Parameter für die Acquisiton sind die Grenzen des zu durchsuchenden Raums.
Die derzeit aktiven Satelliten haben die Codes mit den PRN 1-32 zugewiesen. Der Bereich der möglichen Codephasen ergibt sich aus der Dauer einer Codeperiode (Codelänge/Chippingrate TODO), und der Sampling Frequenz des Frontend ICs (TODO Formel): N=fs/(fchip/Lcode)=16.369MHz/(1.023Mchip/s/1023) = 16369 MHz *TODO Fußnote: Die Samplingfrequenz im Flugmodell beträgt 16.368 MHz).

Da der Doppler Shift die Hauptursache ist entspricht die Abweichung dem maximal erwarteten Doppler Shift. Wie in Kapitel (todo link) besprochen, ist der erwartete Doppler Shift (TODO +-)45 kHz.

Die Schwelle ab der ein Signal nach dem ersten Schritt der Acquisition als „vorhanden“ bewertet wird wird auf ein Peak-to-Average Ratio von 2,5 festgelegt. Dieser Wert wurde durch Simulation für Signale gefunden, die ein gerade noch ausreichendes SNR haben.




Die entscheidenden Entwurfsparameter für den Carrier- und Code Tracking Loop sind die Parameter Dämpfungsgrad und Bandbreite des Loopfilters. Zusammen bestimmen diese die Schnelligkeit mit der der Regelkreis einrastet und die Rauschbandbreite des Steuersignals für den LO.
Die Werte für den Loopfilter wurden zunächst aus (TODO Referenz Buch) übernommen und anschließend, im Hinblick auf den höheren Dynamikbereich der Dopplerverschiebung auf dem Dragsail Cubesat überprüft. Die Extremwerte der Änderungsrate der Dopplerverschiebung wurden bereits in Abschnitt (TODO Link) bestimmt, und sind noch einmal in Tabelle (TODO Tabelle max Dopplershift, Dopplershift Änderung) dargestellt.
In der Simulation ergab sich, dass die Werte auch für den hier vorliegenden Anwendungsfall optimal sind. Tabelle TODO stellt die verwendeten Loop Filter Werte dar.

\paragraph{Filter Topologie und Festkommaentwurf}
Im Hinblick auf die Echtzeitfähigkeit und Stabilität des Regelkreises muss auf die Performanz und die begrenzte Rechengenauigkeit, der Berechnungen geachtet werden. Während in der Matlab Simulation, die nicht Echtzeitfähig ist, mit Fließkommawerten doppelter Genauigkeit gerechnet wird, soll in der Implementierung mit Festkommawerten gerechnet werden.
Dafür sind die Filterkoeffizienten zu quantisieren, und die Filtertopologie so anzupassen, dass der durch die Festkommazahl darstellbare Dynamikbereich optimal ausgenutzt wird. Die Details des Festkomma Entwurfs werden im Folgenden besprochen.

%\subsubsection{NAV Daten Auswertung}
%Präambel usw

\subsection{Arbeitsteilung FPGA/DSP}
% Acquisition im DSP
% Erfordert große Mengen Speicher und hoher Rechenaufwand
% Tracking im FPGA. Tracking erfordert hohe Parallelität bei moderatem Rechenaufwand und hat harte Echtzeitanforderungen.
% Softcore Prozessor. Auswahl. Aufgabe

\subsection{Softwareschnittstelle FPGA/DSP}
%Character Device Treiber
%Je Kanal ein FIFO

\subsection{Matlab Simulation}
%Trackingloop in Simulink

%High Level Simulation sukzessive angepasst an Implementierung in Hardware
%Acquisition Skripte
% MAX2769 Bandpassfilter mittenfrequenz
% 1bit vs >1bit: Große Störungen bemerkt bei >1bit