%%%%%%%%%%%%%%%%%%%%%%%%%%%%%%% Abbildungen %%%%%%%%%%%%%%%%%%%%%%%%%%%%%%%%%%%%%%%
\usepackage[small,bf,format=plain]{caption}
%\usepackage{caption}[2008/08/24]							% Abbildungsunterschrift verändern
	%\renewcommand{\captionfont}{\footnotesize}				% Setzt die Bildunterschrift kleiner (funktioniert, ist aber kleiner als small bei {caption})
	%\renewcommand{\figurename}{{Abb.}}						% Umbennenung Bildunterschrift von Abbildung (Standard) zu Abb., funktioniert hier nicht?!?!?!?!?!?!?!?!?!?!
%\usepackage{caption}
\usepackage{subcaption}									% mehrere Bilder in einer Abbildung	
\usepackage{wrapfig}										% textumfließende Graphik

%% ########## Settings and Shortcuts for Graphics:
%% Search path for images
\graphicspath{{./figures/}}


% Shortcut to include images using the graphicx package:
% \FGimg{filename}{caption}{width}
% filename is at the same time used as label
\newcommand{\FGimg}[4][TODO change short caption]
{    \vspace{2mm}
    \begin{figure}
    \ffigbox[\FBwidth]
    {\caption[#1]{#3}\label{#2}}
    {\includegraphics[width=#4]{#2}}
    \end{figure}
}

% Another Shortcut to include more other graphics. Insert command to include graphic as parameter #1:
% \FG{command}{caption}{label}
\newcommand{\FG}[4][TODO change short caption]
{
    \vspace{2mm}
    \begin{figure}
    \ffigbox[\FBwidth]
        {\caption[#1]{#3}\label{#4}}
        {#2}
    \end{figure}
}


% Makro zum Verweisen auf Bilder
% \FR{RAR.jpg}  Bsp.
% \renewcommand{\figurename}{Abb.}
\newcommand{\FR}[1]{Abb.~\ref{#1}}

% Makro zum verweisen auf Tabellen
\newcommand{\TR}[1]{Tab.~\ref{#1}}


%\renewcommand{\arraystretch}{1.1} % Größerer Abstand zwischen Tabellenzeilen. Default = 1.0
\definecolor{light-gray}{gray}{0.95}
\definecolor{semigray}{gray}{0.8}
\definecolor{IntLinkColor}{gray}{0.90}

\hypersetup{
%    bookmarks=true,         % show bookmarks bar?
%    unicode=false,          % non-Latin characters in Acrobat’s bookmarks
%    pdftoolbar=true,        % show Acrobat’s toolbar?
%    pdfmenubar=true,        % show Acrobat’s menu?
%    pdffitwindow=true,     % window fit to page when opened
    pdfstartview={FitV},    % fits the width of the page to the window
%    pdftitle={???TODO},    % title
    pdfauthor={Stefan Dröge},     % author
%    pdfsubject={Subject},   % subject of the document
    pdfcreator={Stefan Dröge},   % creator of the document
    pdfproducer={Stefan Dröge}, % producer of the document
    pdfkeywords={Thesis} {Masterarbeit} {GPS} {GNSS} {Software Defined Radio}, % list of keywords
%    pdfnewwindow=true,      % links in new window
    colorlinks=false,       % false: boxed links; true: colored links
    linkbordercolor=IntLinkColor,          % color of internal links
%    citecolor=blue,        % color of links to bibliography
%    filecolor=magenta,      % color of file links
%    urlcolor=cyan           % color of external links
}

\usepackage{floatrow}
\floatsetup[table]{style=plaintop}

\newenvironment{myitemize}{\begin{itemize}\itemsep -5pt}{\end{itemize}}
\newenvironment{mydescription}{\begin{description}\itemsep -5pt}{\end{description}}
\newenvironment{myenumerate}{\begin{enumerate}\itemsep -5pt}{\end{enumerate}}

\usepackage{listings}
\usepackage[usenames,dvipsnames]{xcolor}
\lstset{ %
    language=VHDL,
    basicstyle=\footnotesize\ttfamily,
    %prebreak=\mbox{$\hookleftarrow$},
    numberblanklines=true, 
    breaklines=true, 
    numbers=left, 
    stepnumber=1, 
    frame=tlRB,
    keywordstyle=\color{purple}\bfseries,
    commentstyle=\color{ForestGreen},
    identifierstyle=\color{Brown},
    stringstyle=\color{blue},
    rulecolor=\color{black},
    lineskip={-1.5pt}
    %  backgroundcolor=\color{white},   % choose the background color; you must add \usepackage{color} or \usepackage{xcolor}
    %  basicstyle=\footnotesize,        % the size of the fonts that are used for the code
    %  breakatwhitespace=false,         % sets if automatic breaks should only happen at whitespace
    %  breaklines=true,                 % sets automatic line breaking
    %  captionpos=b,                    % sets the caption-position to bottom
    %  commentstyle=\color{mygreen},    % comment style
    %  deletekeywords={...},            % if you want to delete keywords from the given language
    %  escapeinside={\%*}{*)},          % if you want to add LaTeX within your code
    %  extendedchars=true,              % lets you use non-ASCII characters; for 8-bits encodings only, does not work with UTF-8
    %  frame=single,                    % adds a frame around the code
    %  keywordstyle=\color{blue},       % keyword style
    %  language=Octave,                 % the language of the code
    %  morekeywords={*,...},            % if you want to add more keywords to the set
    %  numbers=left,                    % where to put the line-numbers; possible values are (none, left, right)
    %  numbersep=5pt,                   % how far the line-numbers are from the code
    %  numberstyle=\tiny\color{mygray}, % the style that is used for the line-numbers
    %  rulecolor=\color{black},         % if not set, the frame-color may be changed on line-breaks within not-black text (e.g. comments (green here))
    %  showspaces=false,                % show spaces everywhere adding particular underscores; it overrides 'showstringspaces'
    %  showstringspaces=false,          % underline spaces within strings only
    %  showtabs=false,                  % show tabs within strings adding particular underscores
    %  stepnumber=2,                    % the step between two line-numbers. If it's 1, each line will be numbered
    %  stringstyle=\color{mymauve},     % string literal style
    %  tabsize=2,                       % sets default tabsize to 2 spaces
    %  title=\lstname                   % show the filename of files included with \lstinputlisting; also try caption instead of title
}

\newcolumntype{P}[1]{>{\raggedright\arraybackslash}p{#1}}

\newcommand{\ThesisTitle}{Ent\-wick\-lung eines Soft\-ware De\-fined GNSS Emp\-fäng\-ers für den Drag\-sail-Cube\-sat}

\interfootnotelinepenalty=10000

% ####### Glossary Settings ###########
% Better for long glossary description entries
\newglossarystyle{clong}{%
 \renewenvironment{theglossary}%
     {\begin{longtable}{p{.3\linewidth}p{\glsdescwidth}}}%
     {\end{longtable}}%
  \renewcommand*{\glossaryheader}{}%
  \renewcommand*{\glsgroupheading}[1]{}%
  \renewcommand*{\glossaryentryfield}[5]{%
    \glstarget{##1}{##2} & ##3\glspostdescription\space ##5\\}%
  \renewcommand*{\glossarysubentryfield}[6]{%
     & \glstarget{##2}{\strut}##4\glspostdescription\space ##6\\}%
  %\renewcommand*{\glsgroupskip}{ & \\}%
}
\renewcommand*{\glossaryname}{Glossar} % German Glossary Chapter title


%##################################
\setlist[enumerate]{topsep=0pt,itemsep=-1ex,partopsep=1ex,parsep=1ex}

\sloppy % Accepts ugly paragraphs, but helps with overful hboxes (This is mostly needed in the implementation chapter, because the inline listings cannot break whole words, like variable names)
\endinput