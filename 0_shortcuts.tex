% Some shortcuts so that everyhwere the same nomenclature is used for signals in this document
\newcommand{\gpsnav}{d} % output of acquisiton
\newcommand{\gpsin}{\tilde{s}} % Input signal to GPS receiver
\newcommand{\gpslo}{\tilde{f}} % LO signal
\newcommand{\gpsca}{\tilde{g_i}} % CA code in the receiver
\newcommand{\gpsCA}{\tilde{G_i^*}} % FFT CA code
\newcommand{\gpsmx}{\tilde{u}} % output of carrier mixer
\newcommand{\gpsMX}{\tilde{U}} % FFT output of carrier mixer
\newcommand{\gpsXC}{\tilde{V}} % FFT output of code correlator
\newcommand{\gpsxc}{\tilde{v}} % output of code correlator
\newcommand{\gpsaout}{\tilde{a}} % output of acquisiton
\newcommand{\gpsdat}{\tilde{d}} % output of acquisiton
\newcommand{\gpspllout}{\tilde{y}} % output of acquisiton

\newcommand{\gpsTcode}{{T}_{Code}} % (time)length of one codeword
\newcommand{\gpsTchip}{{T}_{Chip}} % (time)length of one code chip

\newcommand{\ifft}[1]{\mathcal{F}^{-1}\left\{#1\right\}} % FFT{}
\newcommand{\fft}[1]{\mathcal{F}\left\{#1\right\}} % inverse FFT{}

\DeclareMathOperator*{\argmax}{arg\,max} % argmax

\newcommand{\comboard}{\emph{COM Board }}
\newcommand{\comfpga}{\emph{XCS6LX9} }
\newcommand{\dscubesat}{\emph{Dragsail-Cubesat} }