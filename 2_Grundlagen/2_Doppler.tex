\section{Dopplerverschiebung}
Da das GPS bekanntlich ein satellitengestütztes Navigationssystem ist werden die Navigationssignale von Sendern in einer Umlaufbahn ausgesendet. Dabei bewegen sich die Satelliten mit einer hohen Bahngeschwindigkeit was eine Frequenzverschiebung durch den Dopplereffekt zur Folge hat. Wie im Abschnitt \ref{Grundlagen_Acquisition} beschrieben ist die maximal mögliche Dopplerverschiebung ein wichtiger Parameter bei der Acquisition. Deshalb wird in diesem Abschnitt eine Gleichung hergeleitet, mit der die maximal mögliche Dopplerverschiebung und die maximal mögliche Änderungsrate der Dopplerverschiebung berechnet werden kann.

Der Empfänger nimmt aufgrund der begrenzten Ausbreitungsgeschwindigkeit der elektromagnetischen Wellen das Signal auf einer anderen Frequenz war. Dieser relativistische Doppler-Effekt ist durch die folgende Gleichung beschrieben:
\begin{equation}
    f_r=f_s \sqrt{\frac{c+v_{rel}}{c-v_{rel}}}
\end{equation}

Hier bezeichnet $v_{rel}$ die relative Geschwindkeit von Empfänger und Sender zueinander ($v_{rel}>0$ wenn sich Empfänger und Sender annähern). $c$ ist die Lichtgeschwindigkeit, $f_s$ die Sendefrequenz, $f_r$ die vom Empfänger wahrgenommene Empfangsfrequenz.

Mit der Dopplerverschiebung $f_d$ sei die Differenz zwischen wahrgenommener Empfangsfrequenz und Sendefrequenz gemeint:

\begin{equation}
\label{EqDoppler}
    f_d=f_r-f_s=f_s \left( \sqrt{\frac{c+v_{rel}}{c-v_{rel}}}-1 \right)
\end{equation}

\subsubsection{Maximale Dopplerverschiebung für den Dragsail Cubesat}
Im folgenden wird eine Formel für die  maximale Dopplerverschiebung hergeleitet. 
Dazu werden einige vereinfachende Annahmen gemacht \footnote{In \cite{Birklykke2010} wurden ähnliche Berechnungen angestellt, und mit Messungen von existierenden Cubesat Missionen verglichen. Dabei hat sich gezeigt, dass die durch die Näherungen eingeführten Fehler vernachlässigbar sind.}: Es wird angenommen, dass sich Sender und Empfänger jeweils auf Kreisbahnen mit Radius $r_A$ bzw $r_B$ bewegen. Man kann sich einfach klarmachen, dass die größte Dopplerverschiebung auftritt, wenn die Kreisbahnen in einer Ebene liegen, und entgegengesetzt durchlaufen werden. Das Szenario ist in \FR{DopplerSzenario} veranschaulicht.

\FG{\definecolor{ffqqqq}{rgb}{1.,0.,0.}
\definecolor{wwzzqq}{rgb}{0.4,0.6,0.}
\definecolor{qqzzqq}{rgb}{0.,0.6,0.}
\definecolor{qqwuqq}{rgb}{0.,0.39215686274509803,0.}
\definecolor{yqyqyq}{rgb}{0.5019607843137255,0.5019607843137255,0.5019607843137255}
\definecolor{xdxdff}{rgb}{0.49019607843137253,0.49019607843137253,1.}
\definecolor{wwccff}{rgb}{0.4,0.8,1.}
\begin{tikzpicture}[line cap=round,line join=round,>=triangle 45,x=1.0cm,y=1.0cm]
\clip(-7.985271985498141,-8.016880185904501) rectangle (18.560876233968703,15.032597553131351);
\draw [color=wwccff,fill=wwccff,fill opacity=1.0] (0.,0.) circle (6.371cm);
\draw [shift={(0.,0.)},color=qqwuqq,fill=qqwuqq,fill opacity=0.1] (0,0) -- (19.68118276735286:1.0704092023978566) arc (19.68118276735286:66.4009838652488:1.0704092023978566) -- cycle;
\draw [dash pattern=on 9pt off 9pt] (0.,0.) circle (7.571cm);

\draw [dash pattern=on 9pt off 9pt] (0.,0.) circle (14.571cm);
\draw [->,line width=1.0pt] (0.,0.) -- (3.0309233916096785,6.937834272609377);

\draw [->,line width=1.0pt] (0.,0.) -- (13.719779719976211,4.907309388588564);

\draw [->,line width=1.0pt,color=yqyqyq] (3.0309233916096785,6.937834272609377) -- (13.719779719976213,4.907309388588564);

\draw [->,line width=1.6pt,color=qqzzqq] (3.0309233916096785,6.937834272609377) -- (7.278094532253406,5.082377677289892);

\draw [->,line width=1.6pt,color=wwzzqq] (13.719779719976211,4.907309388588564) -- (12.984984504069516,6.961638493041415);

\draw [->,line width=0.8pt] (0.,0.) -- (1.,0.);

\draw [->,line width=0.8pt] (0.,0.) -- (0.,1.);

\draw [->,line width=1.6pt,color=ffqqqq] (13.719779719976211,4.907309388588564) -- (8.19447807309766,5.956931737420583);
\begin{scriptsize}
\draw[color=wwccff] (-2.7759472004952386,5.006431539711732) node {$Erde$};
\draw [fill=xdxdff] (3.0309233916096785,6.937834272609377) circle (6.5pt);
\draw[color=xdxdff] (2.1479351305349015,6.897487762883404) node {$A$};
\draw [fill=xdxdff] (13.719779719976211,4.907309388588564) circle (6.5pt);
\draw[color=xdxdff] (14.600362185096634,4.756669397028682) node {$B$};
\draw[color=black] (1.7554517563223542,3.2580965409303757) node {$\vv{a}$};
\draw[color=black] (9.03423433262778,2.7585722555642738) node {$\vv{b}$};
\draw[color=yqyqyq] (6.322531019886542,6.897487762883404) node {$\vv{AB}$};
\draw[color=qqwuqq] (1.5770502225893779,1.1172781750756529) node {$\theta$};
\draw[color=qqzzqq] (6.964776541325256,4.970751233614155) node {$\vv*{v}{A}$};
\draw[color=wwzzqq] (13.815395436671539,7.0402089872737195) node {$\vv*{v}{B}$};
\draw[color=black] (0.934804701150664,-0.23857345663233848) node {$\hat{e}_x$};
\draw[color=black] (-0.6351287956995255,0.9388766445877592) node {$\hat{e}_y$};
\draw[color=ffqqqq] (10.140323841772231,5.3632346006875204) node {$\vv*{v}{rel}$};
\end{scriptsize}
\end{tikzpicture}
}{Szenario zur Berechnung der maximalen Dopplerverschiebung (nicht maßstabsgetreu). Der Cubesat (Punkt $A$) befindet sich in einem \emph{Low-Earth-Orbit}, der GPS Satellit (Punkt $B$) in einem \emph{Medium Earth Orbit}.}{DopplerSzenario}

$\vv{a}$ gibt die Position des Cubesat bezogen auf den Erdmittelpunkt an:
\begin{equation}
    \vv{a}(t)=\begin{bmatrix} 
        A_x(t)\\ 
        A_y(t) 
    \end{bmatrix} = 
    r_A \cdot \begin{bmatrix} 
        \cos(\theta_A(t))\\ 
        \sin(\theta_A(t)) 
    \end{bmatrix}
\end{equation}

und $\vec{b}$ die Position des GPS Satelliten:

\begin{equation}
    \vv{b}(t)=\begin{bmatrix} 
        B_x\\ 
        B_y 
    \end{bmatrix} = 
    r_B \cdot \begin{bmatrix} 
        \cos(\theta_B(t))\\ 
        \sin(\theta_B(t)) 
    \end{bmatrix}
\end{equation}

wobei $\theta_{A,B}(t)=\omega_{A,B} \cdot t$ mit der Winkelgeschwindigkeit $\omega_{A,B}$. Desweiteren sei $\theta = \theta_B-\theta_A$.

Der Betrag der Winkelgeschwindigkeit $\omega$ eines Flugkörpers auf einem Kreisorbit mit Radius $r$ lässt sich über das Kräftegleichgewicht von Erdbeschleunigungskraft $F_g$ und Zentrifugalkraft $F_{Zf}$ berechnen:

\begin{eqnarray}
    \vec{F}_g &=& \vec{F}_{Zf} \\
    m \cdot (-\hat{e}_r) \frac{G M}{r^2} &=& m \cdot \omega^2 \cdot r \cdot \hat{e}_r\\
    \omega &=& \sqrt{\frac{G M}{r^3}}
\end{eqnarray}

$G$ ist hierbei die universelle Gravitationskonstante, $M$ die Erdmasse, $m$ die Masse des Flugkörpers. Davon lässt sich die der Betrag der Bahngeschwindigkeit $v=r \cdot \omega$ ableiten. Die Bahngeschwindigkeit $\vv*{v}{A}$ ist die Ableitung des Ortsvektors $\vv{a}$ nach der Zeit:

\begin{equation}
    \vv*{v}{A}(t)=\frac{\partial}{\partial t}\vv{a}(t)
    = \omega_A \cdot \begin{bmatrix} 
        \sin(\theta_A(t))\\ 
        -\cos(\theta_A(t)) 
    \end{bmatrix}
\end{equation}

Analog ergibt sich $\vv{b}$ wobei hier die im Gegensatz zu $A$ entgegengesetzte Drehrichtung zu vertauschten Vorzeichen führt:
\begin{equation}
    \vv*{v}{B}(t)=\frac{\partial}{\partial t}\vv{b}(t)
    = \omega_B \cdot \begin{bmatrix} 
        -\sin(\theta_B(t))\\ 
        \cos(\theta_B(t)) 
    \end{bmatrix}
\end{equation}

Für die Dopplerverschiebung ist die Relativgeschwindigkeit in Ausbreitungsrichtung $v_{rel}$ der beiden Flugkörper gesucht. Diese kann mit 
\begin{eqnarray}
    v_{rel} &=&(\vv*{v}{A}-\vv*{v}{B} ) \cdot \frac{\vv{AB}}{|\vv{AB}|}\\
     v_{rel} &=& \frac{(r_B v_A + r_A v_B) \cdot \sin{(\overbrace{\theta_A-\theta_B}^{-\theta}})}{\sqrt{r_A^2-2 r_A r_B \cos(\theta_A-\theta_B)+r_B^2}}
\end{eqnarray}

bestimmt werden. Die Maxima und Minima dieser Funktion sind
\begin{eqnarray}
    \theta_{v_{rel,max}}&=&\frac{(4n+1)\pi}{2}-\arcsin\left(\frac{r_A}{r_B}\right)\\
    \theta_{v_{rel,min}}&=&\frac{(4n-1)\pi}{2}+\arcsin\left(\frac{r_A}{r_B}\right)
\end{eqnarray}
mit $n\in \mathbb{N}$ und $r_A<r_B$. Anschaulich sind dies genau die Punkte an denen der $\vv{v}{A}$ in Richtung des GPS Satelliten zeigt. Damit ergibt sich nach einigen Umformungen
\begin{equation}
    v_{rel,max}=\frac{r_A}{r_B}\cdot \sqrt{\frac{G \cdot M}{r_B}} + \sqrt{\frac{G\cdot M}{r_A}}
\end{equation}
Eingesetzt in Gl.\ref{EqDoppler} erhält man folgende Gleichung für die maximale Dopplerverschiebung:
\begin{equation}
	f_{d,max} = f_S \cdot \left(\sqrt{\frac{c+\left(\frac{r_A}{r_B}\cdot \sqrt{\frac{G \cdot M}{r_B}} + \sqrt{\frac{G\cdot M}{r_A}}\right)}{c-\left(\frac{r_A}{r_B}\cdot \sqrt{\frac{G \cdot M}{r_B}} + \sqrt{\frac{G\cdot M}{r_A}}\right)}}-1\right)
\end{equation}
\FR{maxdoppler.png} stellt die maximale Dopplerverschiebung des GPS Signals in Abhängigkeit von der Höhe des Kreisorbits über Grund dar.

\FGimg{maxdoppler.png}{Maximale Dopplerverschiebung des GPS L1 Signals in Abhängigkeit der Höhe des Orbits (über Grund)}{.9\textwidth}

In \FR{doppler.png} ist exemplarisch die Dopplerverschiebung auf dem Weg des Cubesat auf einem Orbit mit der Höhe \SI{120}{\km} dargestellt. 

\FGimg{doppler.png}{Dopplerverschiebung des \SI{1575,42}{\MHz} GPS Signals für einen LEO Satelliten auf einem Kreisorbit mit $r_A=\SI{120}{\km}+r_E$ in Abhängigkeit des Winkels $\theta$ zwischen Cubesat GPS Satellit ($r_E$ bezeichnet den Erdradius). Der Funkschatten der Erde ist nicht berücksichtigt worden.}{.9\textwidth}

Es ist noch die maximale Änderungsrate der Dopplerverschiebung  $\dot f_{d,max}=max\left(\frac{\partial f_d}{\partial t}\right)$ interessant.  Wie man in \FR{doppler.png} sehen kann tritt die Maximale Änderung bei $\theta=0$ auf. Ohne Beschränkung der Allgemeinheit kann die Forderung $\theta(t=0)\overset{!}{=}0$ gemacht werden. Damit ergibt sich $\dot f_{d,max}$ nach einigen Umformungen zu
\begin{equation}
	\dot f_{d,max}= \frac{G \cdot  M \cdot  f_S \cdot  (r_B^3-r_A^3)}{c \cdot  r_A^2 \cdot  r_B^2\cdot  (r_B-r_A)}
\end{equation}
mit $r_B>r_A$.

Tabelle \ref{TabDoppler} gibt die Zahlenwerte für ein Beispiel mit einer Orbithöhe eines Cubesat von \SI{400}{km} an.

\begin{table}[htbp]
    \ttabbox
    {
        \caption{Zusammenfassung der Auswirkungen des Dopplereffekts auf den Empfang des GPS Signals im L1 Band für den Dragsail Cubesat. Die Ergebnisse wurden mit $r_A=r_E+\SI{400}{\km}$ und $r_B=r_E+\SI{20000}{\km}$ berechnet ($r_E$ ist der Erdradius)}
        \label{TabDoppler}
    }
    {
        \rowcolors{2}{light-gray}{White}
    \begin{tabular}{c c p{5.5cm}}
        \toprule
        Name             & Wert & Beschreibung \\
        \midrule
        $f_{d,max}$ & \SI{45.846}{\kHz} &  Maximale Dopplerverschiebung des GPS Signals\\
        $\dot f_{d,max}$ & \SI{61}{\Hz/\s}& Maximale Änderungsrate der Dopplerverschiebung \\
        \bottomrule
    \end{tabular}
}
\end{table}
