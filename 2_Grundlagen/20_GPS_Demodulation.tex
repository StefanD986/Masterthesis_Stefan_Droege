\section{GPS Empfänger}
In diesem Abschnitt wird die prinzipielle Funktion eines Empfängers für das GPS Signal erklärt. Es wird hier davon ausgegangen, dass das von der Antenne kommende Signal bereits auf eine Zwischenfrequenz $f_{ZF}$ herunter gemischt wurde und digitalisiert wurde. Damit gestaltet sich die Demodulation eigentlich einfach: Es werden die Schritte der Modulation erneut durchgeführt: Multiplikation mit dem (ZF-)Träger und Multiplikation mit dem CA Code (siehe \FR{MoDem.png}). Am Ausgang des Demodulators liegt dann das ursprüngliche Informationssignal an. Im Detail ist der Prozess jedoch komplexer und kann in drei Schritte aufgeteilt werden, die im Folgenden beschrieben werden.

\subsection{Acquisition}
\label{Grundlagen_Acquisition}

Die zuvor beschriebene Demodulation hängt von mehreren, zunächst unbekannten, Variablen ab. Diese Unbekannten zu bestimmen ist Aufgabe der Acquisition.
Wie in Abschnitt \ref{basics_cdma} beschrieben wird zum Multiplexen der Signale der über 30 Satelliten das Signal CDMA kodiert, wobei jeder aktive Satellit einen anderen Code verwendet. Deshalb und weil der Signalpegel beim Empfang unter dem thermischen Rauschen liegt, ist aus den Rohdaten zunächst nicht ersichtlich welche Satelliten überhaupt in Sicht sind. Die erste zu bestimmende Unbekannte ist also welche GPS Satelliten überhaupt in Sicht sind.

Die zweite Unbekannte ist die \gls{Codephase}: Wie beschrieben (\FR{GoldcodeAKF.png}) haben die verwendeten Codes eine geringe Autokorrelation für Verschiebungen (\emph{Codephase}) ungleich Null. Wenn also bei der Demodulation die Code Replika eine zu große Verschiebung aufweist wird am Ausgang des Demodulators lediglich Rauschen zu sehen sein.
Die dritte Unbekannte ist die exakte (ZF-)Trägerfrequenz. Die Trägerfrequenz ist zwar spezifiziert, und die ZF aus dem Systemdesign bekannt, aber mehrere Faktoren sorgen in der Realität für Abweichungen: Zum Einen sorgt der in Kapitel \ref{Dopplereffekt} beschriebene Dopplereffekt für eine zeitabhängige Frequenzverschiebung. Weitere Faktoren sind Jitter und Drift des Quarzoszillators auf der Empfängerseite, die ebenfalls Zeit- und Temperaturabhängig sind. Da der Acquisition Algorithmus nicht zwischen den Ursachen der Trägerfrequenzabweichung unterscheidet und die Abweichung größtenteils durch die Doppler Verschiebung bestimmt wird, wird im Weiteren nur von der Bestimmung des Dopplershifts gesprochen. 

Es ist zu betonen, dass alle drei genannten Unbekannten gleichzeitig gefunden werden müssen. Denn wenn zum Beispiel die korrekte Trägerfrequenz bekannt ist, aber die Codephase oder PRN falsch ist, wird der Demodulator lediglich wiederum ein Rauschsignal liefern. Es wird also ein Korrelationsmaximum im dreidimensionalen Raum aus PRN, Frequenz, und Codephase gesucht.

\subsubsection{Suchalgorithmus}
Es gibt verschiedene Möglichkeiten zur Suche. Die einfachste, aber auch langsamste Methode ist die serielle Suche. Dabei wird das Eingangssignal $\gpsin$ punktweise mit einer Kopie des Codes $\gpsca$ und des Trägers $\gpslo$ multipliziert und dann die Energie des Signals bestimmt, wobei nacheinander verschiedene Kombinationen von \gls{PRN}, Dopplerverschiebung und Codephase ausprobiert werden. Aufgrund der großen Anzahl möglicher Kombinationen und Rechenoperationen ist dies jedoch äußerst zeitaufwändig.

Effizienter sind Methoden die sich die Fourier Transformation zunutze machen, um die aufwändige Berechnung der Kreuzkorrelation zu vereinfachen. Diese Suche muss zwar immer noch für jede PRN ausgeführt werden, aber der Rechenaufwand bei den weiteren Schritten ist deutlich geringer. 
In dieser Arbeit wird eine zweischrittige Methode aus paralleler Codephasen Suche und paralleler Frequenzsuche verwendet. Im ersten Schritt wird der Dopplershift lediglich grob bestimmt (\SI{\pm500}{\Hz}), dafür die Codephase aber mit hoher Genauigkeit bestimmt. Im zweiten Schritt ist dann die Codephase bekannt, und es kann eine parallele Suche im Frequenzbereich durchgeführt werden um den Dopplershift genauer zu bestimmen. 

\FGimg[Acquisition Aktivitätsdiagram]{acquisition_activity.png}{Die Acquisition wird nacheinander für alle PRN durchgeführt. Um dem Problem eines möglichen Bitwechsels vorzugreifen wird die Acquisition für zwei aufeinanderfolgende Segemente durchgeführt, von denen für die weitere Verarbeitung nur das mit der höheren Korrelation ausgewählt wird.}{0.8\imgmaxwidth}

\FGimg[Blockschaltbild parallele Codephasensuche]{parallelCodephase.png}{Bei der parallelen Codephasensuche müssen nur alle möglichen Dopplerverschiebungen seriell durchsucht werden. Die Bestimmung der Codephase kann sehr einfach durch eine Maximumsuche nach der Multiplikation im Frequenzbereich und anschließende Rücktransformation bestimmt werden.}{0.9\imgmaxwidth}

\FGimg[Blockschaltbild parallele Frequenzsuche]{parallelfreqspace.png}{Bei der parallelen Frequenzbereichsuche müssen nur die möglichen Codephasen seriell durchsucht werden. Die Bestimmung der Dopplerverschiebung erfolgt durch eine Maximumsuche im Frequenzbereich.}{0.6\imgmaxwidth}

\paragraph{parallele Codephasensuche}
Bei der parallelen Codephasen Suche werden Kopien des Trägers $\gpslo$ (jeweils I und Q Anteil) für den zu durchsuchenden Frequenzbereich $\textbf{D}$ in einem groben Raster (z.B. \SI{500}{\Hz}) generiert. Außerdem wird für jede PRN die Fouriertransformierte des Codes $\gpsCA$ berechnet. Nacheinander wird dann für jede Trägerfrequenz das Eingangssignal mit dem Träger multipliziert:
\begin{equation}
	\gpsmx=\gpsin(t) \cdot (\gpslo_I(t) + j \gpslo_Q(t))
\end{equation}

Im Unterschied zur seriellen Suche wird dann das Signal  fourier-transformiert:
\begin{equation}
	\gpsMX=\fft{\gpsmx}
\end{equation}

$\gpsMX$ wird dann mit $\gpsCA$ punktweise multipliziert und anschließend zurücktransformiert: 
\begin{equation}
	\gpsxc=\ifft{\gpsMX \cdot \gpsCA}
\end{equation}

Im Ergebnisvektor $\gpsxc$ wird dann eine Suche nach dem Maximum durchgeführt und 
\begin{equation}
	\gpsxc_{k,\max}=\max(\gpsxc)
\end{equation}
und
\begin{equation}
     l_{\max}=\argmax_{x\in \textbf{L}}(\gpsxc)
\end{equation} 
gespeichert\footnote{$\textbf{L}=[0, N-1]$ und $N$ gleich der Anzahl der Samples pro Codewort}. Dieser Vorgang wird für alle Dopplerverschiebungen $d\in \textbf{D}$ wiederholt, und anschließend das globale 
\begin{equation}
	\hat{\gpsxc}_{l,\max}=\max(\gpsxc)
\end{equation}
und
\begin{equation}
	\left[l_{\max}, d_{\max}\right]=\argmax_{x,\in \bf{L} \times \bf{D}}(\gpsxc)
\end{equation}
gesucht. Der Vektor $\left[l_{\max}, d_{\max}\right]$ gibt dann die genaue Codephase und den groben Dopplershift für diese PRN an der Stelle des globalen Korrelationsmaximums $\hat{\gpsxc}_{l,\max}$ an. 

Das Finden des Maximums ist allerdings noch nicht hinreichend um sagen zu können, dass der Satellit in Sicht ist, denn es existiert immer ein Maximum. Um Entscheiden zu können ob ein Satellit in Sicht ist, muss deshalb überprüft werden wie hoch das Korrelationsmaximum relativ gesehen ist, die sogenannte \emph{Peak Metric}. Dazu gibt es verschiedene Möglichkeiten. Es kann zum Beispiel das \emph{Peak to Average Ratio} bestimmt werden. Eine weitere Möglichkeit ist, außerhalb eines gewissen Bereichs um das globale Maximum, nach einem zweiten lokalen Maximum zu suchen und das Verhältnis der beiden Maxima zu bestimmen.
Wenn dies über einem im Design festgelegten Wert liegt, wird die PRN zu der Liste der Satelliten in Sicht hinzugefügt. Dieser Vorgang wird für alle PRN wiederholt. 

Für jede der PRN in Sicht wird dann der genaue Dopplershift mit einer parallelen Frequenzbereich Suche bestimmt. Dazu wird das Eingangssignal mit dem Code (dessen Codephase nun bekannt ist) punktweise multipliziert, und somit entspreizt. Damit besteht das Signal nur noch aus dem (sehr niederfrequenten) Datensignal $\gpsdat$ und dem ZF-Trägersignal mit unbekannter Frequenz:
\begin{equation}
	\gpsdat \cos(\omega_x \cdot t) = \gpsin \cdot \gpsca
\end{equation}
Das Ergebnis wird anschließend Fourier transformiert und das Maximum im Frequenzbereich gesucht, wodurch $\omega_x$ und damit der Dopplershift bestimmt wird.

\paragraph{Datenmenge für die Acquisition} \label{DatenmengeAcq}
Prinzipiell ist bei der Acquisition die Erfolgswahrscheinlichkeit größer je länger das Datensegment ist, mit dem die Kreuzkorrelation geprüft wird. Auf der anderen Seite steigt auch der Rechenaufwand für längere Segmente. Außerdem darf während des gesamten Datensegments kein Bitwechsel vorkommen, da dies sonst in einem kleineren Korrelationswert resultieren würde. Dies folgt aus der Dreiecksungleichung:
\begin{align}
   |(\gpsin(t) \star \gpsca)(\tau)|  &= \left|\int_0^T{\gpsin(t) \cdot \gpsca(\tau -t) \textrm{d}t}\right| \\  
&= \Biggl|\int_0^{t_1}{\gpsin_+(t) \cdot \gpsca(\tau -t) \textrm{d}t}\\
     &\phantom{=~}+ \int_{t_1}^T{\gpsin_-(t) \cdot \gpsca(\tau -t) \textrm{d}t}\Biggl|\\
   &\leq \left|\int_0^{t_1}{\gpsin_+(t) \cdot \gpsca(\tau -t) \textrm{d}t}\right| \notag\\
    &\phantom{\leq~} + \left|\int_{t_1}^T{|\gpsin_-(t) \cdot \gpsca(\tau -t) \textrm{d}t}\right|
\end{align}
mit dem Zeitpunkt $t_1$ des Bitwechsels, $T$ der Segmentlänge und $\gpsin_{\pm}$ dem Eingangssignal vor bzw. nach dem Bitwechsel. Ein Bitwechsel kann aber nie ausgeschlossen werden. Wie in \cite{borre2007software} beschrieben behilft man sich deshalb damit, die maximale Segmentlänge auf $1/2 \gpsTbit$ zu begrenzen, aber die Acquisiton für zwei aufeinander folgende Segmente durchzuführen. Dann kann ein Bitwechsel höchstens in einem der beiden Segmente vorkommen, und für die weitere Verarbeitung wird dann das Segment mit dem besseren Ergebnis ausgewählt.

Bei dem oben beschriebenen zweischrittigen Algorithmus (1. parallele Codephasensuche, 2. parallele Frequenzbereichsuche) kann es sich außerdem lohnen, in den beiden Schritten mit unterschiedlichen Segementlängen zu arbeiten: Der erste Schritt ist rechenaufwändiger, weil die Berechnung für alle Dopplerverschiebungen wiederholt werden muss. Aber gleichzeitig bringen bereits kurze Segmente, in der Größenordnung $\gpsTcode$, zuverlässig hohe Korrelationswerte. Es können also kurze Segmente analysiert werden um die Geschwindigkeit zu erhöhen. In der zweiten Phase muss nicht mehr über alle möglichen Codephase iteriert werden. Hier kann dann ein längeres Segment genutzt werden, um die Bestimmung der Dopplerverschiebung zu verbessern\footnote{Unschärferelation der Kurzzeit Fouriertransformation}.