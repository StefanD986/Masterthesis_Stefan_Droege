\section{Systemhardware}
Der in dieser Arbeit entwickelte GPS Empfänger ist Teil des Kommunikationssystems des Dragsail Cubesat. Das Kommunikationssystem wurde in mehreren Arbeiten am Institut für Hochfrequenztechnik der RWTH Aachen und an der Fachhochschule Aachen entwickelt. Die Entwicklung der \comboard Hardware wurde in \cite{DragsailKaiMA} und \cite{DragsailMattiMA} beschrieben und eine gute Beschreibung der DSP-FPGA Schnittstelle ist in \cite{DragsailAndrejMA} zu finden.
In diesem Abschnitt wird ein Überblick über die für diese Arbeit relevanten Teile der \comboard Hardware gegeben werden. Die Hardware des \comboard Subsystems des Dragsail Cubesat ist in \FR{COMTop.png} und \FR{COMBottom.png} zu sehen. 

Die für den GPS Empfänger relevanten Teile des \comboard sind: 
\begin{itemize}
\item Einen leistungsfähigen \gls{DSP} für die Signalverarbeitung und Systemmanagement Aufgaben.
\item Ein \gls{FPGA} der vor allem für die Bildverarbeitung der Kameras notwendig ist, aber auch beim GPS zum Einsatz kommt.
\item Ein GPS HF Frontend, was die Digitalisierung der von den GPS Antennen kommenden HF Signale übernimmt.
\end{itemize}

\FGimg{COMTop.png}{Oberseite der COM Leiterplatte.}{0.7\textwidth}

\FGimg{COMBottom.png}{Unterseite der COM Leiterplatte.}{0.7\textwidth}

\FGimg{SystemBlockdiagram.png}{Vereinfachtes Blockdiagram des \comboard. Es sind lediglich die Komponenten dargestellt, die für den GPS Empfänger relevant sind.}{0.7\textwidth}

\subsection{GPS Frontend MAX2769}
Das GPS Frontend hat die Aufgabe das L-Band Signal (\SI{1575.42}{\MHz}) auf eine \gls{ZF} herabzusetzen und zu digitalisieren. Die digitale ZF wird dem FPGA über eine parallele, synchrone Schnittstelle zugeführt. 



% MAX2769
% FPGA
% DSP 
% FPGA DSP Schnittstelle
% DSP 