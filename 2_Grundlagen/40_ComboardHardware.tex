\section{Systemhardware}
Der in dieser Arbeit entwickelte GPS Empfänger ist Teil des Kommunikationssystems des \dscubesat. Das Kommunikationssystem wurde in mehreren Arbeiten am Institut für Hochfrequenztechnik der RWTH Aachen und an der Fachhochschule Aachen entwickelt. Die Entwicklung der \comboard Hardware wurde in \cite{DragsailKaiMA} und \cite{DragsailMattiMA} beschrieben und eine gute Beschreibung der DSP-FPGA Schnittstelle ist in \cite{DragsailAndrejMA} zu finden.
In diesem Abschnitt wird ein Überblick über die für diese Arbeit relevanten Teile der \comboard Hardware gegeben werden. Die Hardware des \comboard Subsystems des \dscubesat ist in \FR{COMTop.png} und \FR{COMBottom.png} zu sehen. 

Die für den GPS Empfänger relevanten Teile des \comboard sind: 
\begin{itemize}
\item Ein leistungsfähiger \gls{DSP} (Analog Devices ADSP-BF518) für Signalverarbeitung und Systemmanagement Aufgaben.
\item Ein \gls{FPGA} (Xilinx XC6SLX9) der zum einen für die Bildverarbeitung der Kameras notwendig ist, aber auch beim GPS beim Tracking zum Einsatz kommt.
\item Ein GPS HF Frontend (Maxim Integrated MAX2769), was die Digitalisierung der von den GPS Antennen kommenden HF Signale übernimmt.
\end{itemize}

\FGimg[Oberseite der COM Leiterplatte.]{COMTop.png}{Oberseite der COM Leiterplatte.}{9cm}

\FGimg[Unterseite der COM Leiterplatte.]{COMBottom.png}{Unterseite der COM Leiterplatte.}{9cm}

In \FR{SystemBlockdiagram.png} ist ein vereinfachtes Blockdiagramm des \comboard dargestellt. Das GPS Frontend hat eine write-only SPI Schnittstelle, die an den SPI Bus des DSP angeschlossen ist, was aber lediglich der Konfiguration der Frontend Parameter dient. Das digitalisierte \gls{ZF} Signal wird dagegen über eine parallele, synchrone Schnittstelle an den FPGA weitergegeben. Der FPGA übernimmt zum einen Teile der Datenverarbeitung (\emph{Tracking}), zum anderen puffert er die Rohdaten, um sie auf Anfrage an den DSP weiterzugeben. Der Datenaustausch zwischen FPGA und DSP erfolgt über ein asynchrones Speicherinterface. 

\FGimg[Vereinfachtes Blockdiagramm des COM Board]{SystemBlockdiagram.png}{Vereinfachtes Blockdiagram des \comboard. Es sind lediglich die Komponenten dargestellt, die für den GPS Empfänger relevant sind.}{0.7\textwidth}

\subsection{GPS Frontend MAX2769}
\label{max2769frontend}
Das GPS Frontend hat die Aufgabe das L-Band Signal (\SI{1575.42}{\MHz}) auf eine \gls{ZF} herabzusetzen und zu digitalisieren. Dazu beinhaltet das Frontend einen PLL stabilisierten Frequenzsynthesizer, Low-Noise Vorverstärker, Mischer, Filter mit \gls{AGC} und AD Wandler. Die Funktion ist in weiten Bereichen über eine SPI Schnittstelle konfigurierbar: Für aktive Antennen steht eine Phantomspeisung zur Verfügung und es kann aus zwei \gls{LNA} ausgewählt werden die auf passive, bzw. aktive Antennen optimiert sind. Die \gls{LO} Frequenz wird von einer PLL erzeugt, über die die ZF Frequenz ebenfalls wählbar ist. Zur Anpassung an andere ZF Frequenzen sind außerdem die ZF Filter konfigurierbar. Eine Beschreibung aller Funktionen des MAX2769 Frontends ist dem Datenblatt \cite{max2769} zu entnehmen.

Das komplexe (\emph{IQ}) ZF Signal kann mit \SI{1}{\bit}, \SI{1.5}{\bit} oder \SI{2}{\bit}  quantisiert werden. Falls lediglich ein reelles (\emph{I}) ZF Signal gewünscht ist, stehen außerdem  \SI{2.5}{\bit} und \SI{3}{\bit} Quantisierung zur Auswahl.

Der Referenzoszillator ist ein \gls{TCXO} mit \SI{16.368}{MHz}\footnote{Auf der ersten Version des \comboard ist die TCXO Frequenz \SI{16.369}{MHz}!}. Von diesem Referenzoszillator wird die LO Frequenz mit einem Integer- oder Fractional PLL abgeleitet. Die ADC Samplingfrequenz wird ebenfalls davon abgeleitet, wobei hier ebenfalls verschiedene Teiler / Vervielfacher zur Verfügung stehen.

\FGimg[Blockdiagram MAX2769]{MAX2769.pdf}{Blockdiagramm des MAX2769 GPS Frontend ICs (entnommen aus \cite{max2769})}{0.8\textwidth}

\subsection{Digitaler Signalprozessor}
Auf dem \gls{DSP} des \comboard läuft ein embedded Linux System und der Zugriff auf den FPGA und die SPI Schnittstelle erfolgt über Treiber die als ladbare Kernelmodule ausgeführt sind \FR{LinuxSystem.png}. Anwendungsprogramme (wie der GPS Prozess) können dann über die Linux Devices auf die Hardware zugreifen. Dadurch ist die Low-Level Programmierung klar von der Anwendungsprogrammierung getrennt. Weitere Vorteile des Linux Systems sind das einfache Multitasking und der Möglichkeit der Verwendung von Standardbibliotheken. Einige Nachteile gegenüber einer \emph{Bare Metal} Implementierung sind allerdings die fehlende Echtzeitfähigkeit, weshalb in dieser Arbeit der DSP nur für nicht-zeitkritische Berechnungen, verwendet wird.

\FGimg[Systemarchitektur Linux]{LinuxSystem.png}{Prinzipielle Systemarchitektur eines Linux Systems. Die Hardware wird durch den Kernel und die Kerneltreiber abstrahiert}{0.7\textwidth}

\paragraph{EBIU Schnittstelle} \label{ebiuinterface}
Der BF518 DSP bietet zur Anbindung an externe Peripherie und Speicher die \gls{EBIU} Schnittstelle zur Verfügung. Die EBIU Schnittstelle besteht aus einem \SI{16}{\bit} Datenbus, einem \SI{19}{\bit} Adressbus und dazugehörigen Steuersignalen \FR{EBIU_blockdiagram.pdf}. Damit lassen sich sowohl synchrone Speicher (also getaktete) als auch asynchrone (zustandsgesteuerte) Speicher ansprechen. Die Peripherie ist natürlich nicht auf Speicher begrenzt, aber die Peripherie muss aus Sicht des DSP \emph{wie} ein Speicher ansprechbar sein. Aus Sicht des auf dem DSP laufenden Programms (bzw Betriebssystem) wird einfach auf eine bestimmte Speicherregion zugegriffen. Zugriffe auf diese Speicherregion aktivieren den externen Daten-/Adressbus.

\FGimg[EBIU Blockdiagramm]{EBIU_blockdiagram.pdf}{EBIU Blockdiagramm. Die EBIU bietet eine Schnittstelle zu externen synchronen und asynchronen Speichern oder Peripheriegeräten (entnommen aus \cite{BlackfinHWReference}).}{0.8\textwidth}

\FGimg[EBIU Memory Map]{EBIU_memmap.png}{Memory Map des ADSP-BF518 mit den für die EBIU reservierten Bereiche.}{0.65\textwidth}

Die EBIU Schnittstelle ist an den \gls{DMA} Controller den BF518 angeschlossen. Damit können sehr schnell große Datenmengen von Peripherie in den RAM des DSP übertragen werden.

Auf dem \comboard sind nicht alle Steuersignale der EBIU Schnittstelle an den FPGA herangeführt. Daher kann für Zugriffe auf den FPGA lediglich der \gls{AMC} genutzt werden. Im Unterschied zum synchronen Speicherinterface wird hier kein Takt verwendet, und Schreib-/Lesevorgänge werden ausschließlich über die Steuerleitungen signalisiert. Um sicherzugehen, dass die Gegenseite die Kommandos erkannt hat werden Wartezustände eingebaut. Dadurch ist die Anbindung über das Asynchrone Speicherinterface langsamer als das synchrone Speicherinterface. Auf Clientseite muss dazu eine State Machine implementiert werden, die protokolliert in welcher Phase (siehe \FR{EBIU_readwrite.pdf}) der Datenübertragung sich die Schnittstelle gerade befindet.
%Addressbus
%Datenbus
%Write Enable
%DMA

\FGimg[Timingdiagramm EBIU Asynchronous Read Write]{EBIU_readwrite.pdf}{Timing der Signale der EBIU Schnittstelle für einen asynchronen Schreib- bzw. Lesevorgang. Der Takt ist auf Clientseite nicht unbedingt benötigt. Nicht alle Steuersignale sind auf dem \comboard an den FPGA geführt. (entnommen aus \cite{BlackfinHWReference}}{0.98\textwidth}

\subsection{FPGA}
Ein FPGA ist eine Integrierter Schaltkreis in dem sich digitale Schaltkreise realisieren lassen. Der FPGA besteht einfach gesagt aus einer großen Anzahl von Gattern und Flip-Flops. Die Programmierung von FPGAs erfolgt üblicherweise über eine Hardwarebeschreibungssprache (kurz HDL) wie z.B. \emph{VHDL} oder \emph{Verilog}. Mit der HDL wird beschrieben \emph{was} die Schaltung machen soll. \emph{Wie} dies umgesetzt wird wird nicht beschrieben. Daher ist HDL Code auch ersteinmal unabhängig von der Realisierung (sofern keine herstellerspezifischen Komponenten verwendet werden). Der HDL Code könnte genau so für die Herstellung eines ASIC genutzt werden.

Die Realisierung wird stattdessen durch ein Synthesewerkzeug entschieden, dass den Quellcode analysiert und in eine Netzliste übersetzt. Die Netzliste beschreibt die Schaltung auf Gatterebene. Mit der Netzliste wird dann durch weitere Tools ein \emph{Place \& Route} durchgeführt, das die Elemente der Netzliste den physischen Ressourcen des FPGAs zuordnet. Zum Schluss wird daraus ein sogenannter \emph{Bitstream} erstellt, der in den FPGA geladen wird.

Neben reiner programmierbare kombinatorischer und sequentieller Logik, bieten heutige FPGAs außerdem dedizierte Hardware, die sich zwar in Logik realisieren lassen würde, aber unnötige Ressourcen verbrauchen würde. So beinhaltet der verwendete \comfpga mehrere \SI{18}{\bit} Hardware Multiplizierer, insgesamt \SI{576}{\kilo\bit} Block RAM, verschiedene PLL und Digital Clock Management Module und unterstützt mehrere IO Level Standards verschiedenen IO Bänken.

Die Programmierung des FPGA ist flüchtig, wodurch der FPGA auf dem \comboard sehr flexibel einsetzbar ist. Für dieses Projekt hat der FPGA einerseits die Aufgabe eines Bindeglieds zwischen GPS Frontend und DSP, andererseits bietet die programmierbare Logik hervorragende Voraussetzungen um parallelisierbare Prozesse mit Echtzeitanforderungen zu implementieren.