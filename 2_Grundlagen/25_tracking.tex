\subsection{Tracking}
Die Acquisition ist wie beschrieben ein relativ aufwändiger Prozess, und wird lediglich durchgeführt, wenn neue GPS Satelliten gesucht werden, beispielsweise nach dem Systemstart, oder wenn das Signal verloren wird. Sobald die Parameter (PRN, Codephase, Dopplershift) durch die Acquisition bestimmt wurden, werden diese an den eigentlichen Demodulator, den Trackingloop, weitergegeben. Der Begriff Trackingloop kommt daher, dass die bei der Acquisition bestimmte Codephase bzw. Dopplershift jeweils nur Momentaufnahmen sind. Dopplershift und Codephase ändern sich ständig und auch Ungenauigkeiten der Takterzeugung beim Empfänger müssen ausgeglichen werden. Der Trackingloop hat die Aufgabe die in der Acquisitonphase bestimmten Werte mittels eines Regelkreises (\emph{control loop}) nachzuführen (\emph{to track}). 

\FGimg[Blockschaltbild GPS Trackingloop]{Trackingloop.png}{Blockschaltbild des GPS Trackingloops. Der Trackingloop besteht aus zwei Regelkreisen, die die LO-Frequenz bzw. die Codefrequenz nachführen. Grün markierte Blöcke haben weniger kritische Echtzeitanforderungen (\ref{echtzeitanforderungen}).}{0.9\textwidth}

\FR{Trackingloop.png} zeigt ein Blockschaltbild des Trackingloops. Im Folgenden werden kurz die Funktionen der einzelnen Blöcke erklärt. Anschließend wird ein Überblick über die Funktion des gesamten Trackingloops gegeben. Eine detaillierte Beschreibung des ist in \cite{borre2007software} zu finden.

\paragraph{Carrier NCO} Der Carrier \gls{NCO} generiert eine lokale Kopie des (ZF-)Trägers (unter Berücksichtigung der Dopplerverschiebung). Die Frequenz des NCOs ist digital einstellbar.

\paragraph{Carrier Mischer} Die zwei Carrier Mischer mischen das Eingangssignal $\gpsin$ mit dem von dem NCO erzeugten \gls{LO} Signal $\gpslo_I$ bzw. $\gpslo_Q$. Bei korrekter Einstellung des NCO wird das Eingangssignal von  $f_{ZF}+f_{Doppler}$ herunter auf \SI{0}{\Hz} gemischt.

\paragraph{Carrier Loop Diskriminator} Der Carrier Loop Diskriminator analysiert das heruntergemischte Signal, und generiert ein Signal, dass den Phasenfehler angibt.

\paragraph{Carrier Loop Filter} Der Carrier Loop Filter wirkt als \emph{PI-Regler} (\emph{Proportional, Integral}) in dem Regelkreis. Wenn der Wert an seinem Eingang $\neq 0$ ist, integriert er diesen so lange, bis die Frequenz auf einen Wert angestiegen bzw. abgesunken ist, dass der Fehler null wird.

\paragraph{C/A Code Generator} Der \gls{CA} Code Generator erzeugt eine drei Kopien des Codes. Dabei ist die Codefrequenz, also die Geschwindigkeit mit der die Codesequenz \enquote{abgespielt} wird, digital einstellbar. Vereinfacht gesagt  funktioniert der C/A Code Generator intern als \gls{NCO} der ein Taktsignal für die erzeugenden \gls{LFSR} generiert. Die drei Kopien unterscheiden sich leicht in der Codephase, üblicherweise um die Dauer eines $\frac{1}{2}$ Chips. Deshalb werden die Kopien mit \emph{Early}, \emph{Prompt} und \emph{Late} bezeichnet. Der Zweck dieser drei Pfade wird in Abschnitt \ref{EarlyLateGate} erläutert.

\paragraph{Code Mischer} Nach der Mischung mit dem LO Signal wird das Signal $\gpsmx_I$ bzw. $\gpsmx_Q$ mit den Kopien des Codes gemischt. Nach der Mischung mit dem Code entspricht das Signal $\gpsxc_{I,P}$ theoretisch bereits wieder dem Datensignal $\gpsdat$. Zusammen mit den \emph{Integrate \& Dump} Blöcken (s.u.) bilden sie die Korrelatoren, die die Kreuzkorrelation zwischen Eingangssignal und lokaler Codekopie berechnen.

\paragraph{Integrate \& Dump} \label{IntegrateDump} Die Integrate \& Dump Blöcke integrieren das Signal an ihrem Eingang jeweils für die Dauer eines Codewortes $\gpsTcode$, speichern dann den Wert in einem Ausgangsregister bevor der Akkumulator auf Null zurückgesetzt wird und die Integration von Neuem beginnt. Die Integratoren haben eine Doppelfunktion: Zum einen wirken sie als Tiefpassfilter für die PLL. Zum anderen wird im Zusammenspiel mit den Code-Mischern die zyklische\footnote{Durch das Zurücksetzen des Integrators können $\gpsxc$ und $\gpsca$ als Funktionen behandelt werden, die außerhalb von $0\leq t < \gpsTcode$ verschwinden. Dadurch ergibt sich die zyklische Fortsetzung als $\xi_T(t)=\sum_{m=-\infty}^{\infty}{\xi(t-m\cdot \gpsTcode)}$ mit $\xi=\gpsca$ oder $\xi=\gpsxc$} Kreuzkorrelation zwischen $\gpsmx$ und $\gpsca$ berechnet. Zum Zeitpunkt $t=n\cdot \gpsTcode$, $n\in \mathbb{N}$ liegt am Ausgang des Integrators:
\begin{eqnarray}
	\gpspllout(t) &=& \int_{-\gpsTcode}^{0}{\gpsca(\tau) \cdot \gpsmx(t+\tau) \textrm{d}\tau}\\
    &=& (\gpsca(-\tau) * \gpsmx(\tau))(t)
\end{eqnarray}

Dies entspricht genau der zyklischen Kreuzkorrelation $(\gpsca \star \gpsmx)(t)$. Der \emph{Dump}, also das Speichern des Integrationswerts und Zurücksetzen des Integrators auf null, muss synchron zu den Code Perioden erfolgen. Diese Anforderung geht zwar nicht aus der zyklischen Kreuzkorrelation hervor, aber es kann sonst vorkommen, dass ein Bitwechsel innerhalb der Integrationsperiode auftritt.

\paragraph{Code Diskriminator} Der Code Diskriminator berechnet aus seinen Eingangswerten ein Steuersignal, das bestimmt ob die Codephase erhöht oder verringert werden muss.

\paragraph{Code Loop Filter} Genau wie der Carrier Loop Filter ist der Code Loop Filter ein \emph{PI-Glied} und erzeugt aus dem Phasenfehlersignal des Code Diskriminators ein Steuersignal für den NCO im C/A Code Generator.

\subsubsection{Carrier Tracking}
Zum Synchronisieren der LO Frequenz mit der Trägerfrequenz des Empfangssignals wird im Allgemeinen ein Phase Locked Loop (PLL) verwendet, der versucht die Phasendifferenz zwischen Eingangssignal und LO Signal auf null zu halten. Im Falle des GPS Signals ist jedoch zu beachten, dass der Träger BPSK moduliert ist, also Phasendifferenzen von \SI{0}{\degree} oder \SI{180}{\degree} gewollt sind. Eine Variante von PLL die diesen \SI{180}{\degree} Phasenfehler ignoriert ist der Costas Loop (\FR{Costasloop.png}), benannt nach dem Erfinder John P. Costas \cite{CostasSyncComm}.

\FGimg[Blockschaltbild Costas Loop]{Costasloop.png}{Der Costas Loop ist eine PLL Variante, die \gls{BPSK} modulierte Signale demodulieren kann.}{0.8\textwidth}

Dazu erzeugt der LO zwei Kopien des Trägers, die zueinander \SI{90}{\degree} verschoben sind: $\gpslo_I=\cos(\omega_{f} t)$ und  $\gpslo_Q=\sin(\omega_{f} t)$. Das Eingangssignal $\gpsin=\gpsdat \cdot \cos(\omega_{ZF} \cdot t)$ wird im jeweiligen Arm mit dem $I$ bzw $Q$ Träger gemischt und tiefpassgefiltert:
\begin{eqnarray}
	\gpsmx_I &=& \gpsin \cdot \cos(\omega_{f} t)\\
		    &=& \gpsdat \cdot \cos(\omega_{ZF} \cdot t) \cdot \cos(\omega_{f} t)\\
            &=& \frac{\gpsdat}{2} \cdot (\cos((\omega_{ZF}-\omega_{f}) t)+ \cos((\omega_{ZF}+\omega_{f}) t)) \\
    \gpspllout_I &=& \frac{\gpsdat}{2} \cdot \cos((\omega_{ZF}-\omega_{f})\cdot t)
\end{eqnarray}
$\gpspllout$  ergibt sich durch Tiefpassfilterung von $\gpsmx$, wodurch der hintere Term wegfällt. Im Q-Pfad ergibt sich nach ähnlicher Rechnung:
\begin{eqnarray}
\gpsmx_Q &=& \frac{\gpsdat}{2} \cdot (\sin((\omega_{ZF}-\omega_{f}) t)+ \sin((\omega_{ZF}+\omega_{f}) t))\\
\gpspllout_Q &=& \frac{\gpsdat}{2} \cdot \sin((\omega_{ZF}-\omega_{f})\cdot t)
\end{eqnarray}
Man sieht nun, dass wenn $\omega_{ZF}=\omega_{f}$ gilt, dann $\gpspllout_I = \frac{\gpsdat}{2}$  und $\gpspllout_Q =0$. Das Ziel des Regelkreises ist also die gesamte Energie im $I$-Arm zu halten. 
Dazu vergleicht der Loopdiskriminator die Energie im I und Q Arm, und generiert ein entsprechendes Steuersignal falls das Regelziel nicht erfüllt ist. Als Loopdiskriminator bietet sich $\arctan\left(\frac{Q}{I}\right)$ an, denn: 
\begin{equation}
	\frac{\gpspllout_Q}{\gpspllout_I} = \frac{ \frac{\gpsdat}{2} \cdot \sin((\omega_{ZF}-\omega_{f})\cdot t)}{\frac{\gpsdat}{2} \cdot \cos((\omega_{ZF}-\omega_{f})\cdot t)} = \tan({\underbrace{(\omega_{ZF}-\omega_{f})\cdot t}_{\phi_f}})
\end{equation}
Die Charakteristik des $\arctan$ Diskriminators ist in \FR{arctandiscriminator.png} dargestellt. Wie man sieht bestehen Nullstellen für \SI{0}{\degree} und \SI{180}{\degree}, woraus sich die Immunität gegen die BPSK Modulation ergibt.

\FGimg[$\arctan$ Loop Diskriminator Characteristik]{arctandiscriminator.png}{Charakteristik des $\arctan$ Diskriminators der im Costas Loop zur Anwendung kommt. Die Nullstellen bei $\phi=n\cdot \pi$ mit $n\in \mathbb{N}$ sorgen dafür, dass der Costas Loop immun gegen die durch die Modulation entstandenen \SI{180}{\degree} Phasenwechsel ist.}{0.8\textwidth}

\subsubsection{Code Tracking}
Der Code Tracking Loop ist dafür zuständig die Phase des Codegenerators so anzupassen, dass sie mit der Phase des Codes im Empfangssignal übereinstimmt. Bekannte Methoden sind zum Beispiel \emph{Direct Maximization} oder \emph{Early-Late-Gate}, wobei letzteres in dieser Arbeit verwendet wird. 

\FGimg[Blockschaltbild Early Late Gate Synchronizer]{EarlyLateBlockdiagram.png}{Vereinfachtes Blockschaltbild für die Code Synchronisation mittels Early Late Gate. Hier wird nur eine einfache Summe als Loop Diskriminator verwendet. Außerdem nicht dargestellt ist, dass die Korrelation im I und Q Arm Gleichzeitig durchgeführt wird, um Phasenfehler des Carrier Loops auszugleichen.}{0.8\textwidth}

\paragraph{Early Late Gate} \label{EarlyLateGate} In \FR{EarlyLateBlockdiagram.png} ist ein vereinfachtes Blockschaltbild des Early Late Gate Synchronisierers dargestellt. Für das Early Late Gate werden zwei Kopien des \emph{pünktlichen} Codes $\gpsca_P$ erzeugt, \emph{Early} ($\gpsca_E$) und \emph{Late} $\gpsca_L$, die $\gpsca_P$ jeweils  um $\Delta t$ voreilen ($\gpsca_E$), bzw. nacheilen ($\gpsca_L$). $\Delta t$ wird dabei üblicherweise als $\frac{1}{2} \gpsTchip$ gewählt.\footnote{Damit Verschiebungen $0<k<\gpsTchip$ möglich sind, muss das Signal über-abgetastet sein. Dies ist aber üblicherweise ohnehin der Fall um Aliasing zu verhindern.}

Das vom Carrier Mischer kommende Signal $\gpsmx$ wird dann mit den beiden Codekopien $\gpsca_E$ und $\gpsca_L$ korreliert. Bei perfekter Synchronisation werden die \emph{Early} und \emph{Late} Korrelatoren einen gleichgroßen Wert liefern (\FR{EarlyLate_punctual.png}). Falls die Codekopie voreilt oder nacheilt wird sich dagegen das Gewicht zum Early bzw Late Korrelator verschieben. 

\FGimg[Korrelator Charakteristik]{CorrelatorCharacteristic.png}{Ausgang der drei Korrelatoren in Abhängigkeit von $k$, der Verschiebung  des Eingangssignals zur pünktlichen Codekopie.}{0.8\textwidth}

\newcommand{\markerEarly}{\tikz{ \filldraw[fill=bmh02, draw=black, thin] (0,0) circle (.8ex); }~}
\newcommand{\markerLate}{\tikz{ \filldraw[fill=bmh04, draw=black, thin] (0,0) rectangle (1.4ex,1.4ex); }~}

Als Beispiel ist in \FR{EarlyLate_delayed.png} die Kreuzkorrelation zwischen einem Eingangssignal und einer um $1/4$ Chip nacheilenden Codekopie dargestellt. Weil das Eingangssignal relativ zur Codekopie voreilt, zeigt der \emph{Early} Korrelator \markerEarly einen höheren Wert an als der \emph{Late} Korrelator \markerLate. In Folge dessen ist das Signal des Code Loop Diskriminators  $\gpspllout_E - \gpspllout_L>0$ und der Code NCO wird dementsprechend seine Frequenz erhöhen, bis die Differenz ausgeglichen ist.

\FGimg[Early-Late Gate bei perfekter Synchronisation]{EarlyLate_punctual.png}{Kreuzkorrelation in der Umgebung von $k=0$ bei dem die Codekopie mit dem Eingangssignal perfekt synchron ist. Das Signal ist 4-fach überabgetastet.}{0.8\textwidth}

\FGimg[Early-Late Gate bei verspäteter Codekopie]{EarlyLate_delayed.png}{Kreuzkorrelation in der Umgebung von $k=0$ für ein Eingangssignal dem die Codekopie um $1/4$ Chip nacheilt. Das Signal ist 4-fach überabgetastet. Der Early Korrelator zeigt einen größeren Wert an, weil das Eingangssignal in Bezug auf die Codekopie voreilt.}{0.8\textwidth}

Der im Blockschaltbild (\FR{EarlyLateBlockdiagram.png}) verwendete einfache Code Loop Diskriminator hat den Nachteil, dass er einen von der Signalstärke, Codelänge, und Abtastrate abhängigen Ausgangswert erzeugt \cite{borre2007software}. Deshalb wird in dieser Arbeit ein Diskriminator verwendet, der einen normalisierten Ausgangswert erzeugt:
\begin{equation}
    \phi_g = \frac{\sqrt{\tilde{y}_{I,E}^2+\tilde{y}_{Q,E}^2}-\sqrt{\tilde{y}_{I,L}^2+\tilde{y}_{Q,L}^2}}{\sqrt{\tilde{y}_{I,E}^2+\tilde{y}_{Q,E}^2}+\sqrt{\tilde{y}_{I,L}^2+\tilde{y}_{Q,L}^2}}
\end{equation}
Die Characteristik dieses Diskriminators ist in \FR{CodeDiskriminatorCharacteristic.png} dargestellt.

\FGimg[Normalisierender Code Loop Diskriminator]{CodeDiskriminatorCharacteristic.png}{Charakteristik des in dieser Arbeit verwendeten Code Loop Diskriminators}{0.8\textwidth}

\subsubsection{Loop Filter} \label{loopfilter}
Die Regelkreise werden hier als PLL zweiter Ordnung modelliert \cite{borre2007software}. Das Modell ist in \FR{SecondOrderPLL.png} dargestellt. Wie schon erwähnt ist der Loop Filter ein \emph{PI-Glied}, hat also die Übertragungsfunktion \cite{gardner2005phaselock}
\begin{equation}
\label{Eq:analogloopfilter}
    F(s)=\frac{1 + \tau_2 \cdot s}{\tau_1 s}
\end{equation}

\FGimg[PLL zweiter Ordnung]{SecondOrderPLL.png}{PLL zweiter Ordnung, bestehend aus Phasenkomparator, Filter $F(s)$ und VCO $N(s)$}{0.9\textwidth}

Der VCO hat die Übertragungsfunktion eines \emph{I-Glieds} \cite{gardner2005phaselock}
\begin{equation}
    N(s)=\frac{K_0}{s}
\end{equation}

Der Einfluss des Loop Diskriminators ist durch ein einfaches \emph{P-Glied} mit Verstärkungsfaktor $K_d$ modelliert.

Zusammen ergibt sich damit die Übertragungsfunktion des geschlossenen Regelkreises zweiter Ordnung zu:
\begin{align}
    \frac{\Phi_{out}(s)}{\Phi_{in}(s)}=H(s) &= \frac{K_d ~ F(s)~ N(s)}{1+K_d ~ F(s)~ N(s)}\\
         &= \frac{K_0 ~K_d \cdot (\tau_2 ~s + 1)}{\tau_1 ~s^2 + K_0 ~K_d~\tau_2~s+K_0~K_d}\\
         &= \frac{2  ~ \zeta  ~ \omega_n ~ s + \omega_n^2}{s^2 + 2 ~\zeta~\omega_n~s + \omega_n}
\end{align}
Der geschlossene Regelkreis hat damit die Form eines \emph{PDT2-Glieds}. Im letzten Schritt wurden deshalb die für \emph{PDT2-Glieder} üblichen Parameter $\omega_n$, die Eigenfrequenz und $\zeta$, die Dämpfung eingeführt \cite{gardner2005phaselock}.
\begin{align}
    \omega_n &=\sqrt{\frac{K_0~K_d}{\tau_1}} \label{Eq:omegaN}\\
    \zeta &=\frac{\tau_2~\omega_n}{2} \label{Eq:zeta}
\end{align}
In \cite{parkinsonGPS} wurde darüber hinaus der Zusammenhang zur Rauschbandbreite $B_L$ einer PLL zweiter Ordnung hergeleitet:
\begin{equation}
    B_L = \frac{\omega_n \cdot (4\zeta^2+1)}{8\zeta} \label{Eq:NoiseBW}
\end{equation}

\paragraph{Zeitdiskretes Filter}
Die bisherige Erklärung bezog sich lediglich auf einen zeitkontiniuierlichen Regelkreis. Im GPS Empfänger wird das Signal allerdings abgetastet, es kommt daher ein digitales Loop Filter zum Einsatz. Das digitale Filter kann aus \eqref{Eq:analogloopfilter} mit Bilinearer Tranformation\footnote{Bilineare Tranformation: $s=\frac{2\cdot(z-1)}{T\cdot(z+1)}$ mit $T$ dem Kehrwert der Abtastrate.} gewonnen werden:
\begin{align}
    F(z) &= \frac{b_0+b_1 \cdot z^{-1}}{a_0 + a_1 z^{-1}}\\
         &= \frac{\frac{T+2\tau_2}{2 \tau_1}+\frac{T+2\tau_2}{2 \tau_1} \cdot z^{-1}}{1-z^{-1}}
\end{align}
Damit ergeben sich die Koeffizienten des digitalen Loop Filters, in Abhängigkeit von Rauschbandbreite und Dämpfung, nach Einsetzen von \eqref{Eq:omegaN},  \eqref{Eq:zeta} und \eqref{Eq:NoiseBW} zu:
\begin{align}
    b_0 &= \frac{16 ~B_L ~\zeta^2\cdot(4~\zeta^2+2 ~B_L~ T + 1)}{K_0 ~K_d \cdot (4 ~\zeta^2+1)^2} \label{Eq:FilterB0}\\
    b_1 &= \frac{-16 ~B_L ~\zeta^2\cdot(4~\zeta^2-2 ~B_L~ T + 1)}{K_0 ~K_d \cdot (4 ~\zeta^2+1)^2} \label{Eq:FilterB1}\\
    a_0 &= 1 \label{Eq:FilterA0}\\
    a_1 &= -1 \label{Eq:FilterA1}
\end{align}
