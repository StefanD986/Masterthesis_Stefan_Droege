\section{GPS Signal}
Der GNSS Empfänger der im Rahmen dieser Arbeit entwickelt werden soll verwendet das \gls{GPS} als Quelle für Navigationsdaten. In diesem Abschnitt soll ein Überblick über die Eigenschaften des Signals, sowie ein Überblick über die Demodulation Signals gegeben werden. Dabei wird sich auf die für die für diese Arbeit wichtigen Teile beschränkt: Das GPS verwendet mehrere Frequenzen und in neueren Versionen auch mehrere Modulationsverfahren \cite{specification2010gps}. Der im Rahmen dieser Arbeit entworfene Empfänger verwendet lediglich das zivil nutzbare Signal auf der L1 Frequenz mit C/A Kodierung\footnote{Beginned mit dem Jahr 2016 sollen in Zukunft 2016 auch weitere Übertragungskanäle für zivile Anwendunge zur Verfügung stehen. Siehe \cite{interface1gps}, \cite{specification2010gps} und \cite{navstar2006space}}. Für die anderen Teile des Standards sei auf die offizielle GPS Spezifikation verwiesen (\cite{specification2010gps}) verwiesen. Weiterhin gibt \cite{borre2007software} eine sehr gute Einführung zu \gls{SDR} Empfängern für \gls{GPS} und Galileo. 

Auch wenn GPS aus Sicht des Empfängers kein Netzwerkprotokoll ist, orientiert sich die folgende Beschreibung grob am OSI Modell: Zu Beginn wird die Bitübertragungsschicht (\emph{Physical Layer}) diskutiert. Im zweiten Unterabschnitt wird das verwendete Multiplex Verfahren beschrieben (\emph{\gls{MAC} Layer}). Abschließend werden die übertragenen Nutzdaten beschrieben.

\subsection{Bitübertragungsschicht}
Die von der darüberliegenden Schicht erhaltenen Daten werden auf der Bitübertragungsschicht auf einen Träger aufmoduliert und ausgesendet. 

\paragraph{Träger} 
Auf dem Satelliten befindet sich eine hoch genaue Taktquelle mit \SI{10.2299999954326}{\MHz}. Durch relativistische Effekte erscheint diese Frequenz für einen Beobachter auf der Erde wie \SI{10.23}{\MHz}. Von dieser Taktquelle sind alle weiteren Frequenzen abgeleitet. Ein Frequenzsynthesizer generiert aus dem Referenztakt durch Vervielfachung mit dem Faktor 154 den Träger für das \gls{PY} Signal für die sogenannte L1 Frequenz (\SI{1575.42}{\MHz}). Der Träger für das zivil genutzte \gls{CA} Signal wird aus dem \gls{PY} Träger durch \SI{90}{\degree} Phasenverschiebung gewonnen.

\paragraph{Modulation}
Die von der darüberliegenden Schicht erhaltenen Daten werden auf den Träger mittels \gls{BPSK} moduliert. Die Zuordnung der $1$en und $0$en zu den Phasenlagen bezieht sich dabei auf das gleichzeitig übertragene \gls{PY} Signal\footnote{Das \gls{PY} Signal ist das militärisch genutzte GPS Signal auf der L1 Frequenz. Wie jedoch später beschrieben wird, ist es zur Dekodierung des \gls{CA} Signals nicht notwendig das \gls{PY} Signal dekodieren zu können}: Die Phasenlage die der \gls{PY} Träger annimmt wenn eine $0$ übertragen wird ist als \SI{0}{\degree} definiert. Wenn das \gls{CA} Signal den Wert $1$ übertragen soll, eilt der \gls{CA} Träger dem \gls{PY} \SI{90}{\degree} vor. Bei einer $0$ eilt das CA Signal \SI{90}{\degree} nach.

Das ausgesendete Signal ergibt sich durch Addition des \gls{CA} Signals mit dem um \SI{3}{\dB} gedämpften \gls{PY} Signal. \FR{GPSConstellation} stellt die Symbolkonstellation grafisch dar. Wie man sieht, gibt es je zwei Symbole die eine $1$ bzw. eine $0$ des CA Signals signalisieren. Ein Empfänger wendet auf das empfangene Signal einen Tiefpassfilter an. Da das \gls{PY} Signal durch ein Informationssignal mit deutlich höherer Bandbreite moduliert ist und durch den Chipping Code ein pseudo-Zu\-falls\-ver\-hal\-ten aufweist ergeben sich nach der Tiefpassfilterung wieder die Konstellationspunkte $C_{1,2}$.

\FG{\definecolor{qqwuqq}{rgb}{0.,0.39215686274509803,0.}
\definecolor{qqzzff}{rgb}{0.,0.6,1.}
\definecolor{ffqqqq}{rgb}{1.,0.,0.}
\definecolor{qqccqq}{rgb}{0.,0.8,0.}
\begin{tikzpicture}[line cap=round,
line join=round,
>=latex,
x=1.0cm,
y=1.0cm,
scale=2.5]

\draw[->,color=black] (-1.4179421301831812,0.) -- (1.5320124928341259,0.);

\foreach \x in {-1.4,-1.2,-1.,-0.8,-0.6,-0.4,-0.2,0.2,0.4,0.6,0.8,1.,1.2,1.4}
	\draw[shift={(\x,0)},color=black] (0pt,1pt) -- (0pt,-1pt);

\draw[->,color=black] (0.,-1.2176630477463106) -- (0.,1.293102020064641);

\foreach \y in {-1.2,-1.,-0.8,-0.6,-0.4,-0.2,0.2,0.4,0.6,0.8,1.,1.2}
	\draw[shift={(0,\y)},color=black] (1pt,0pt) -- (-1pt,0pt);

\clip(-1.4179421301831812,-1.2176630477463106) rectangle (1.5320124928341259,1.493102020064641);

\draw [shift={(0.,0.)},color=qqwuqq,fill=qqwuqq,fill opacity=0.1] (0,0) -- (0.014235420690942465:0.11659899695720582) arc (0.014235420690942465:109.48545605518163:0.11659899695720582) -- cycle;
\draw(0.,0.) circle (1.cm);

\draw(0.,0.) circle (0.7071067811865475cm);

\draw [->] (0.,0.) -- (-0.40853516893812,1.1545990714272647);

\begin{scriptsize}

\draw[color=black] (-0.4890367877574414,0.8033862328445482) node {$d$};

\draw[color=black] (-0.34523135817688755,0.5546417060025964) node {$c$};

\draw [fill=qqccqq] (0.40810483232589984,0.5774516826819791) circle (1.5pt);

\draw[color=qqccqq] (0.4437551879002052,0.6673540697278559) node {$P_0$};

\draw [fill=ffqqqq] (-0.8166400012640198,0.5771473887452856) circle (1.5pt);

\draw[color=ffqqqq] (-0.9126798100352892,0.6945605023511943) node {$C_1$};

\draw [fill=ffqqqq] (0.8166400012640198,-0.5771473887452856) circle (1.5pt);

\draw[color=ffqqqq] (0.8751714766418667,-0.572481931249998) node {$C_0$};

\draw [fill=qqccqq] (-0.40810483232589984,-0.5774516826819791) circle (1.5pt);

\draw[color=qqccqq] (-0.37243779080023554,-0.486976000148077) node {$P_1$};

\draw[color=black] (-0.1625595962772651,0.6012813047854624) node {$b$};

\draw [fill=qqzzff] (-0.40853516893812,1.1545990714272647) circle (1.5pt);

\draw[color=qqzzff] (-0.353571458481167,1.2581223209774914) node {$P_0+C_1$};

\draw [fill=qqzzff] (1.2247448335899196,3.0429393669351157E-4) circle (1.5pt);

\draw[color=qqzzff] (1.267154599223994,0.10824571817393667) node {$P_0+C_0$};

\draw [fill=qqzzff] (0.40853516893812,-1.1545990714272647) circle (1.5pt);

\draw[color=qqzzff] (0.47096162052355317,-1.0460843517019962) node {$P_1+C_0$};

\draw [fill=qqzzff] (-1.2247448335899196,-3.0429393669351157E-4) circle (1.5pt);

\draw[color=qqzzff] (-1.1814243368773283,0.08824571817393667) node {$P_1+C_1$};

\draw[color=qqwuqq] (0.2810834260005827,0.15820511634823564) node {$\alpha = \SI{109.5}{\degree}$};

\end{scriptsize}
\end{tikzpicture}}{Konstellationsdiagramm des L1 GPS Signals (blau, $P_{0,1}+C_{0,1}$). Das GPS Signals ergibt sich durch Addition des modulierten \gls{CA} Trägers (rot, $C_{0,1}$) mit dem modulierten, um \SI{3}{\dB} gedämpften, \gls{PY} Trägers.}{GPSConstellation}

\paragraph{Polarisation} Das Signal wird mit einer rechtszirkular polarisierten Antenne ausgestrahlt. Dies hilft Multipath-Ausbreitung zu unterdrücken: Eine Reflektion dreht die Polarisation zu einer linkspolariserten Welle. Durch Verwendung einer rechtszirkular polarisierten Empfangsantenne wird das reflektierte Signal beim Empfang unterdrückt.

\subsection{Medium Access Control mit CDMA}
\label{basics_cdma}
Wie im vorherigen Abschnitt beschrieben senden alle GPS Satelliten auf derselben Frequenz. Zum Multiplexen der vielen Signale kommt bei GPS \gls{CDMA} zum Einsatz. Dabei wird das eigentliche Informationssignal vor der Modulation auf dem Träger mit einem Codewort multipliziert. Jeder aktive Satellit hat ein individuelles Codewort zugewiesen. Die unterschiedlichen Codeworte - und damit die Satelliten - werden durch die sogenannte \gls{PRN} identifiziert.

Da die Coderate deutlich größer ist als die Informationrate erhöht sich durch die Multiplikation mit dem Codewort die Bandbreite des Signals stark. Daher zählt \gls{CDMA} zu den \emph{Spread Spectrum} Verfahren (\enquote{\emph{aufspreizen}}).

Die Codeworte haben die Eigenschaft orthogonal zueinander zu sein, haben also eine geringe Kreuzkorrelation (\FR{GoldcodeXcorr.png}). Außerdem weisen sie für Verschiebungen $k\neq0$ eine sehr geringe Autokorrelation auf (\FR{GoldcodeAKF.png}). Letzteres ist ein weiteres Mittel um Multipath Empfangswege zu unterdrücken: Die reflektierten Signale kommen im Vergleich zum direkten Ausbreitungsweg verzögert an. Bei der Demodulation wird die Frequenzspreizung wieder rückgängig gemacht, was nur bei exakter Übereinstimmung der Codephase $k$ gelingt. Folglich werden die reflektierten Signale stark unterdrückt.

\FGimg{GoldcodeAKF.png}{Zyklische Autokorrelationsfunktion ($\mathcal{F}^{-1}(X\cdot X^*)_k$) eines Goldcode Wortes. Die Autokorrelation ist lediglich bei perfekter Überlagerung (Verschiebung $k=0$) groß, ansonsten sehr klein. (vgl mit \FR{GoldcodeXcorr.png})}{.9\textwidth}

\FGimg{GoldcodeXcorr.png}{Zyklische Kreuzkorrelation ($\mathcal{F}^{-1}(X\cdot Y^*)_k$) zweier unterschiedlicher Goldcode Worte. Die Kreuzkorrelation ist sehr gering (vgl mit \FR{GoldcodeAKF.png})}{.9\textwidth}

Da die Bits des Codeworts keine Information enthalten werden sie zur besseren Unterscheidung als \glspl{Chip} bezeichnet. Die wichtigen Parameter für CDMA sind die \emph{Coderate} und der verwendete Code. Das zivil nutzbare GPS Signal hat eine Coderate von \SI{1.023}{MChip/s} und verwendet \emph{Gold Codefolgen} (benannt nach Robert Gold). Die Gold Codefolgen werden durch die punktweise Modulo-2 Addition zweier \emph{Maximum Length Sequences} \gls{LFSR} erzeugt. Je nach \gls{PRN} werden bei der Addition andere Abgriffe (\emph{Taps}) an dem \gls{LFSR} verwendet. Die erzeugenden Polynome der beiden ML Sequenzen sind $G_1: 1 + x^3 + x^10$ und $G_2: 1 + x^2 + x^3 +x^6 + x^8 + x^9 + x^{10}$. \FR{LFSRGoldcode.png} stellt die Generierung der Gold Codes grafisch dar. Eine Liste mit der Zuordnung der verwendeten Abgriffe ist \cite{specification2010gps} zu entnehmen.

\FGimg{LFSRGoldcode.png}{Erzeugung der Goldcode Folgen durch punktweiser Modulo-2 Addition (\emph{XOR}) zweier Maximum Length Sequences die jeweils durch ein \gls{LFSR} erzeugt werden. Die Abgriffe die bei der Addition verwendet werden sind Abhängig von der \gls{PRN}}{.9\textwidth}

\subsection{GPS Nutzdaten}
Die Nutzdaten - im GPS Kontext \emph{NAV Daten} genannt - werden mit einer Bitrate von \SI{50}{\bit\per\sec} übertragen. Im folgenden werden die Struktur und die wichtigsten Nutzdaten beschrieben.

\paragraph{Übertragene Daten}
Prinzipiell erfolgt die Positionsbestimmung beim Empfänger über einfache geometrische und physikalische Zusammenhänge, indem aus dem Zeitpunkt der Aussendung des Signals und der Lichtgeschwindigkeit die Entfernung des Empfängers vom Sender bestimmt wird. Mit mindestens 3 Satelliten lässt sich theoretisch die Position im dreidimensionalen Raum bestimmen\footnote{Da die exakte Zeit beim Empfänger eine weitere Unbekannte ist, sind in der Realität mindestens 4 Satellitensignale erforderlich.}.

Deshalb enthalten die NAV Daten die folgenden Informationen:
\begin{itemize}
\item Zeitpunkt der Aussendung der Nachricht.
\item Ephemeride die Auskunft über den Orbit des aussendenden GPS \gls{SV} geben.
\item den \emph{Almanach} der Information über die Orbits der anderen GPS \gls{SV} gibt.
\item weitere Statusdaten wie Information über die Ionosphäre\footnote{Die volle Beschreibung der Daten ist in \cite{specification2010gps} zu finden.}.
\end{itemize}

\paragraph{Datenstruktur}
Die \emph{NAV Daten} sind in \emph{Frames} und \emph{Subframes} und \emph{Words} unterteilt. Ein \emph{Frame} ist \SI{1500}{\bit} lang, bestehend aus 5 \emph{Subframes}, von je \SI{300}{\bit} Länge. Die \emph{Subframes} sind weiter unterteilt in \emph{Words} von je \SI{30}{\bit} Länge. Zur Erkennung von Übertragungsfehlern ist jedes Word mit 6 Paritätsbits abgesichert. 

\FGimg{GPSData.png}{Übersicht über die Datenstruktur mit der die NAV Nachrichten übertragen werden. Subframes 4 und 5 unterscheiden sich in jeder \emph{Page}. Jeder Subframe beginnt mit einer Präambel im ersten Wort.}{0.9\textwidth}

Zur Synchronisation des Empfängers beginnt das erste Wort jedes Subframes mit einer Preamble. Diese hat zwei Funktionen: Der Demodulator hat zwei stabile Zustände, die zu einer Negierung der Informationsbits führen können. Die Preamble wird genutzt, um dies zu erkennen bzw. zu korrigieren. Außerdem wird die Preamble genutzt um den Beginn eines Subframes zu finden.

Alle Subframes übertragen in den ersten zwei Words die aktuelle Zeit zum Zeitpunkt der Aussendung. Die Ephemeriden mit den Orbitdaten und weiteren Informationen über das aussendende \gls{SV} werden in den Subframes 1, 2 und 3 übertragen. Diese drei Subframes werden in jedem Frame wiederholt. Die Almanach Daten sind deutlich länger, und werden deshalb segmentiert und über 25 Frames in den Subframes 4 und 5 übertragen. Die Übertragung eines vollständigen Datensatzes dauert damit \SI{12.5}{\minute}.

\FGimg{MoDem.png}{Prinzip der GPS Signalerzeugung und Informationsrückgewinnung: Das Datensignal wird mit dem \gls{CA} Code Modulo-2 addiert (\emph{gespreizt}), und anschließend auf den Träger moduliert}{0.7\textwidth}