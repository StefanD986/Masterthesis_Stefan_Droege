\subsubsection{Code NCO}
Diese Komponente ist in  \lstinline$code_nco.vhd$ implementiert und erzeugt drei Steuersignale: Zwei Clock Enable Signale für die \lstinline$code_replica_generator$ Komponente und ein drittes Signal, das von der \lstinline$integrate_and_dump$ Komponente benutzt wird und außerdem zur Interrupt Signalisierung an den MBlite Prozessor dient.

Das keine \emph{Taktsignale}, sondern nur \emph{Steuersignale} generiert werden ist ein wichtiges Detail: Dadurch werden  nachfolgende Komponenten weiterhin mit dem Systemtakt getrieben und bleiben somit synchron zu den anderen Komponenten des Trackingloop und dem MBlite Prozessor. Andernfalls könnte dies Timingprobleme und somit Metastabilität verursachen.

Die drei Ausgangssignale \lstinline$Ena_clk$, \lstinline$Ena_clk_x2$ und \lstinline$Ena_clk_div1023$ sind in dem \lstinline$record$ \lstinline$o_nco$ zusammengefasst (\TR{TabCodeNCO_Type}). Die Ausgangssignale sind Impulse, d.h. sie sind nur genau für die Dauer $1/\gpsfsamp$ gleich \lstinline$'1'$ und sonst \lstinline$'0'$. Die Frequenz, mit der sich die Impulse wiederholen, ist abhängig von dem Wert an \lstinline$i_fcw$. Genau wie bei dem Carrier NCO ist die Frequenz \emph{proportional} aber \emph{nicht gleich}  zu dem \gls{FCW}:

\begin{equation}
    FCW=f_{out}\cdot K_{0,code} =f_{out}\cdot  \frac{2^{N_{codeNCO}}}{f_{clk}}
\end{equation}

\FGimg[Entity/Architecture des Code NCO]{UML_class_code_nco.pdf}{Entity und Architecture und darin definierte Typen der \lstinline$code_nco$ Komponente.}{7cm}


\paragraph{Schnittstelle (Entity)}
In \TR{TabCALFSR_Entity} ist die Schnittstelle der \lstinline$code_nco$ Komponente beschrieben. In \TR{TabCodeNCO_Type} ist die Struktur des Typs des Ausgangssignals \lstinline$o_nco$ beschrieben.

\begin{table}[htbp]
    \ttabbox
    {
        \caption[Code NCO Schnittstelle]{Schnittstellenbeschreibung (Entity) der \lstinline$code_nco$ Komponente}
        \label{TabCodeNCO_Entity}
    }
    {
        \rowcolors{2}{light-gray}{White}
    \begin{tabular}{c c  p{2cm} p{6cm}}
        \toprule
        Name                    & I/O  & Typ                               & Beschreibung \\
        \midrule
        \lstinline$i_clk$       & I         & \lstinline$std_logic$             & Taktsignal das auch die anderen Teile des Tracking Loop antreibt. Dies sollte das (aufgefrischte) Taktsignal des GPS Frontends sein.\\
        \lstinline$i_reset$     & I         & \lstinline$std_logic$             & Asynchrones Reset Signal (aktiv wenn \lstinline$i_reset='1'$) \\
        \lstinline$i_fcw$    & I         & \lstinline$t_code_fcw$             & Frequency Control Word \\
        \lstinline$o_nco$       & O         & \lstinline$t_code_ nco_out$ & RECORD holding the Clock Enable signals. \\
        \bottomrule
    \end{tabular}
}
\end{table}

\begin{table}[htbp]
    \ttabbox
    {
        \caption[Typdefinition \lstinline$t_code_nco_out$]{Beschreibung der Struktur des \lstinline$t_code_nco_out$ Typs. Der Typ ist in \lstinline$tracking_loop_pkg.vhd$ definiert.}
        \label{TabCodeNCO_Type}
    }
    {
        \rowcolors{2}{light-gray}{White}
    \begin{tabular}{p{2cm}  p{2cm} p{6cm}}
        \toprule
        Name				& Typ                   & Beschreibung \\
        \midrule
        \lstinline$Ena_clk$		& \lstinline$std_logic$	& Impuls, mit Abstand zwischen zwei Impulsen von etwa $\gpsTchip=1/\gpsfchip$\footnote{Wenn der Regelkreis eingeschwungen ist und unter Berücksichtigung der Dopplerverschiebung.} \\
        \lstinline$Ena_clk_x2$		& \lstinline$std_logic$	& Impuls, mit Impulsabstand von etwa $\gpsTchip/2$ \\
        \lstinline$Ena_clk_ div1023$	& \lstinline$std_logic$	& Impuls, mit Impulsabstand von etwa $1023\cdot \gpsTchip$ \\
        \bottomrule
    \end{tabular}
}
\end{table}


\paragraph{Implementierung (Architecture)}

\begin{table}[htbp]
    \ttabbox
    {
        \caption[Code NCO interne Signale]{Interne Signale der \lstinline$code_nco$ Komponente}
        \label{TabCodeNCO_ArchSignals}
    }
    {
        \rowcolors{2}{light-gray}{White}
    \begin{tabular}{c  l p{5cm}}
        \toprule
        Name      & Typ         & Beschreibung \\
        \midrule
        \lstinline$r$		& \lstinline$t_code_nco_reg$	& \gls{FSM} Zustandsregister (siehe auch \TR{Tab_t_code_nco_reg_Type})\\
        \lstinline$r_next$	& \lstinline$t_code_nco_reg$	& Signal für den zukünftigen Zustand der \gls{FSM}.\\
        \bottomrule
    \end{tabular}
}
\end{table}

\begin{table}[htbp]
    \ttabbox
    {
        \caption[Typdefinition \lstinline$t_code_nco_out$]{Beschreibung der Struktur des Code NCO Zustandsregisters (\lstinline$t_code_nco_out$ Typ).}
        \label{Tab_t_code_nco_reg_Type}
    }
    {
        \rowcolors{2}{light-gray}{White}
    \begin{tabular}{p{2cm}  p{2cm} p{6cm}}
        \toprule
        Name				& Typ						& Beschreibung \\
        \midrule
        \lstinline$count$		& \lstinline$unsigned (N_code_nco-1 DOWNTO 0)$	& NCO Phasenakkumulator, mit dem \lstinline$o_nco.Ena_clk$ und \lstinline$o_nco.Ena_clk_x2$ generiert werden. \\
        \lstinline$count2$		& \lstinline$integer range 0 to 1023$		& Zähler (bzw. Taktteiler) mit dem \lstinline$o_nco.Ena_clk_div1023$ generiert wird. \\
        \lstinline$prev_count$		& \lstinline$unsigned (1 DOWNTO 0)$		& Register das der vorherigen Zustand der obersten zwei Bits des  \lstinline$count$  Registers speichert. Wird verwendet um Impuls Signale zu generieren, die nur für genau eine Taktperiode \lstinline$'1'$ sind. \\
        \lstinline$nco_to_ ca_codegen$	& \lstinline$t_code_ nco_out$			& Register für die Ausgangssignale.\\
        \bottomrule
    \end{tabular}
}
\end{table}

Wie die Architecture des Carrier NCO enthält auch die Code NCO Archtiecture einen Phasenakkumulator. Im Gegensatz zu dem Carrier NCO wird zur Erzeugung des Ausgangssignals aber keine \gls{LUT} verwendet. Stattdessen wird eine Logik eingesetzt, die das MSB des Akkumulators überwacht und \lstinline$Ena_clk$ für genau eine Taktperiode auf \lstinline$'1'$ setzt. Dazu wird ein zweites Register \lstinline$prev_count$ benutzt, dass den vorherigen Zustand des MSB speichert und \lstinline$Ena_clk$ nur auf \lstinline$'1'$ setzt, wenn das MSB den Zustand \emph{gewechselt} hat. Alternativ hätte auch der Akkumulatorwert mit $2^N_{codeNCO}-1$ verglichen werden können. In diesem Fall müssten aber $N_{codeNCO}$ Bits verglichen werden, wofür mehr Logikressourcen benötigt werden. Daher wurde die zuvor erklärte Logik verwendet, die lediglich ein Flipflop und einen \SI{1}{\bit} Vergleicher benötigt.

Ähnliche Logik wird für die Generierung des Signals \lstinline$Ena_clk_x2$ eingesetzt, hier wird aber stattdessen das zweihöchste Bit überwacht.

Gleichzeitig wird bei einem \lstinline$Ena_clk$ Impuls ein Zähler (\lstinline$r.count2$) inkrementiert. Wenn dieser Zähler einen Stand von 1023 erreicht, wird er auf $0$ zurück gesetzt und \lstinline$Ena_clk_div1023$ für eine Taktperiode auf \lstinline$'1'$ gesetzt.

