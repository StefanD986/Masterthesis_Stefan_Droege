\subsection{Tracking Loop}
Der Trackingloop ist teils in Hardware und Teils in Software im MBlite Softcore Prozessor implementiert. In diesem Teil wird der Hardware Teil beschrieben. Die Beziehungen der einzelnen Klassen des Tracking Loop sind in \FR{Impl_UML_TL_classdiagram.pdf} dargestellt. Alle dort abgebildeten Komponenten sind in der Bibliothek \lstinline$gps$ zusammengefasst. In \FR{UML_class_tracking_loop_pkg.pdf} sind die im \emph{Package} \lstinline$tracking_loop_pkg$, definierten Typen dargestellt. Außerdem sind dort einige Konstanten und die Component Deklarationen zusammenfasst (nicht dargestellt in \FR{UML_class_tracking_loop_pkg.pdf}).

\FGimg[Klassendiagramm des Tracking Loop]{Impl_UML_TL_classdiagram.pdf}{Klassendiagramm des Trackingloop. Die VHDL \emph{Entities} sind als UML \emph{Interfaces} modelliert, und die dazugehörigen VHDL \emph{Architectures} als UML \emph{Klassen}.}{0.95\textwidth}

\FGimg[Tracking Loop Package]{UML_class_tracking_loop_pkg.pdf}{Weitere Datentypen die im Package \lstinline$tracking_loop_pkg$ definiert sind. Außerdem werden in diesem Package einige Konstanten und die Component Deklarationen zusammenfasst (hier nicht dargestellt).}{9cm}

\subsubsection{Carrier NCO}
\label{VCOimplementierung}
Diese Komponente ist in  \lstinline$carrier_nco.vhd$ implementiert und erzeugt die zwei zueinander \SI{90}{\degree} verschobenen LO Signale, mit einer zur Laufzeit einstellbaren Frequenz. Die Frequenz wird durch das \gls{FCW} bestimmt, welches nicht \emph{gleich}, aber \emph{proportional} zu der Frequenz $f_{out}$ des Ausgangssignals ist:

\begin{equation}
    FCW=f_{out}\cdot K_{0,carrier} =f_{out}\cdot  \frac{2^{N_{carrierNCO}}}{f_{clk}}
\end{equation}
mit der Frequenz $f_{clk}$ des Signals \lstinline$i_clk$.

\FGimg[Entity/Architecture des Carrier NCO]{UML_class_carrier_nco.pdf}{Entity und Architecture welche die Trägerkopie erzeugt.}{10cm}

\paragraph{Schnittstelle (Entity)}
In \TR{TabCarrierNCO_Entity} ist die Schnittstelle der \lstinline$carrier_nco$ Komponente beschrieben. In \TR{Tab_t_carrier_nco_out} ist die Struktur des Typs des \lstinline$o_nco$ Ausgangssignals beschrieben.

\begin{table}[htbp]
    \ttabbox
    {
        \caption[Carrier NCO Schnittstelle]{Schnittstellenbeschreibung (Entity) der \lstinline$carrier_nco$ Komponente.}
        \label{TabCarrierNCO_Entity}
    }
    {
        \rowcolors{2}{light-gray}{White}
    \begin{tabular}{c c  p{2cm} p{6cm}}
        \toprule
        Name                    & I/O  & Typ                               & Beschreibung \\
        \midrule
        \lstinline$i_clk$       & I         & \lstinline$std_logic$             & Taktsignal das auch die anderen Teile des Tracking Loop antreibt. Dies sollte das (aufgefrischte) Taktsignal des GPS Frontends sein.\\
        \lstinline$i_reset$     & I         & \lstinline$std_logic$             & Asynchrones Reset Signal (aktiv wenn \lstinline$i_reset='1'$) \\
        \lstinline$i_deltaf$    & I         & \lstinline$t_carrier _fcw$             & Frequency Control Word \\
        \lstinline$o_nco$       & O         & \lstinline$t_carrier _nco_out$ & \lstinline$record$ der die beiden LO Signale (I und Q) enthält. \\
        \bottomrule
    \end{tabular}
}
\end{table}

\begin{table}[htbp]
    \ttabbox
    {
        \caption[Typdefinition Code NCO Ausgangssignal]{Beschreibung der Struktur des \lstinline$t_carrier_nco_out$ Typs. Der Typ ist in \lstinline$tracking_loop_pkg.vhd$ definiert.}
        \label{Tab_t_carrier_nco_out}
    }
    {
        \rowcolors{2}{light-gray}{White}
    \begin{tabular}{c  p{2cm} p{6cm}}
        \toprule
        Name				& Typ                   & Beschreibung \\
        \midrule
        \lstinline$I$		& \lstinline$std_logic$	& \SI{1}{\bit} quantisiertes Sinus Signal mit Frequenz $f_{out}=f(FCW)$\\
        \lstinline$Q$		& \lstinline$std_logic$	& \SI{1}{\bit} quantisiertes Cosinus Signal mit Frequenz $f_{out}=f(FCW)$ \\
        \bottomrule
    \end{tabular}
}
\end{table}

\paragraph{Implementierung (Architecture)}

\begin{table}[htbp]
    \ttabbox
    {
        \caption[Carrier NCO interne Signale]{Interne Signale der \lstinline$carrier_nco$ Komponente.}
        \label{TabCarrierNCO_ArchSignals}
    }
    {
        \rowcolors{2}{light-gray}{White}
    \begin{tabular}{c  p{2cm} p{6cm}}
        \toprule
        Name      & Typ         & Beschreibung \\
        \midrule
        \lstinline$count$  & \lstinline$unsigned (N_carrier_nco-1 DOWNTO 0)$             & \gls{FSM} Zustandsregister\\
        \lstinline$LUsin$  & \lstinline$SLV(3 downto 1)$ & Konstante mit einer Sinus Look-Up-Table.\\
        \lstinline$LUcos$  & \lstinline$SLV(3 downto 1)$ & Konstante mit einer Cosinus Look-Up-Table.\\
        \bottomrule
    \end{tabular}
}
\end{table}

Der NCO besteht aus einem Phasenakkumulator, und einer \gls{LUT}, die eine Zuordnung von Amplitudenwerten $f(\phi)$ zu einem Phasenwert $\phi$ enthält. Da hier lediglich mit \SI{1}{\bit} Werten gearbeitet wird, gestaltet sich die \gls{LUT} sehr einfach: Für Werte von $\phi$ an denen $f(\phi)\geq 0$ ist, enthält sie eine \lstinline$'1'$. Für Werte von $\phi$ bei denen $f(\phi)<0$ ist, enthält sie eine \lstinline$'0'$. Hier ein Ausschnitt aus dem Code:

\begin{lstlisting}
constant LUsin 	: std_logic_vector(3 DOWNTO 0):= "1100"; --! The `sin(x)` Look-up-table
constant LUcos 	: std_logic_vector(3 DOWNTO 0):= "1001"; --! The `cos(x)` Look-up-table
\end{lstlisting}

Der Phasenakkumulator ist in dieser Implementierung $N_carrier_nco=\SI{28}{\bit}$ weit, und ist als sequentieller Prozess realisiert. Mit jeder steigenden Taktflanke wird zu dem Phasenakkumulator der Wert an \lstinline$i_deltaf$ addiert. Die beiden MSB des Akkumulators werden dann zur Addressierung der \gls{LUT} verwendet, um die Amplitudenwerte zu erhalten. Damit wird auch klar woher das oben genannte NCO Gain $K_{0,carrier}=\frac{2^{N_{carrierNCO}}}{f_{clk}}$ kommt: Die Zeit zwischen zwei Akkumulator Überläufen ist die Periodendauer mit der die \gls{LUT} genau einmal durchlaufen wird. Die Zeit zwischen zwei Überläufen beträgt
\begin{equation}
    T=\frac{2^{N_{carrierNCO}}}{f_{clk}\cdot FCW}=\frac{1}{f_{out}}
\end{equation}




\subsubsection{Code NCO}
Diese Komponente ist in  \lstinline$code_nco.vhd$ implementiert und erzeugt drei Steuersignale: Zwei Clock Enable Signale für die \lstinline$code_replica_generator$ Komponente und ein drittes Signal, das von der \lstinline$integrate_and_dump$ Komponente benutzt wird und außerdem zur Interrupt Signalisierung an den MBlite Prozessor dient.

Das keine \emph{Taktsignale}, sondern nur \emph{Steuersignale} generiert werden ist ein wichtiges Detail: Dadurch werden  nachfolgende Komponenten weiterhin mit dem Systemtakt getrieben und bleiben somit synchron zu den anderen Komponenten des Trackingloop und dem MBlite Prozessor. Andernfalls könnte dies Timingprobleme und somit Metastabilität verursachen.

Die drei Ausgangssignale \lstinline$Ena_clk$, \lstinline$Ena_clk_x2$ und \lstinline$Ena_clk_div1023$ sind in dem \lstinline$record$ \lstinline$o_nco$ zusammengefasst (\TR{TabCodeNCO_Type}). Die Ausgangssignale sind Impulse, d.h. sie sind nur genau für die Dauer $1/\gpsfsamp$ gleich \lstinline$'1'$ und sonst \lstinline$'0'$. Die Frequenz, mit der sich die Impulse wiederholen, ist abhängig von dem Wert an \lstinline$i_fcw$. Genau wie bei dem Carrier NCO ist die Frequenz \emph{proportional} aber \emph{nicht gleich}  zu dem \gls{FCW}:

\begin{equation}
    FCW=f_{out}\cdot K_{0,code} =f_{out}\cdot  \frac{2^{N_{codeNCO}}}{f_{clk}}
\end{equation}

\FGimg[Entity/Architecture des Code NCO]{UML_class_code_nco.pdf}{Entity und Architecture und darin definierte Typen der \lstinline$code_nco$ Komponente.}{7cm}


\paragraph{Schnittstelle (Entity)}
In \TR{TabCALFSR_Entity} ist die Schnittstelle der \lstinline$code_nco$ Komponente beschrieben. In \TR{TabCodeNCO_Type} ist die Struktur des Typs des Ausgangssignals \lstinline$o_nco$ beschrieben.

\begin{table}[htbp]
    \ttabbox
    {
        \caption[Code NCO Schnittstelle]{Schnittstellenbeschreibung (Entity) der \lstinline$code_nco$ Komponente}
        \label{TabCodeNCO_Entity}
    }
    {
        \rowcolors{2}{light-gray}{White}
    \begin{tabular}{c c  p{2cm} p{6cm}}
        \toprule
        Name                    & I/O  & Typ                               & Beschreibung \\
        \midrule
        \lstinline$i_clk$       & I         & \lstinline$std_logic$             & Taktsignal das auch die anderen Teile des Tracking Loop antreibt. Dies sollte das (aufgefrischte) Taktsignal des GPS Frontends sein.\\
        \lstinline$i_reset$     & I         & \lstinline$std_logic$             & Asynchrones Reset Signal (aktiv wenn \lstinline$i_reset='1'$) \\
        \lstinline$i_fcw$    & I         & \lstinline$t_code_fcw$             & Frequency Control Word \\
        \lstinline$o_nco$       & O         & \lstinline$t_code_ nco_out$ & RECORD holding the Clock Enable signals. \\
        \bottomrule
    \end{tabular}
}
\end{table}

\begin{table}[htbp]
    \ttabbox
    {
        \caption[Typdefinition \lstinline$t_code_nco_out$]{Beschreibung der Struktur des \lstinline$t_code_nco_out$ Typs. Der Typ ist in \lstinline$tracking_loop_pkg.vhd$ definiert.}
        \label{TabCodeNCO_Type}
    }
    {
        \rowcolors{2}{light-gray}{White}
    \begin{tabular}{p{2cm}  p{2cm} p{6cm}}
        \toprule
        Name				& Typ                   & Beschreibung \\
        \midrule
        \lstinline$Ena_clk$		& \lstinline$std_logic$	& Impuls, mit Abstand zwischen zwei Impulsen von etwa $\gpsTchip=1/\gpsfchip$\footnote{Wenn der Regelkreis eingeschwungen ist und unter Berücksichtigung der Dopplerverschiebung.} \\
        \lstinline$Ena_clk_x2$		& \lstinline$std_logic$	& Impuls, mit Impulsabstand von etwa $\gpsTchip/2$ \\
        \lstinline$Ena_clk_ div1023$	& \lstinline$std_logic$	& Impuls, mit Impulsabstand von etwa $1023\cdot \gpsTchip$ \\
        \bottomrule
    \end{tabular}
}
\end{table}


\paragraph{Implementierung (Architecture)}

\begin{table}[htbp]
    \ttabbox
    {
        \caption[Code NCO interne Signale]{Interne Signale der \lstinline$code_nco$ Komponente}
        \label{TabCodeNCO_ArchSignals}
    }
    {
        \rowcolors{2}{light-gray}{White}
    \begin{tabular}{c  l p{5cm}}
        \toprule
        Name      & Typ         & Beschreibung \\
        \midrule
        \lstinline$r$		& \lstinline$t_code_nco_reg$	& \gls{FSM} Zustandsregister (siehe auch \TR{Tab_t_code_nco_reg_Type})\\
        \lstinline$r_next$	& \lstinline$t_code_nco_reg$	& Signal für den zukünftigen Zustand der \gls{FSM}.\\
        \bottomrule
    \end{tabular}
}
\end{table}

\begin{table}[htbp]
    \ttabbox
    {
        \caption[Typdefinition \lstinline$t_code_nco_out$]{Beschreibung der Struktur des Code NCO Zustandsregisters (\lstinline$t_code_nco_out$ Typ).}
        \label{Tab_t_code_nco_reg_Type}
    }
    {
        \rowcolors{2}{light-gray}{White}
    \begin{tabular}{p{2cm}  p{2cm} p{6cm}}
        \toprule
        Name				& Typ						& Beschreibung \\
        \midrule
        \lstinline$count$		& \lstinline$unsigned (N_code_nco-1 DOWNTO 0)$	& NCO Phasenakkumulator, mit dem \lstinline$o_nco.Ena_clk$ und \lstinline$o_nco.Ena_clk_x2$ generiert werden. \\
        \lstinline$count2$		& \lstinline$integer range 0 to 1023$		& Zähler (bzw. Taktteiler) mit dem \lstinline$o_nco.Ena_clk_div1023$ generiert wird. \\
        \lstinline$prev_count$		& \lstinline$unsigned (1 DOWNTO 0)$		& Register das der vorherigen Zustand der obersten zwei Bits des  \lstinline$count$  Registers speichert. Wird verwendet um Impuls Signale zu generieren, die nur für genau eine Taktperiode \lstinline$'1'$ sind. \\
        \lstinline$nco_to_ ca_codegen$	& \lstinline$t_code_ nco_out$			& Register für die Ausgangssignale.\\
        \bottomrule
    \end{tabular}
}
\end{table}

Wie die Architecture des Carrier NCO enthält auch die Code NCO Archtiecture einen Phasenakkumulator. Im Gegensatz zu dem Carrier NCO wird zur Erzeugung des Ausgangssignals aber keine \gls{LUT} verwendet. Stattdessen wird eine Logik eingesetzt, die das MSB des Akkumulators überwacht und \lstinline$Ena_clk$ für genau eine Taktperiode auf \lstinline$'1'$ setzt. Dazu wird ein zweites Register \lstinline$prev_count$ benutzt, dass den vorherigen Zustand des MSB speichert und \lstinline$Ena_clk$ nur auf \lstinline$'1'$ setzt, wenn das MSB den Zustand \emph{gewechselt} hat. Alternativ hätte auch der Akkumulatorwert mit $2^N_{codeNCO}-1$ verglichen werden können. In diesem Fall müssten aber $N_{codeNCO}$ Bits verglichen werden, wofür mehr Logikressourcen benötigt werden. Daher wurde die zuvor erklärte Logik verwendet, die lediglich ein Flipflop und einen \SI{1}{\bit} Vergleicher benötigt.

Ähnliche Logik wird für die Generierung des Signals \lstinline$Ena_clk_x2$ eingesetzt, hier wird aber stattdessen das zweihöchste Bit überwacht.

Gleichzeitig wird bei einem \lstinline$Ena_clk$ Impuls ein Zähler (\lstinline$r.count2$) inkrementiert. Wenn dieser Zähler einen Stand von 1023 erreicht, wird er auf $0$ zurück gesetzt und \lstinline$Ena_clk_div1023$ für eine Taktperiode auf \lstinline$'1'$ gesetzt.



\subsubsection{Gold Code LFSR}
Diese Komponente ist in  \lstinline$ca_lfsr_pkg.vhd$ implementiert und generiert die \gls{CA} Code Sequenz zu einer PRN.

\FGimg[Entity/Architecture des Gold Code LFSR]{UML_class_LFSR.pdf}{Entity und Architecture des \gls{LFSR}, das die Gold Code Sequenz erzeugt.}{8cm}

\paragraph{Schnittstelle (Entity)}
In \TR{TabCALFSR_Entity} ist die Schnittstelle der \lstinline$ca_lfsr$ Komponente beschrieben.

\begin{table}[htbp]
    \ttabbox
    {
        \caption[LFSR Schnittstelle]{Schnittstellenbeschreibung (Entity) der \lstinline$ca_lfsr$ Komponente}
        \label{TabCALFSR_Entity}
    }
    {
        \rowcolors{2}{light-gray}{White}
    \begin{tabular}{c c  p{2cm} p{6cm}}
        \toprule
        Name                    & I/O  & Typ                               & Beschreibung \\
        \midrule
        \lstinline$i_clk$       & I         & \lstinline$std_logic$             & Taktsignal das auch die anderen Teile des Tracking Loop antreibt. Dies sollte das (aufgefrischte) Taktsignal des GPS Frontends sein.\\
        \lstinline$i_reset$     & I         & \lstinline$std_logic$             & Asynchrones Reset Signal. Bei \lstinline$i_reset='1'$ werden die LFSR \lstinline$g1$ und \lstinline$g2$ auf den Zustand \lstinline$\"1111111111\"$ zurückgesetzt.\\
        \lstinline$i_EnaClk$    & I         & \lstinline$std_logic$             & Clock Enable. Nur bei \lstinline$i_EnaClk='1'$ und einer steigende Taktflanke an \lstinline$i_clk$ schieben die LFSR \lstinline$g1$ und \lstinline$g2$ ihren Inhalt eine Position weiter.\\
        \lstinline$i_PRN$       & I         & \lstinline$unsigned (5 downto 0)$ & Wählt über die PRN Nummer die zu generierende Gold bzw. \gls{CA} Code Sequenz aus. Zulässige Werte sind im Bereich 1 bis 32.\\
        \lstinline$o_chip$      & O         & \lstinline$std_logic$             & Die generierte \gls{CA} Code Sequenz.\\
        \bottomrule
    \end{tabular}
}
\end{table}

\paragraph{Implementierung (Architecture)}

\begin{table}[htbp]
    \ttabbox
    {
        \caption[LFSR interne Signale]{Interne Signale der \lstinline$ca_lfsr$ Komponente}
        \label{TabCALFSR_ArchSignals}
    }
    {
        \rowcolors{2}{light-gray}{White}
    \begin{tabular}{c  p{2cm} p{6cm}}
        \toprule
        Name      & Typ         & Beschreibung \\
        \midrule
        \lstinline$g1$  & \lstinline$SLV(10 downto 1)$             & Register, das das LFSR mit dem $G1$ Generatorpolynom realisiert.\\
        \lstinline$g2$  & \lstinline$SLV(10 downto 1)$             & Register, das das LFSR mit dem $G2$ Generatorpolynom realisiert.\\
        \bottomrule
    \end{tabular}
}
\end{table}

Die Architecture enthält einen sequentiellen Prozess, der zwei \gls{LFSR} \lstinline$g1$ und \lstinline$g2$ mit Länge $L=10$ realisiert. Wenn \lstinline$i_ClkEna='1'$ ist, wird bei einer steigenden Taktflanke der Inhalt der Register nach links geschoben. An der LSB Stelle wird ein Wert eingeschoben, der sich aus einer Rückkopplung des Schieberegisters ergibt. Daher der Name \emph{Linear Feedback Shift Register} (LFSR).

Die Rückkopplungen sind dabei durch die Generatorpolynome festgelegt, die in der GPS Spezifikation \cite{specification2010gps} genannt sind (siehe auch Abschnitt \ref{basics_cdma} und \FR{LFSRGoldcode.png}). Damit erzeugt jedes der beiden Schieberegister eine sogenannte Maximum-Length Sequenz, eine pseudo-zufällige Bitfolge, mit Länge $L=2^{10}-1$.

Die beiden ML-Sequenzen werden dann nach dem Schema wie in \FR{LFSRGoldcode.png} modulo-2 addiert (was einer XOR Operation entspricht). Wie in \FR{LFSRGoldcode.png} zu sehen sind die Taps (Anzapfungen) bei \lstinline$g2$ Variabel, und werden durch den Wert an \lstinline$i_PRN$ ausgewählt. Zur Auswahl der Taps die einer PRN zugeordnet sind wird die Funktion \lstinline$get_taps(PRN: unsigned(5 downto 0))$ aufgerufen, die  eine Datenstruktur mit den Tap Nummern zurück gibt\footnote{Die Funktion wird bei der Synthese in eine Hardware Look-Up-Table umgesetzt. Daher dient das \lstinline$report$ Statement (welches eine Fehlermeldung ausgibt falls eine unzulässige PRN ausgewählt wurde) lediglich der Simulation und Verifikation. Solche nicht-synthetisierbaren Elemente, die lediglich der Verifikation dienen, werden auch an anderen Stellen im Code verwendet, worauf im weiteren Verlauf aber nicht gesondert hingewiesen wird.}.



\subsubsection{Code Replika Generator}
Diese Komponente ist in  \lstinline$code_replica_generator_pkg.vhd$ implementiert und erzeugt drei, zueinander um $\gpsTchip /2$ verzögerte, Codekopien (\emph{Early}, \emph{Prompt}, und \emph{Late}).

\paragraph{Schnittstelle (Entity)}
In \TR{TabCodeGen_Entity} ist die Schnittstelle der \lstinline$code_replica_generator$ Komponente beschrieben.

\begin{table}[htbp]
    \ttabbox
    {
        \caption[Code Replika Generator Schnittstelle]{Schnittstellenbeschreibung (Entity) der \lstinline$code_replica_generator$ Komponente.}
        \label{TabCodeGen_Entity}
    }
    {
        \rowcolors{2}{light-gray}{White}
    \begin{tabular}{c c  p{2cm} p{6cm}}
        \toprule
        Name                    & I/O  & Typ                               & Beschreibung \\
        \midrule
        \lstinline$i_clk$       & I         & \lstinline$std_logic$             & Taktsignal das auch die anderen Teile des Tracking Loop antreibt. Dies sollte das (aufgefrischte) Taktsignal des GPS Frontends sein.\\
        \lstinline$i_reset$     & I         & \lstinline$std_logic$             & Asynchrones Reset Signal. Bei \lstinline$i_reset='1'$ werden die LFSR \lstinline$g1$ und \lstinline$g2$ auf den Zustand \lstinline$\"1111111111\"$ zurückgesetzt.\\
        \lstinline$i_nco$    & I         & \lstinline$std_logic$             & Ein \lstinline$record$ mit mehreren Clock Enable Signalen, die von \lstinline$code_nco$ Komponente erzeugt werden.\\
        \lstinline$i_PRN$       & I         & \lstinline$unsigned (5 downto 0)$ & Wählt über die PRN Nummer die zu generierende Gold bzw. \gls{CA} Code Sequenz aus. Zulässige Werte sind im Bereich 1 bis 32.\\
        \lstinline$o_chip$      & O         & \lstinline$t_codegen_out$             & Ein \lstinline$record$ mit den drei Code Kopien.\\
        \bottomrule
    \end{tabular}
}
\end{table}


\FGimg[Entity/Architecture des Code Replika Generators]{UML_class_code_replica_generator.pdf}{Entity und Architecture welche die drei Code Kopien erzeugt.}{9.26cm}

\paragraph{Implementierung (Architecture)}

\begin{table}[htbp]
    \ttabbox
    {
        \caption[Code Replika Generator interne Signale]{Interne Signale der \lstinline$code_replica_generator$ Komponente.}
        \label{TabCodeGen_ArchSignals}
    }
    {
        \rowcolors{2}{light-gray}{White}
    \begin{tabular}{c  p{2cm} p{6cm}}
        \toprule
        Name      & Typ         & Beschreibung \\
        \midrule
        \lstinline$lfsr_out$  & \lstinline$std_logic$             & LFSR Ausgang\\
        \lstinline$shiftregister$  & \lstinline$SLV(2 downto 0)$             &  Verzögerungsleitung realisiert als dreistufiges Schieberegister\\
        \bottomrule
    \end{tabular}
}
\end{table}

Die Architecture der \lstinline$code_replica_generator$ Komponente enthält eine Instanz der \lstinline$ca_lfsr$ Komponente. Der \lstinline$i_ClkEna$ Eingang ist mit \lstinline$i_nco.Ena_clk$ verbunden. Dieses Signal 
und einen sequenziellen Prozess. Der Prozess realisiert ein dreistufiges Schieberegister, was seinen Inhalt bei \lstinline$i_nco.Ena_clk_x2='1'$ und einer steigenden Taktflanke nach rechts schiebt. 
was als Verzögerungsleitung dient: Dazu wird am MSB der von \lstinline$ca_lfsr$ erzeugte Code eingeschoben. 


\subsubsection{Korrelatoren und Integrate \& Dump}

\FGimg[Entity/Architecture des Integrate and Dump Blocks]{UML_class_IandD.pdf}{Entity und Architecture des Integrate and Dump Blocks.}{6cm}

Diese Komponente ist in \lstinline$integrate_and_dump_pkg.vhd$ implementiert und integriert das Signal an seinem Eingang \lstinline$i_data$, bis es über einen \lstinline$'1'$ Impuls an \lstinline$i_dump$ zurückgesetzt wird. Zum selben Zeitpunkt wird der Wert des Integrators in das Ausgangsregister kopiert, wo es während der sonstigen Zeit gehalten wird. Bei der Integration wird eine \lstinline$'1'$ als $+1$ und eine \lstinline$'0'$ als $-1$ interpretiert.

\paragraph{Schnittstelle (Entity)}
In \TR{TabIandD_Entity} ist die Schnittstelle der \lstinline$integrate_and_dump$ Komponente beschrieben. In \FR{UML_class_tracking_loop_pkg.pdf} ist der \lstinline$t_IandD_accu$ Typ definiert.

\begin{table}[htbp]
    \ttabbox
    {
        \caption[Integrate \& Dump Block Schnittstelle]{Schnittstellenbeschreibung (Entity) der \lstinline$integrate_and_dump$ Komponente.}
        \label{TabIandD_Entity}
    }
    {
        \rowcolors{2}{light-gray}{White}
    \begin{tabular}{c c  p{2cm} p{6cm}}
        \toprule
        Name                    & I/O	& Typ				& Beschreibung \\
        \midrule
        \lstinline$i_clk$	& I	& \lstinline$std_logic$		& Taktsignal das auch die anderen Teile des Tracking Loop antreibt. Dies sollte das (aufgefrischte) Taktsignal des GPS Frontends sein.\\
        \lstinline$i_reset$	& I	& \lstinline$std_logic$		& Asynchrones Reset Signal (aktiv wenn \lstinline$i_reset='1'$) \\
        \lstinline$i_dump$	& I	& \lstinline$std_logic$		& \emph{Dump} Impuls. Eine \lstinline$'1'$ an diesem Eingang kopiert den Integrator Wert in das Ausgangsregister und setzt den Integrator auf $0$ zurück.\\
        \lstinline$i_data$	& I	& \lstinline$std_logic$		& Das Signal an diesem Eingang wird integriert, wobei eine \lstinline$'1'$ als $+1$ und eine \lstinline$'0'$ als $-1$ interpretiert wird.\\
        \lstinline$o_IandD$	& O	& \lstinline$t_IandD_accu$	& Ausgangsregister. Enthält den Wert, des über den Zeitraum zwischen zwei \emph{Dump} Impulsen integrierten \lstinline$i_data$ Signals.\\
        \bottomrule
    \end{tabular}
}
\end{table}


\paragraph{Implementierung (Architecture)}

\begin{table}[htbp]
    \ttabbox
    {
        \caption[Integrate \& Dump interne Signale]{Interne Signale der \lstinline$integrate_and_dump$ Komponente.}
        \label{TabIandD_ArchSignals}
    }
    {
        \rowcolors{2}{light-gray}{White}
    \begin{tabular}{c  p{2cm} p{6cm}}
        \toprule
        Name      		& Typ         & Beschreibung \\
        \midrule
        \lstinline$r$		& \lstinline$t_iandd_reg$ & \gls{FSM} Zustandsregister (siehe auch \TR{Tab_t_iandd_reg_Type}) \\
        \lstinline$r_next$	& \lstinline$t_iandd_reg$ & Signal für den zukünftigen Zustand der \gls{FSM}\\
        \bottomrule
    \end{tabular}
}
\end{table}

\begin{table}[htbp]
    \ttabbox
    {
        \caption[Typdefinition \lstinline$t_iandd_reg$]{Beschreibung der Struktur des Integrate \& Dump Zustandsregisters (\lstinline$t_iandd_reg$ Typ).}
        \label{Tab_t_iandd_reg_Type}
    }
    {
        \rowcolors{2}{light-gray}{White}
    \begin{tabular}{c  p{2cm} p{6cm}}
        \toprule
        Name				& Typ						& Beschreibung \\
        \midrule
        \lstinline$count$		& \lstinline$t_IandD_accu$	&  Akkumulator des Integrators. Abhängig vom Wert an \lstinline$i_data$ wird dieser Wert mit jeder steigenden Taktflanke inkrementiert, bzw. dekrementiert.\\
        \lstinline$output$		& \lstinline$t_IandD_accu$	&  Ausgangsregister. Wenn \lstinline$i_dump='1'$ wird bei einer steigenden Taktflanke der Wert von \lstinline$count$ in dieses Register kopiert. \\
        \bottomrule
    \end{tabular}
}
\end{table}

Die Architecture implementiert den Integrator als einfachen Hoch/Runter Zähler: Bei einer steigenden Taktflanke an \lstinline$i_clk$ wird \lstinline$r.count$ abhängig von \lstinline$i_data$ inkrementiert (\lstinline$i_data='1'$) oder dekrementiert (\lstinline$i_data='0'$). Wenn \lstinline$i_dump='1'$ wird der Zählerstand in das Ausgangsregister \lstinline$r.output$ kopiert und der Zähler zurückgesetzt. Gleichzeitig wird durch das \emph{Dump} Signal bei dem MBlite Prozessor ein Interrupt ausgelöst, der daraufhin den Wert aus dem Ausgangsregister ausliest.


\subsubsection{Tracking  Loop Komponente}
Diese Komponente ist in \lstinline$tracking_loop.vhd$ beschrieben, und eine Übersicht ist in \FR{UML_class_tracking_loop.pdf} abgebildet. Sie entspricht der in \FR{GPS_UML_deployment1.png} dargestellten \emph{TrackingHW} Komponente. Für jeden Kanal wird eine Instanz dieser Komponente benötigt. Zusammen mit dem Software Teil der im MBlite Prozessor läuft, übernimmt diese Komponente nach der Acquistion das Nachführen des Lokaloszillators und des Code Generators, so wie es in Abschnitt \ref{GrundlagenTracking} beschrieben wurde.

Die Funktion dieser Komponente ist eng verknüpft mit der \emph{TrackingSW} Komponente die im MBlite Prozessor läuft: Sie muss die \emph{TrackingHW} Komponente mit den Werten für Carrier FCW, Code FCW und initialer Codephase konfigurieren. Danach muss die TrackingSW Komponente das Startsignal geben. Während des Betriebs muss sie die Korrelator Ausgabewerte lesen, damit die neuen Werte für Carrier FCW und Code FCW berechnen und die neuen Werte an die TrackingHW Komponente übermitteln.

\FGimg[Entity/Architecture Tracking Loop]{UML_class_tracking_loop.pdf}{Die übergeordnete Entity/Architecture des Tracking Loop}{10cm}

\paragraph{Schnittstelle (Entity)}
\label{ImplMemoryMapTrackingloop}
In \TR{Tab_tracking_loop_Entity} ist die Schnittstelle der \lstinline$tracking_loop$ Komponente beschrieben. Die Schnittstelle verbindet die Komponente mit dem MAX2769 GPS Frontend und dem MBlite Prozessor. Von dem GPS Frontend bekommt die Komponente die Daten und ein Taktsignal. Die gesamte Komponente arbeitet synchron zu dem ADC Taktsignal. Zum MBLite Prozessor besteht die Schnittstelle aus einem Daten- und Addressbus, und einigen Steuersignalen. Außerdem hat der Tracking Loop die Möglichkeit den Prozessor mit einem Interruptsignal über neue Daten zu informieren.

Eine Übersicht über das Memory Layout aus Sicht des MBlite Prozessors wurde bereits in Abschnitt \ref{SoftwareSchnittstelle_TL_MBlite} gegeben. In  \TR{Tab_tracking_loop_memory_map} ist eine detaillierte Beschreibung der Register gegeben. Die Registeradressen sind an \num{16} bzw. \SI{32}{\bit} Grenzen ausgerichtet und können somit vom Prozessor als \lstinline$int16_t$ bzw. \lstinline$int32_t$ Typen angesprochen werden. Trotzdem ist zu beachten, dass auf Tracking Loop Seite unter Umständen nicht alle Bits ausgewertet werden: Beispielsweise ist das Carrier FCW lediglich \SI{28}{\bit} lang. Bei Werten, die mehr als die vorgesehene Bitzahl benötigen, werden die oberen Bits abgeschnitten. Die genaue Größe der Register ist \FR{UML_class_tracking_loop_pkg.pdf} und dem Quellcode zu entnehmen.

Um den Tracking Loop zu starten muss der MBlite Prozessor folgende Schritte durchführen:

\begin{enumerate}
    \item Tracking Loop Kanal zurücksetzten: Alle Flags auf \lstinline$'0'$ setzen.
    \item Kanal aktivieren: \lstinline$enable$ Flag auf \lstinline$'1'$ setzen.
    \item Samples für Acquisition anfangen aufzunehmen und gleichzeitig \lstinline$acq_started$ Flag auf \lstinline$'1'$ setzen.
    \item Nach erfolgreicher Acquisition die Ergebnisse (Carrier FCW, Code FCW, Codephase und PRN) in die entsprechenden Register des Kanals schreiben.
    \item Abschließend die \lstinline$config_done$ Flag auf \lstinline$'1'$ setzen.
\end{enumerate}

Daraufhin wird der Trackingloop sich mit der Codephase des Eingangssignals synchronisieren (siehe den Abschnitt \emph{Architecture} für Details). Sobald neue Werte an den Korrelatoren vorliegen wird der MBlite Prozessor über die \lstinline$o_intr$ Leitung informiert. Der Prozessor muss daraufhin zeitnah (\SI{<1}{\milli\second}) die neuen Werte für Code- und Carrier FCW berechnen und in die Kanal Register schreiben. Der Kanal nimmt automatisch an, dass die Werte in den Registern aktuell sind, es ist also kein erneutes setzen der \lstinline$config_done$ Flag notwendig.

\begin{table}[htbp]
    \ttabbox
    {
        \caption[Carrier NCO Schnittstelle]{Schnittstellenbeschreibung (Entity) der \lstinline$tracking_loop$ Komponente.}
        \label{Tab_tracking_loop_Entity}
    }
    {
        \rowcolors{2}{light-gray}{White}
    \begin{tabular}{c c  p{1.5cm} p{6cm}}
        \toprule
        Name                    & I/O	& Typ				& Beschreibung \\
        \midrule
        \lstinline$channel_num$	& I	& \lstinline$natural$		& Die Kanalnummer des Tracking Loop. Dies wird nur für die Simulation benötigt. \\
        \lstinline$i_clk_ADC$	& I	& \lstinline$std_logic$		& Taktsignal das auch die anderen Teile des Tracking Loop antreibt. Dies sollte das (aufgefrischte) Taktsignal des GPS Frontends sein.\\
        \lstinline$i_reset$	& I	& \lstinline$std_logic$		& Asynchrones Reset Signal (aktiv wenn \lstinline$i_reset='1'$). \\
        \lstinline$i_samples$	& I	& \lstinline$std_logic$		& ADC Samples \\
        \lstinline$i_dmem$	& I	& \lstinline$dmem_out _type$	& Schnittstelle zum MBlite Prozessor. Steuerleitungen und Daten- und Adressbus. \\
        \lstinline$o_dmem$	& O	& \lstinline$dmem_in _type$	& Datenbus zum MBlite Prozessor. \\
        \lstinline$o_intr$	& O	& \lstinline$std_logic$		& Interruptausgang. \lstinline$'1'$ Impuls wenn Integration abgeschlossen (\emph{Dump} Signal).\\
        \bottomrule
    \end{tabular}
}
\end{table}

\begin{table}[htbp]
    \ttabbox
    {
        \caption[Memory Map Tracking Loop]{Register Memory Map der \lstinline$tracking_loop$ Komponente.}
        \label{Tab_tracking_loop_memory_map}
    }
    {
        \rowcolors{2}{light-gray}{White}
    \begin{tabular}{c c  p{1.5cm} p{5.5cm}}
        \toprule
        Name                    & R/W	& Address Offset    & Beschreibung \\
        \midrule
        \lstinline$I_Early$	    & R	    & $0x00$              &  Ausgang des I Early Korrelators. \\
        \lstinline$Q_Early$	    & R	    & $0x02$              &  Ausgang des Q Early Korrelators.\\
        \lstinline$I_Prompt$    & R	    & $0x04$              &  Ausgang des I Prompt Korrelators.\\
        \lstinline$Q_Prompt$    & R	    & $0x06$              &  Ausgang des Q Prompt Korrelators.\\
        \lstinline$I_Late$	    & R	    & $0x08$              &  Ausgang des I Late Korrelators.\\
        \lstinline$Q_Late$	    & R	    & $0x0A$              &  Ausgang des Q Late Korrelators.\\
        \lstinline$code_fcw$    & RW    & $0x0C$              &  Code NCO Frequency Control Word.\\
        \lstinline$carrier_fcw$ & RW    & $0x10$              &  Carrier NCO Frequency Control Word.\\
        \lstinline$PRN$         & RW    & $0x14$              &  PRN des zu trackenden Satellitensignals.\\
        \lstinline$codephase$   & RW    & $0x18$              &  Bei der Acquisiton festgestellte Codephase.\\
        \lstinline$flags$       & RW    & $0x1C$              &  Flags (siehe auch \TR{Tab_TL_flags}).\\
        \bottomrule
    \end{tabular}
}
\end{table}

\begin{table}[htbp]
    \ttabbox
    {
        \caption[Tracking Loop Flag Register]{Bedeutung der Bits im \lstinline$flags$ Register der \lstinline$tracking_loop$ Komponente.}
        \label{Tab_TL_flags}
    }
    {
        \rowcolors{2}{light-gray}{White}
    \begin{tabular}{c c p{6cm}}
        \toprule
        Name                    & Bit Offset    & Beschreibung \\
        \midrule
        \lstinline$enable$	    & $0$             & Eine \lstinline$'1'$ aktiviert den Tracking Kanal. \\
        \lstinline$acq_started$ & $1$             & Wenn angefangen wird Samples für die Acquisition aufzunehmen muss gleichzeitig dieses Bit auf \lstinline$'1'$ gesetzt werden. \\
        \lstinline$config_done$ & $2$             & Wenn alle anderen Konfigurationswerte geschrieben wurden, startet eine \lstinline$'1'$ das Tracking. \\
        \bottomrule
    \end{tabular}
}
\end{table}

\paragraph{Implementierung (Architecture)}

Die Architecture der Tracking Loop Komponente beinhaltet Code- und Carrier Mischer und alle in den vorherigen Abschnitten beschriebenen Blöcke. Außerdem beinhaltet die Architecture noch Steuerlogik in Form einer State Machine, die auf Schreib-/Lesezugriffe von der MBlite Schnittstelle reagiert. Weiterhin steuert die State Machine die Initialisierung des Tracking Loop Kanals.

Wie bereits früher erwähnt, werden lediglich \SI{1}{\bit} Samples verwendet, wobei eine \lstinline$'0'$ als $+1$ interpretiert wird und eine \lstinline$'1'$ als $-1$\footnote{Dies ist genau umgekehrt zu der Interpretation in der Integrate \& Dump Komponente. Da allerdings ohnehin mit einer \SI{\pm180}{\degree} Unsicherheit gerechnet werden muss, ist dies nicht schlimm. Die nachfolgende Signalverarbeitung zur NAV Daten Extraktion invertiert das Signal, falls notwendig.}. Damit können die Mischer einfach als XOR Glieder realisiert werden:

\begin{table}[htbp]
    \ttabbox
    {
        \caption[Exklusiv-Oder Glied]{Vergleich Multiplikation und Exklusiv-Oder (XOR) Glied.}
        \label{TabXOR}
    }
    {
        \rowcolors{2}{light-gray}{White}
    \begin{tabular}{c c c c c c c}
        \toprule
        $a$ 	& $b$ 	& $a\cdot b$ 	& & $A$	& $B$		& $A~ \textrm{XOR} ~B$ \\
        \midrule
        $+1$	& $+1$ 	& $+1$ 		& & \lstinline$'0'$	& \lstinline$'0'$ & \lstinline$'0'$\\
        $+1$	& $-1$ 	& $-1$ 		& & \lstinline$'0'$	& \lstinline$'1'$ & \lstinline$'1'$\\
        $-1$	& $+1$ 	& $-1$ 		& & \lstinline$'1'$	& \lstinline$'0'$ & \lstinline$'1'$\\
        $-1$	& $-1$ 	& $+1$ 		& & \lstinline$'1'$	& \lstinline$'1'$ & \lstinline$'0'$\\
        \bottomrule
    \end{tabular}
}
\end{table}

Die innere Struktur der Architecture orientiert sich grob an dem Blockschaltbild in \FR{Trackingloop.png} wobei die grünen Blöcke, wie bereits erwähnt, in Software in MBlite Prozessor realisiert sind.

\subparagraph{Start Prozedur} In der Beschreibung der Schnittstelle wurden bereits die notwendigen Schritte zum Starten eines Tracking Loop Kanals beschrieben. Durch das Zurücksetzen des Trackingloop (Schritt 1), befinden sich die LFSR, die den \gls{CA} Code erzeugen, im Zustand \lstinline$"1111111111"$, was der Codephase $0$ entspricht. Auch nach Setzen der \lstinline$enable$ Flag in Schritt 2 bleiben die LFSR in diesem Zustand. 

Durch das Setzen des \lstinline$acq_started$ Flag (Schritt 3) wird ein Zähler gestartet, der alle $N=\frac{\textrm{Samples}}{\textrm{Codewort}}$ überläuft. Nachdem die Acquisition und Konfiguration des Kanals abgeschlossen ist (Schritte 4 und 5), wartet die State Machine bis der Zählerstand das nächste Mal den Wert erreicht, der in das \lstinline$codephase$ Register geschrieben wurde und startet dann das LFSR. \lstinline$codephase$ wurde bei der Acquisition als der Wert bestimmt an dem die Codephase gleich Null ist und somit mit dem zurückgesetzten LFSR übereinstimmt. 