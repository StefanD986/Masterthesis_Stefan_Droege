\subsubsection{Code Replika Generator}
Diese Komponente ist in  \lstinline$code_replica_generator_pkg.vhd$ implementiert und erzeugt drei, zueinander um $\gpsTchip /2$ verzögerte, Codekopien (\emph{Early}, \emph{Prompt}, und \emph{Late}).

\paragraph{Schnittstelle (Entity)}
In \TR{TabCodeGen_Entity} ist die Schnittstelle der \lstinline$code_replica_generator$ Komponente beschrieben.

\begin{table}[htbp]
    \ttabbox
    {
        \caption[Code Replika Generator Schnittstelle]{Schnittstellenbeschreibung (Entity) der \lstinline$code_replica_generator$ Komponente.}
        \label{TabCodeGen_Entity}
    }
    {
        \rowcolors{2}{light-gray}{White}
    \begin{tabular}{c c  p{2cm} p{6cm}}
        \toprule
        Name                    & I/O  & Typ                               & Beschreibung \\
        \midrule
        \lstinline$i_clk$       & I         & \lstinline$std_logic$             & Taktsignal das auch die anderen Teile des Tracking Loop antreibt. Dies sollte das (aufgefrischte) Taktsignal des GPS Frontends sein.\\
        \lstinline$i_reset$     & I         & \lstinline$std_logic$             & Asynchrones Reset Signal. Bei \lstinline$i_reset='1'$ werden die LFSR \lstinline$g1$ und \lstinline$g2$ auf den Zustand \lstinline$\"1111111111\"$ zurückgesetzt.\\
        \lstinline$i_nco$    & I         & \lstinline$std_logic$             & Ein \lstinline$record$ mit mehreren Clock Enable Signalen, die von \lstinline$code_nco$ Komponente erzeugt werden.\\
        \lstinline$i_PRN$       & I         & \lstinline$unsigned (5 downto 0)$ & Wählt über die PRN Nummer die zu generierende Gold bzw. \gls{CA} Code Sequenz aus. Zulässige Werte sind im Bereich 1 bis 32.\\
        \lstinline$o_chip$      & O         & \lstinline$t_codegen_out$             & Ein \lstinline$record$ mit den drei Code Kopien.\\
        \bottomrule
    \end{tabular}
}
\end{table}


\FGimg[Entity/Architecture des Code Replika Generators]{UML_class_code_replica_generator.pdf}{Entity und Architecture welche die drei Code Kopien erzeugt.}{9.26cm}

\paragraph{Implementierung (Architecture)}

\begin{table}[htbp]
    \ttabbox
    {
        \caption[Code Replika Generator interne Signale]{Interne Signale der \lstinline$code_replica_generator$ Komponente.}
        \label{TabCodeGen_ArchSignals}
    }
    {
        \rowcolors{2}{light-gray}{White}
    \begin{tabular}{c  p{2cm} p{6cm}}
        \toprule
        Name      & Typ         & Beschreibung \\
        \midrule
        \lstinline$lfsr_out$  & \lstinline$std_logic$             & LFSR Ausgang\\
        \lstinline$shiftregister$  & \lstinline$SLV(2 downto 0)$             &  Verzögerungsleitung realisiert als dreistufiges Schieberegister\\
        \bottomrule
    \end{tabular}
}
\end{table}

Die Architecture der \lstinline$code_replica_generator$ Komponente enthält eine Instanz der \lstinline$ca_lfsr$ Komponente. Dessen \lstinline$i_ClkEna$ Eingang ist mit \lstinline$i_nco.Ena_clk$ verbunden. Zur Erinnerung: Dieses Signal ist ein Impuls, der für genau eine Taktperiode \lstinline$'1'$ ist und sich etwa alle $1/\gpsTchip$ wiederholt. Damit wird mit jedem Impuls ein neuer Chip in dem \gls{LFSR} erzeugt.

Weiterhin enhält die Architecture einen sequenziellen Prozess der ein dreistufiges Schieberegister realisiert, was seinen Inhalt bei \lstinline$i_nco.Ena_clk_x2='1'$ und einer steigenden Taktflanke nach rechts schiebt. Die Impulsfrequenz von \lstinline$i_nco.Ena_clk_x2$ ist doppelt so hoch wie die von \lstinline$i_nco.Ena_clk$, wodurch jede Stufe des Schieberegisters eine Verzögerung von $\gpsTchip /2$ ergibt.

Der Ausgang der ersten Stufe ist \lstinline$o_chip.Early$ zugewiesen, der der zweiten Stufe \lstinline$o_chip.Prompt$, und der der dritten Stufe \lstinline$o_chip.Late$.
