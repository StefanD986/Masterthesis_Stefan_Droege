\subsubsection{EBIU Schnittstelle auf FPGA Seite}

\FGimg[Entity/Architecture der EBIU Schnittstelle auf FPGA Seite]{UML_class_PSFSM2.pdf}{Entity und Architecture der \lstinline$PSFSM2$ Komponente, die das Bindeglied zwischen dem asynchronen Memory Interface des DSP und dem Rohdaten FIFO ist. Die in den \lstinline$t_EBIU_*$ Typen efinierten Signalen entsprechen der in \cite{BlackfinHWReference} definierten EBIU Schnittstelle.}{0.9\imgmaxwidth}

Auf der FPGA Seite muss ein Gegenstück zu dem Asynchronen Memory Interface des DSP implementiert werden. Diese Komponente ist in \lstinline$PSFSM2.vhd$ implementiert. Sie dient als Bindeglied zwischen dem EBIU Interface und dem Rohdaten FIFO.

Die Herausforderung bei dem Asynchronen Memory Interface ist, wie der Name schon sagt, die Asynchronität der Signale zu dem Systemtakt des FPGA: Wenn die Setup \& Hold Zeiten der Flipflops nicht eingehalten werden, kann das FF in einen Metastabilen Zustand gelangen, der erst nach längerer, nicht vorhersagbarer Zeit verlassen wird\footnote{Einige weiterführende Informationen zu dem Thema sind in \cite{FPGAFAQMetastability} und \cite{ginosar2011metastability} zu finden. Eine umfangreiche Artikelsammlung zu dem Thema bietet \cite{MetastabilityBibliography}}. 
Eine perfekte Lösung Metastabilität im ersten Flipflop, auf das das Eingangssignal trifft, zu verhindern existiert nicht. Es gibt lediglich die Möglichkeit durch das Einbauen von Wartezyklen die Wahrscheinlichkeit, dass der metastabile Zustand auch an nachfolgende Flipflop Stufen propagiert wird, zu verringern. Damit lässt sich die \gls{MTBF} auf ein akzeptables Maß vergrößern. Genau dies wird auch bei dem Asynchronen Memory Interface der EBIU gemacht. Die Einstellung der Wartezeiten wurde in \TR{TabEBIUTimingConfig} gegeben.

Außerdem wird durch die PSFSM2 die Memory Map festgelegt, also unter welcher Adresse der DSP den FIFO bzw. dessen Statusregister auslesen kann (\TR{Tab_RohdatenMemMap}).

\paragraph{Schnittstelle (Entity)}
In \TR{TabPSFSM2_Entity} ist die Schnittstelle der \lstinline$PSFSM2$ Komponente beschrieben. In \FR{UML_class_PSFSM2.pdf} sind außerdem die von der Entity verwendeten Typen dargestellt.

\begin{table}[htbp]
    \ttabbox
    {
        \caption[PSFSM2 Schnittstelle]{Schnittstellenbeschreibung (Entity) der \lstinline$PSFSM2$ Komponente.}
        \label{TabPSFSM2_Entity}
    }
    {
        \rowcolors{2}{light-gray}{White}
    \begin{tabular}{c c  p{2.5cm} p{5.5cm}}
        \toprule
        Name                & I/O	& Typ				                & Beschreibung \\
        \midrule
        \lstinline$clk$	    & I	    & \lstinline$std_logic$	        	& FPGA Systemtakt\\
        \lstinline$reset$	& I	    & \lstinline$std_logic$	        	& Asynchrones Reset Signal (aktiv wenn \lstinline$reset='1'$) \\
        \lstinline$i_EBIU$	& I	    & \lstinline$t_EBIU_out$	    	& EBIU Signale vom DSP zum FPGA\\
        \lstinline$o_EBIU$	& O	    & \lstinline$t_EBIU_in$		        & EBIU Signale vom FPGA zum DSP\\
        \lstinline$i_FIFO$	& I	    & \lstinline$t_fifo_ps_Read_out$	& Signale von PSFSM2 Komponente zum Rohdaten FIFO \\
        \lstinline$o_FIFO$	& O	    & \lstinline$t_fifo_ps_Read_out$	& Signale vom Rohdaten FIFO zur PSFSM2 Komponente \\
        \bottomrule
    \end{tabular}
}
\end{table}


\paragraph{Implementierung (Architecture)}

\begin{table}[htbp]
    \ttabbox
    {
        \caption[Interne Signale der PSFSM2 Architecture]{Interne Signale der \lstinline$PSFSM2$ Komponente.}
        \label{TabPSFSM2_ArchSignals}
    }
    {
        \rowcolors{2}{light-gray}{White}
    \begin{tabular}{c  p{2cm} p{6cm}}
        \toprule
        Name      		& Typ         & Beschreibung \\
        \midrule
        \lstinline$r$		& \lstinline$reg_type$ & \gls{FSM} Zustandsregister.\\
        \lstinline$r_next$	& \lstinline$reg_type$ & Signal für den zukünftigen Zustand der \gls{FSM}.\\
        \bottomrule
    \end{tabular}
}
\end{table}

\begin{table}[htbp]
    \ttabbox
    {
        \caption[Typdefinition PSFSM2 Zustandsregister]{Beschreibung der Struktur des Zustandsregisters der State Machine in der \lstinline$PSFSM2$ Komponente (\lstinline$reg_type$ Typ).}
        \label{Tab_reg_typ}
    }
    {
        \rowcolors{2}{light-gray}{White}
    \begin{tabular}{c  p{2cm} p{6cm}}
        \toprule
        Name				        & Typ						        & Beschreibung \\
        \midrule
        \lstinline$state$		    & \lstinline$state_type$	        & \gls{FSM} Zustand.\\
        \lstinline$psfsm_to_EBIU$	& \lstinline$t_EBIU_in$	            &  Ausgangsregister.\\
        \lstinline$psfsm_to_fifo$   & \lstinline$t_fifo_ps_Read_in$     & Register für Steuersignale zum FIFO.\\
        \lstinline$fifo_status$     & \lstinline$SLV(15 downto 0)$      & FIFO Füllstand und Full/Empty Flags\\
	    \lstinline$readcount$       & \lstinline$SLV(15 downto 0)$      & Register für Testzwecke (siehe Architecture Beschreibung.)\\
        \bottomrule
    \end{tabular}
}
\end{table}

\begin{table}[htbp]
    \ttabbox
    {
        \caption[Memory Map Rohdatenschnittstelle]{Die Register die über die EBIU Schnittstelle ausgelesen werden können. Alle Register sind \SI{16}{\bit} breit.}
        \label{Tab_RohdatenMemMap}
    }
    {
        \rowcolors{2}{light-gray}{White}
    \begin{tabular}{c  l p{6cm}}
        \toprule
        Adresse				    & Bezeichnung			& Beschreibung \\
        \midrule
        \lstinline$0x20000000$  & \lstinline$status0$   & Enthält FIFO Füllstand (Bit 15-2) und Status Flags (Bit 1: Full; Bit 0: Empty).\\
        \lstinline$0x20000020$  & \lstinline$test$      & Register um Datenübertragung zwischen FPGA und DSP zu testen (siehe auch Beschreibung im Text).\\
	    \lstinline$0x20000040$  & \lstinline$fifo_read$ & FIFO Ausgangsregister\\
        \bottomrule
    \end{tabular}
}
\end{table}

%Die Signale der EBIU Schnittstelle (zusammengefasst im Typ \lstinline$t_EBIU_out$) steuern eine \gls{FSM}. Die \gls{FSM} mit dem Systemtakt des FPGAs getaktet, damit die Schnittstelle synchron zu dem Systemtakt der restlichen Komponenten im FPGA ist. Je nachdem von welcher Adresse der DSP Daten angefordert hat sorgt die \gls{FSM} dafür die Daten aus FIFO, Statusregister, oder Test Register auf den EBIU Datenbus zu legen.

Das \lstinline$status0$ Register beinhaltet in den Bits 15 bis 2 den Füllstand des FIFOs. Dieser Wert ist immer garantiert kleiner oder gleich dem momentanen Füllstand. Dadurch ist sichergestellt, dass wenn der DSP diesen Wert abfragt und im nächsten Schritt genau so viele Daten aus dem FIFO ausliest, kein FIFO \emph{Unterlauf} auftritt. Die untersten beiden Bits enthalten die \emph{Full} bzw. \emph{Empty} Flags des FIFOs.

Das \lstinline$fifo_read$ Register ist der Ausgang des Rohdatenpuffers. Nachdem ein Wort von dieser Adresse gelesen wird, wird es aus dem Puffer gelöscht und das nächste Wort steht bereit. So kann durch wiederholtes Lesen von der \lstinline$fifo_read$ Adresse nach und nach der gesamte Puffer ausgelesen werden.

Das \lstinline$test$ Register erfüllt keinen für den Betrieb wichtigen Zweck, kann jedoch genutzt werden um die Übertragung zwischen FPGA und DSP zu testen. Mit jedem Lesevorgang auf dieses Register wird der Inhalt des Registers inkrementiert. Wenn keine Fehler bei der Übertragung auftreten sollte der DSP also eine lückenlose Sequenz von Zahlen in aufsteigender Reihenfolge empfangen (mit Ausnahme des Überlaufs von $2^{16}-1$ auf $0$).
