\subsubsection{Carrier NCO}
Diese Komponente ist in  \lstinline$carrier_nco.vhd$ implementiert und erzeugt die zwei zueinander \SI{90}{\degree} verschobenen LO Signale, mit einer zur Laufzeit einstellbaren Frequenz. Die Frequenz wird durch das \gls{FCW} bestimmt, welches nicht \emph{gleich}, aber \emph{proportional} zu der Frequenz $f_{out}$ des Ausgangssignals ist:

\begin{equation}
    FCW=f_{out}\cdot K_{0,carrier} =f_{out}\cdot  \frac{2^{N_{carrierNCO}}}{f_{clk}}
\end{equation}
mit der Frequenz $f_{clk}$ des Signals \lstinline$i_clk$. 

\FGimg[Entity/Architecture des Carrier NCO]{UML_class_carrier_nco.pdf}{Entity und Architecture welche die Trägerkopie erzeugt.}{10cm}

\paragraph{Schnittstelle (Entity)}
In \TR{TabCarrierNCO_Entity} ist die Schnittstelle der \lstinline$carrier_nco$ Komponente beschrieben.

\begin{table}[htbp]
    \ttabbox
    {
        \caption[Carrier NCO Schnittstelle]{Schnittstellenbeschreibung (Entity) der \lstinline$carrier_nco$ Komponente.}
        \label{TabCarrierNCO_Entity}
    }
    {
        \rowcolors{2}{light-gray}{White}
    \begin{tabular}{c c  p{2cm} p{6cm}}
        \toprule
        Name                    & I/O  & Typ                               & Beschreibung \\
        \midrule
        \lstinline$i_clk$       & I         & \lstinline$std_logic$             & Taktsignal das auch die anderen Teile des Tracking Loop antreibt. Dies sollte das (aufgefrischte) Taktsignal des GPS Frontends sein.\\
        \lstinline$i_reset$     & I         & \lstinline$std_logic$             & Asynchrones Reset Signal (aktiv wenn \lstinline$i_reset='1'$) \\
        \lstinline$i_deltaf$    & I         & \lstinline$t_carrier_fcw$             & Frequency Control Word \\
        \lstinline$o_nco$       & O         & \lstinline$t_carrier_nco_out$ & \lstinline$record$ der die beiden LO Signale (I und Q) enthält. \\
        \bottomrule
    \end{tabular}
}
\end{table}

\paragraph{Implementierung (Architecture)}

\begin{table}[htbp]
    \ttabbox
    {
        \caption[Carrier NCO interne Signale]{Interne Signale der \lstinline$carrier_nco$ Komponente.}
        \label{TabCarrierNCO_ArchSignals}
    }
    {
        \rowcolors{2}{light-gray}{White}
    \begin{tabular}{c  p{2cm} p{6cm}}
        \toprule
        Name      & Typ         & Beschreibung \\
        \midrule
        \lstinline$count$  & \lstinline$unsigned (N_carrier_nco-1 DOWNTO 0)$             & \gls{FSM} Zustandsregister\\
        \lstinline$LUsin$  & \lstinline$SLV(3 downto 1)$ & Konstante mit einer Sinus Look-Up-Table.\\
        \lstinline$LUcos$  & \lstinline$SLV(3 downto 1)$ & Konstante mit einer Cosinus Look-Up-Table.\\
        \bottomrule
    \end{tabular}
}
\end{table}

Der NCO besteht aus einem Phasenakkumulator, und einer \gls{LUT}, die eine Zuordnung von Amplitudenwerten $f(\phi)$ zu einem Phasenwert $\phi$ enthält. Da hier lediglich mit \SI{1}{\bit} Werten gearbeitet wird, gestaltet sich die \gls{LUT} sehr einfach: Für Werte von $\phi$ an denen $f(\phi)\geq 0$ ist, enthält sie eine \lstinline$'1'$. Für Werte von $\phi$ bei denen $f(\phi)<0$ ist, enthält sie eine \lstinline$'0'$. Hier ein Ausschnitt aus dem Code:

\begin{lstlisting}
    constant LUsin 	: std_logic_vector(3 DOWNTO 0):= "1100"; --! The `sin(x)` Look-up-table
	constant LUcos 	: std_logic_vector(3 DOWNTO 0):= "1001"; --! The `cos(x)` Look-up-table
\end{lstlisting}

Der Phasenakkumulator ist in dieser Implementierung \SI{28}{\bit} weit, und ist als sequentieller Prozess realisiert. Mit jeder steigenden Taktflanke wird zu dem Phasenakkumulator der Wert an \lstinline$i_deltaf$ addiert. Die beiden MSB des Akkumulators werden dann zur Addressierung der \gls{LUT} verwendet, um die Amplitudenwerte zu erhalten. Damit wird auch klar woher das oben genannte NCO Gain $K_{0,carrier}=\frac{2^{N_{carrierNCO}}}{f_{clk}}$ kommt.
Die Frequenz des Ausgangssignals ergibt sich damit 

