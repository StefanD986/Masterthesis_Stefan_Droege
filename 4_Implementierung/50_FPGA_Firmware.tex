\section{FPGA Firmware}
In diesem Abschnitt wird die im Rahmen dieser Arbeit entwickelte Implementierung auf FPGA Seite besprochen. Begonnen wird mit einem Überblick und einer Beschreibung der Top-Level Entity und der Aufbereitung des Taktsignals vom GPS Frontend. Der MBlite Prozessor selbst wird hier nicht im Detail besprochen\footnote{Der MBlite Softcore ist in \cite{MBliteThesis} beschrieben.}, aber die Firmware die für den MBlite Prozessor geschrieben wurde und der Interrupt Controller der entwickelt wurde. Danach beschäftigt sich der größte Teil mit der Implementierung des Tracking Loop.

Da die Dokumentation in dieser Arbeit nur den derzeitigen Entwicklungsstand widerspiegelt sei für zukünftige Weiterentwicklungen auch auf die Doxygen Dokumentation\footnote{Die Doxygen Dokumentation ist im git Repository unter \lstinline$gps_fpga/doc$ zu finden.} verwiesen, die unter Umständen aktuellere Informationen enthält. 

Im Folgenden wird von UML Klassendiagrammen Gebrauch gemacht um Beziehungen zwischen den Komponenten darzustellen. Deshalb soll kurz erklärt werden, wie diese im Zusammenhang mit der Programmiersprache VHDL zu interpretieren sind: Die Hardwarebeschreibung in VHDL gliedert sich in \emph{Components}. Ein \emph{Component} ist die Instanz einer Einheit aus \emph{Entity} und  \emph{Architecture}. Eine \emph{Entity} beschreibt die Schnittstelle eines \emph{Component}, und die \emph{Architecture} den inneren Aufbau und Funktion des \emph{Component}. Die \emph{Architecture} Aufbau besteht dabei aus kombinatorischer oder sequentieller Logik, kann aber auch wiederum andere \emph{Components} enthalten.

Typdefinitionen und Componentdeklarationen usw. können zur einfacheren Verwendung in \emph{Packages} zusammengefasst werden. Außerdem können \emph{Entities}, \emph{Architectures} und \emph{Packages} in \emph{Libraries} zusammengefasst werden\footnote{Wenn ein Element nicht aktiv einer Bibliothek hinzugefügt wird, so wird es implizit der \lstinline$work$ Bibliothek hinzugefügt.}. So wurden zum Beispiel die Komponenten des Trackingloop in der \lstinline$gps$ Bibliothek zusammengefasst. Die meisten Teile des MBlite Softcores sind Teil der \lstinline$mblite$ Bibliothek.

In den UML Klassendiagrammen\footnote{In den Diagrammen wird der VHDL \lstinline$std_logic_vector$ Typ abkürzend als \emph{SLV} bezeichnet.} sind die \emph{Entities} als  \emph{Interfaces} modelliert, wobei die VHDL Ports als Attribute mit Sichtbarkeit \emph{public} dargestellt sind. Operationen bietet die Schnittstelle selbst keine. Die \emph{Architectures} sind die Klassen, die eine Interface realisieren. Die in den Architectures enhaltenen Signale sind als Attribute mit Sichtbarkeit \emph{private} modelliert. Wenn eine \emph{Architecture} andere \emph{Components} enthält ist das als UML Komposition modelliert.

\subsection{Top Level Entity}\label{Section_gps_top_entity}
Die Top Level Entity ist, wie der Name sagt, die übergeordnete Komponente die zum Einen Schnittstellen zu DSP, MAX2769, und Clock bereitstellt und zum Anderen in der dazugehörigen Architecture die anderen Komponenten enthält und verbindet.

In \FR{UML_comp_structure_gps_top.pdf} ist der Aufbau der \lstinline$gps_top$ Komponente etwas vereinfacht dargestellt. Nicht dargestellt ist die in Abschnitt \ref{RohdatenFIFOSchnittstelle} beschriebene Rohdaten Schnittstelle zum DSP, die zu Anfang der Entwicklungsphase genutzt wurde um Rohdaten auf den DSP zu übertragen. Diese wird in einem eigenen Abschnitt beschrieben, da sie zu diesem Zeitpunkt aus den in Abschnitt \ref{UserIOSchnittstelle} genannten Gründen noch nicht mit dem derzeitigen Entwicklungsstand zusammengeführt wurde. Aus Gründen der Übersicht ist in \FR{UML_comp_structure_gps_top.pdf} der Clock Synthesizer nicht abgebildet. Der Clock Synthesizer frischt das vom GPS Frontend kommende Taktsignal auf und versorgt mit dem aufgefrischten Signal die anderen Komponenten.

\FGimg[Übersicht FPGA Firmware]{UML_comp_structure_gps_top.pdf}{Vereinfachte Struktur der Top Level Komponente \lstinline$gps_top$ und einige der verwendeten Typen.}{0.95\imgmaxwidth}

Wie bereits in Abschnitt \ref{SoftcoreAuswahl} erwähnt ist der  MBlite Prozessor eine Implementierung des Microblaze und hat somit eine \emph{Harvard Architektur} bei der Data und Instruction Memory über zwei getrennte Busse angesprochen werden.

Für beide Speicher wird in dieser Arbeit der interne Dual Port Block RAM des FPGA genutzt. Dabei gibt es zwei Möglichkeiten die Speicher zu implementieren: Es können entweder zwei getrennte Block RAM Instanzen als Single Port RAM benutzt werden. Oder es wird die Dual Port Funktionalität genutzt und beide Speicher liegen in  einer gemeinsamen Block RAM Instanz. Von der Funktion unterscheiden sich beide Varianten nicht, aber erstere Variante vereinfacht das Erstellen der Hexfiles etwas, weshalb hier zwei Block RAM Instanzen verwendet werden (siehe dazu auch Abschnitt \ref{MBliteFirmware}).

Der MBlite Prozessorkern ist direkt über den ersten Bus mit dem Instruction Memory verbunden. Der zweite Bus geht an einen Address Decoder. Dieser wertet die Adresse aus und entscheidet anhand einer Memory Map (festgelegt in \lstinline$config_Pkg.vhd$) welche der an ihn angeschlossenen Komponenten Zugriff auf den Datenbus erhält.

Die an den Address Decoder angeschlossenen Komponenten sind der Data Memory, die sechs Tracking Loop Kanäle und der Interrupt Controller. Diese Komponenten und der Clock Synthesizer werden in den weiteren Abschnitten einzeln beschrieben.

\subsection{Clock Synthesizer}
In \FR{Eye_Diagram_MAX2769_output_color2.png} ist eine Oszilloskop Messung des vom GPS Frontend kommenden Daten- und Taktsignals dargestellt. Wie man sieht sind die Taktflanken stark abgerundet. Obwohl das Taktsignal innerhalb des FPGAs durch einen \lstinline$ibufg$ Buffer aufgefrischt wird, hat das schlechte Taktsignal bei der Implementierung zu einigen schwer zu findenden Fehlern geführt: Durch die abgerundeten Flanken reicht die Energie nicht immer aus um den Buffer zuverlässig zu treiben, was sich in fehlenden Taktflanken bemerkbar macht. Eine Messung die das Problem illustriert ist in \FR{ODDR_Buffer_Runt_Pulse_and_ADC_clock.png} zu sehen.

Deshalb wird das Taktsignal im FPGA durch einen Clock Synthesizer aufgefrischt. Im \emph{XC6SLX9} FPGA stehen dazu verschiedene \glspl{Primitive} zur Verfügung: \lstinline$DCM_SP$, \lstinline$DCM_CLKGEN$, \lstinline$PLL_BASE$ oder \lstinline$PLL_ADV$. Für den hier vorliegenden Fall zeigte sich das \lstinline$DCM_CLKGEN$ \gls{Primitive} als am besten geeignet, da es robust gegenüber unzuverlässigen Taktsignalen mit hohem Jitter und/oder Rauschen ist. Eine Auffrischung mit dem \lstinline$DCM_SP$ \gls{Primitive} wurde ebenfalls ausprobiert, zeigte sich jedoch anfällig gegenüber dem schlechten Taktsignal. Eine genaue Beschreibung der \glspl{Primitive} ist in \cite{SP6Clock} zu finden.

\FGimg[MAX2769 Takt- und Ausgangssignal]{Eye_Diagram_MAX2769_output_color2.png}{Daten- (oben) und Taktsignal (unten) des MAX2769 GPS Frontend.}{0.8\imgmaxwidth}

\FGimg[Fehlende Taktflanken]{ODDR_Buffer_Runt_Pulse_and_ADC_clock.png}{Taktsignal des GPS Fronted (unten) und das durch einen \lstinline$ibufg$ Buffer aufgefrischte Signal (oben). Wie man sieht reicht die Energie des Taktsignals nicht immer aus den Clock Buffer zu treiben. Für die Messung wurde das aufgefrischte Taktsignal über ein \lstinline$ODDR2$ \gls{Primitive} nach außen geführt.}{0.8\imgmaxwidth}

Die \lstinline$phaseshift180$ Entity (\FR{UML_class_phaseshift180.pdf}) ist die Schnittstelle zu dem \lstinline$DCM_CLKGEN$  Clock Synthesizer \gls{Primitive}. Die Entity wurde mit dem Xilinx Core Generator erzeugt und die einzelnen Signale der Schnittstelle sind in \TR{TabPhaseshift180_Entity} beschrieben.

In \FR{Eye_Diagram_MAX2769_output_color2.png} ist zu sehen, dass das GPS Frontend die Daten auf der steigenden Taktflanke ändert. Damit die Daten von den weiteren Stufen im FPGA sicher eingelesen werden, werden die weiteren Stufen mit dem \SI{180}{\degree} verschobenen Taktsignal \lstinline$o_clk_180$ getaktet. So ist sichergestellt, dass die Setup und Hold Zeiten der Flipflops eingehalten werden.

\FGimg[Clock Synthesizer Schnittstelle]{UML_class_phaseshift180.pdf}{Die \lstinline$phaseshift180$ Entity ist die Schnittstelle zum \lstinline$DCM_CLKGEN$ Clock Synthesizer \gls{Primitive}.}{4cm}

\begin{table}[htbp]
    \ttabbox
    {
        \caption[Clock Synthesizer Schnittstelle]{Schnittstellenbeschreibung (Entity) der \lstinline$phaseshift180$ Komponente.}
        \label{TabPhaseshift180_Entity}
    }
    {
        \rowcolors{2}{light-gray}{White}
    \begin{tabular}{c c  l p{5.8cm}}
        \toprule
        Name                    & I/O   & Typ                   & Beschreibung \\
        \midrule
        \lstinline$i_clk$       & I     & \lstinline$std_logic$ & Eingang für das aufzufrischende GPS Frontend Taktsignal. \\
        \lstinline$i_reset$     & I     & \lstinline$std_logic$ & Asynchrones Reset Signal (aktiv wenn \lstinline$i_reset='1'$). \\
        \lstinline$o_clk_buf$   & O     & \lstinline$std_logic$ & Aufgefrischtes Taktsignal, synchron und Phasengleich zu \lstinline$i_clk$. \\
        \lstinline$o_clk_180$   & O     & \lstinline$std_logic$ & Aufgefrischtes Taktsignal, \SI{180}{\degree} phasenverschoben aber sonst synchron zu \lstinline$i_clk$. \\
        \lstinline$o_CLK_VALID$ & O     & \lstinline$std_logic$ & Statussignal, \lstinline$'1'$ wenn Clock Synthesizer eingerastet ist. Für Details zu diesem Signal siehe \cite{SP6Clock}.\\
        \bottomrule
    \end{tabular}
}
\end{table}




%Instruction und Data Memory. Addressmapping.

\subsection{Interrupt Controller}
Die sechs Tracking Loop Kanäle können unabhängig voneinander Interrupt Signale liefern. Der MBlite Prozessor hat allerdings nur einen Interrupt Eingang. Daher wurde ein Interrupt Controller entwickelt, der eine beliebige Anzahl an Interrupts entgegennehmen kann und an den MBlite Prozessor weiterleitet. Das besondere an dem hier entwickelten Interrupt Controller ist, dass die Signale in einem FIFO gespeichert werden und so sichergestellt wird, dass die Interrupts in der Eingangsreihenfolge der Ereignisse abgearbeitet werden. Dies ist wichtig um die Echtzeitanforderungen einzuhalten. Die Komponente ist in \lstinline$interrupt_ctrl.vhd$ implementiert. 

\FGimg[Entity/Architecture des Interrupt Controllers]{UML_class_IRQctrl.pdf}{Entity und Architecture des Interrupt Controllers.}{8cm}

\paragraph{Schnittstelle (Entity)}
In \TR{TabIRQCtrl_Entity} ist die Schnittstelle der \lstinline$interrupt_ctrl$ Komponente beschrieben.

Die Anzahl der Interrupt Quellen und die maximal mögliche Länge der Warteschlange kann über die zwei \emph{Generics} \lstinline$n_intr$ bzw. \lstinline$max_pending$ konfiguriert werden. Das Interrupt Signal einer Quelle wird mit einem Element des Vektors \lstinline$i_intr$ verbunden. Der Index des Bits in dem Vektor ist gleichzeitig die Interruptnummer, d.h. dass eine Quelle die mit \lstinline$i_intr(3)$ verbunden ist die Nummer 3 zugewiesen bekommt.

Der Signal \lstinline$o_intr$ wird mit dem Interrupt Eingang des MBlite Prozessors verbunden. 

Ein Interrupt muss von den Quellen durch einen  \lstinline$'1'$ Impuls signalisiert werden, das heißt, dass die Leitung nur für eine Taktperiode auf  \lstinline$'1'$ gehalten werden soll und danach wieder auf  \lstinline$'0'$ gesetzt wird. Der Interrupt Controller wird daraufhin die Nummer des Interrupts in der FIFO Warteschlange speichern. Je Taktperiode kann ein Ereignis in die Warteschlange eingefügt werden. Bei mehreren gleichzeitig auftretenden Interrupt Ereignissen gehen die weiteren Ereignisse jedoch \emph{nicht} verloren, sondern werden während der darauf folgenden Taktperioden eingefügt. So lange sich Interrupts in der Warteschlange befinden, ist \lstinline$o_intr='1'$. 

In der \gls{ISR} muss die CPU dann zuerst die Quelle des Interrupts abfragen. Dazu ist der Ausgang des FIFO über den Address Decoder an den Datenbus der CPU angeschlossen. Die Adresse unter der der Interrupt Controller zu finden ist wird über die Memory Map in \lstinline$config_Pkg.vhd$ festgelegt. Lesen von dieser Adresse liefert die Nummer der Interrupt Quelle. Anschließend muss die CPU den Interrupt bestätigen um ihn aus der Warteschlange zu löschen. Dazu wird auf die selbe Adresse ein Schreibvorgang ausgeführt. Dabei ist es unwichtig welcher Wert geschrieben wird, intern wertet die Interrupt Controller Komponente nur das \emph{Write Enable} Signal aus (\lstinline$i_dmem.we_o$).

\subparagraph{Einschränkungen} Es gibt einige Einschränkungen die bei der Nutzung des Interrupt Controllers zu beachten sind:
\begin{enumerate}
    \item Im Falle von Interrupt Ereignissen die in der selben Taktperiode auftreten, werden die Ereignisse in den darauf folgenden Taktperioden nach ihrer Interrupt Nummer, in aufsteigender Reihenfolge in die Warteschlange eingefügt.
    \item Wenn noch \emph{nicht} alle Ereignisse in die Warteschlange eingefügt wurden (siehe Punkt 1) und neue Ereignisse auftreten, so werden die neuen und alten Ereignisse in aufsteigender Reihenfolge in die Warteschlange eingefügt. Dadurch kann eventuell ein später aufgetretenes Ereignis mit niedriger Interruptnummer vor einem früheren Ereignis mit höherer Nummer in die Warteschlange eingefügt werden.
    \item Bevor ein erneuter Interrupt von der selben Quelle verarbeitet werden kann, muss der frühere Interrupt in die Warteschlange eingefügt worden sein.
\end{enumerate}

In der Praxis sind diese Einschränkungen jedoch kein Problem, da sie lediglich zum tragen kommen wenn die Ereignisse in sehr kurzer Zeitabfolge (wenige Taktperioden) aufeinander folgen. Eine Änderung der Bearbeitungsreihenfolge ist in diesen Fällen nicht gravierend.

\begin{table}[htbp]
    \ttabbox
    {
        \caption[Code Replika Generator Schnittstelle]{Schnittstellenbeschreibung (Entity) der \lstinline$interrupt_ctrl$ Komponente.}
        \label{TabIRQCtrl_Entity}
    }
    {
        \rowcolors{2}{light-gray}{White}
    \begin{tabular}{p{1.5cm} c  p{2cm} p{5.5cm}}
        \toprule
        Name                        & I/O       & Typ                       & Beschreibung \\
        \midrule
        \lstinline$n_intr$          & I         & \lstinline$positive$      & \lstinline$generic$, dass die Anzahl der Interruptquellen festlegt.\\
        \lstinline$max _pending$     & I         & \lstinline$positive$      & \lstinline$generic$, dass die maximale Anzahl von unbearbeiten Interrupts in der Warteschlange festlegt.\\
        \lstinline$i_clk$           & I         & \lstinline$std_logic$     & Taktsignal das auch den MBlite Core antreibt.\\
        \lstinline$i_reset$         & I         & \lstinline$std_logic$     & Asynchrones Reset Signal.\\
        \lstinline$i_intr$          & I         & \lstinline$SLV(n_intr -1 DOWNTO 0)$    & Ein Vektor mit den Interruptsignalen der Interruptquellen. Eine \lstinline$'1'$ signalisiert einen Interrupt. Die Leitung sollte nur für eine Taktperiode auf \lstinline$'1'$ gehalten werden.\\
        \lstinline$i_dmem$          & I         & \lstinline$dmem_out _type$            & Schnittstelle zum MBlite Prozessor. Steuerleitungen und Daten- und Adressbus.\\
        \lstinline$o_dmem$          & O	        & \lstinline$dmem_in _type$	& Datenbus zum MBlite Prozessor. \\
        \lstinline$o_intr$          & O         & \lstinline$std_logic$     & Interruptleitung zum MBlite Prozessor. \lstinline$'1'$ solange Interrupts in der Warteschlange stehen.\\
        \bottomrule
    \end{tabular}
}
\end{table}



\subsection{MBlite Firmware}
\label{MBliteFirmware}
Ein im MBlite Prozessor laufendes Programm realisiert die \emph{TrackingSW} Komponente (\FR{GPS_UML_deployment1.png}). Diese realisiert die in \FR{Trackingloop.png} grün markierten Funktionsblöcke, also die Carrier und Code Loop Diskriminatoren und Filter. Die zeitkritische Neuberechnung der Werte ist vollständig Interrupt gesteuert, wodurch im Hauptprozess andere Aufgaben bearbeitet werden können, die weniger zeitkritisch sind. So würde beispielsweise die Initialisierung eines Tracking Loop Kanals oder die Kommunikation mit dem DSP im Hauptprozess ablaufen\footnote{Die Kommunikation mit dem DSP ist derzeit noch nicht implementiert.}.



\FGimg[MBlite Firmware Übersicht]{UML_class_MBliteFirmware.pdf}{Klassendiagramm MBlite Firmware}{0.9\textwidth}

\subsubsection{\lstinline$trackingloop.c$}
% Adressoffset? Makefile?
% Cordic



% Synchronität
%\subsection{Schnittstellen}

\subsection{Tracking Loop}
Der Trackingloop ist teils in Hardware und Teils in Software im MBlite Softcore Prozessor implementiert. In diesem Teil wird der Hardware Teil beschrieben. Die Bezihungen der einzelnen Klassen des Tracking Loop sind in \FR{Impl_UML_TL_classdiagram.pdf} dargestellt. Alle dort abgebildeten Komponenten sind in der Bibliothek \lstinline$gps$ zusammengefasst. Nicht dargestellt ist das \emph{Package} \lstinline$tracking_loop_pkg$, welches die Typdefinitionen, einige Konstanten und Component Deklarationen zusammenfasst.

\FGimg[Klassendiagramm des Tracking Loop]{Impl_UML_TL_classdiagram.pdf}{Klassendiagramm des Trackingloop. Die VHDL \emph{Entities} sind als UML \emph{Interfaces} modelliert, und die dazugehörigen VHDL \emph{Architectures} als UML \emph{Klassen}}{0.95\textwidth}

\subsubsection{tracking\_loop}
\FGimg[Entity/Architecture des Carrier NCO]{UML_class_carrier_nco.pdf}{Entity und Architecture welche die Trägerkopie erzeugt.}{10cm}

\FGimg[Entity/Architecture des Code NCO]{UML_class_code_nco.pdf}{Entity und Architecture welche die Takt Steuersignale für den Code Kopie Generator erzeugt.}{7cm}

\FGimg[Entity/Architecture des Code Replika Generators]{UML_class_code_replica_generator.pdf}{Entity und Architecture welche die drei Code Kopien erzeugt.}{9cm}

\FGimg[Entity/Architecture des Integrate and Dump Blocks]{UML_class_IandD.pdf}{Entity und Architecture des Integrate and Dump Blocks.}{6.4cm}

\FGimg[Entity/Architecture des Gold Code LFSR]{UML_class_LFSR.pdf}{Entity und Architecture des \gls{LFSR}, das die Gold Code Sequenz erzeugt.}{10cm}

\FGimg[Entity/Architecture Tracking Loop]{UML_class_tracking_loop.pdf}{Die übergeordnete Entity/Architecture des Tracking Loop}{10cm}

\subsubsection{Gold Code LFSR}
\subsubsection{Code Replika Generator}
\subsubsection{Code NCO}
\subsubsection{Carrier NCO}
\subsubsection{Korrelatoren und Integrate \& Dump}

\subsubsection{EBIU Schnittstelle auf FPGA Seite}

\FGimg[Entity/Architecture der EBIU Schnittstelle auf FPGA Seite]{UML_class_PSFSM2.pdf}{Entity und Architecture der \lstinline$PSFSM2$ Komponente, die das Bindeglied zwischen dem asynchronen Memory Interface des DSP und dem Rohdaten FIFO ist.}{0.9\textwidth}

Auf der FPGA Seite muss ein Gegenstück zu dem Asynchronen Memory Interface des DSP implementiert werden. Diese Komponente ist in \lstinline$PSFSM2.vhd$ implementiert. Sie dient als Bindeglied zwischen dem EBIU Interface und dem Rohdaten FIFO.

Die Herausforderung bei dem Asynchronen Memory Interface ist, wie der Name schon sagt, die Asynchronität der Signale zu dem Systemtakt des FPGA: Wenn die Setup \& Hold Zeiten der Flipflops nicht eingehalten werden, kann das FF in einen Metastabilen Zustand gelangen, der erst nach längerer, nicht vorhersagbarer Zeit verlassen wird\footnote{Einige weiterführende Informationen zu dem Thema sind in \cite{FPGAFAQMetastability} und \cite{ginosar2011metastability} zu finden. Eine umfangreiche Artikelsammlung zu dem Thema bietet \cite{MetastabilityBibliography}}. 
Eine perfekte Lösung Metastabilität im ersten Flipflop, auf das das Eingangssignal trifft, zu verhindern existiert nicht. Es gibt lediglich die Möglichkeit durch das Einbauen von Wartezyklen die Wahrscheinlichkeit, dass der metastabile Zustand auch an nachfolgende Flipflop Stufen propagiert wird zu verringern. Damit lässt sich die \gls{MTBF} auf ein akzeptables Maß vergrößern. Genau dies wird auch bei dem Asynchronen Memory Interface der EBIU gemacht. Die Einstellung der Wartezeiten wurde in \TR{TabEBIUTimingConfig} gegeben.

Die Signale der EBIU Schnittstelle (zusammengefasst im Typ \lstinline$t_EBIU_out$) steuern eine \gls{FSM}, die je nach Zustand die Lesezugriffe an den FIFO weiterleitet. Die \gls{FSM} mit dem Systemtakt des FPGAs getaktet, damit die Schnittstelle synchron zu dem Systemtakt der restlichen Komponenten im FPGA ist.

\paragraph{Schnittstelle (Entity)}
In \TR{TabPSFSM_Entity} ist die Schnittstelle der \lstinline$PSFSM2$ Komponente beschrieben. In \FR{UML_class_PSFSM2.pdf} ist der \lstinline$t_IandD_accu$ Typ definiert.

\begin{table}[htbp]
    \ttabbox
    {
        \caption[Carrier NCO Schnittstelle]{Schnittstellenbeschreibung (Entity) der \lstinline$carrier_nco$ Komponente.}
        \label{TabPSFSM_Entity}
    }
    {
        \rowcolors{2}{light-gray}{White}
    \begin{tabular}{c c  p{2cm} p{6cm}}
        \toprule
        Name                & I/O	& Typ				                & Beschreibung \\
        \midrule
        \lstinline$clk$	    & I	    & \lstinline$std_logic$	        	& FPGA Systemtakt\\
        \lstinline$reset$	& I	    & \lstinline$std_logic$	        	& Asynchrones Reset Signal (aktiv wenn \lstinline$reset='1'$) \\
        \lstinline$i_EBIU$	& I	    & \lstinline$t_EBIU_out$	    	& EBIU Signale vom DSP zum FPGA\\
        \lstinline$o_EBIU$	& O	    & \lstinline$t_EBIU_in$		        & EBIU Signale vom FPGA zum DSP\\
        \lstinline$i_FIFO$	& I	    & \lstinline$t_fifo_ps_Read_out$	& Signale von PSFSM2 Komponente zum Rohdaten FIFO \\
        \lstinline$o_FIFO$	& O	    & \lstinline$t_fifo_ps_Read_out$	& Signale vom Rohdaten FIFO zur PSFSM2 Komponente \\
        \bottomrule
    \end{tabular}
}
\end{table}


\paragraph{Implementierung (Architecture)}

\begin{table}[htbp]
    \ttabbox
    {
        \caption[Carrier NCO interne Signale]{Interne Signale der \lstinline$integrate_and_dump$ Komponente.}
        \label{TabIandD_ArchSignals}
    }
    {
        \rowcolors{2}{light-gray}{White}
    \begin{tabular}{c  p{2cm} p{6cm}}
        \toprule
        Name      		& Typ         & Beschreibung \\
        \midrule
        \lstinline$r$		& \lstinline$t_iandd_reg$ & \\
        \lstinline$r_next$	& \lstinline$t_iandd_reg$ & \\
        \bottomrule
    \end{tabular}
}
\end{table}

\begin{table}[htbp]
    \ttabbox
    {
        \caption[Typdefinition Code NCO Zustandsregister]{Beschreibung der Struktur des Integrate \& Dump Zustandsregisters (\lstinline$t_iandd_reg$ Typ).}
        \label{Tab_t_iandd_reg_Type}
    }
    {
        \rowcolors{2}{light-gray}{White}
    \begin{tabular}{c  p{2cm} p{6cm}}
        \toprule
        Name				& Typ						& Beschreibung \\
        \midrule
        \lstinline$count$		& \lstinline$t_IandD_accu$	&  Akkumulator des Integrators. Abhängig vom Wert an \lstinline$i_data$ wird dieser Wert mit jeder steigenden Taktflanke inkrementiert, bzw. dekrementiert.\\
        \lstinline$output$		& \lstinline$t_IandD_accu$	&  Ausgangsregister. Wenn \lstinline$i_dump='1'$ wird bei einer steigenden Taktflanke der Wert von \lstinline$count$ in dieses Register kopiert. \\
        \bottomrule
    \end{tabular}
}
\end{table}

Die Architecture implementiert den Integrator als einfachen Hoch/Runter Zähler: Bei einer steigenden Taktflanke an \lstinline$i_clk$ wird \lstinline$r.count$ abhängig von \lstinline$i_data$ inkrementiert (\lstinline$i_data='1'$) oder dekrementiert (\lstinline$i_data='0'$). Wenn \lstinline$i_dump='1'$ wird der Zählerstand in das Ausgangsregister \lstinline$r.output$ kopiert und der Zähler zurückgesetzt. Gleichzeitig wird durch das \emph{Dump} Signal bei dem MBlite Prozessor ein Interrupt ausgelöst, der daraufhin den Wert aus dem Ausgangsregister ausliest.


