\section{FPGA Firmware}
In diesem Abschnitt wird die im Rahmen dieser Arbeit entwickelte Implementierung auf FPGA Seite besprochen. Begonnen wird mit einem Überblick und einer Beschreibung der Top-Level Entity. Danach beschäftigt sich der größte Teil mit der Implementierung des Tracking Loop. Der MBlite Prozessor selbst wird hier nicht im Detail besprochen\footnote{Der MBlite Softcore ist in \cite{MBliteThesis} beschrieben.}, aber der Interrupt Controller der entwickelt wurde und die Schnittstelle zwischen MBlite und den Tracking Loop Instanzen.

Im Folgenden wird von UML Klassendiagrammen Gebrauch gemacht um Beziehungen zwischen den Komponenten darzustellen. Deshalb sll kurz erklärt wie diese im Zusammenhang mit der Programmiersprache VHDL zu interpretieren sind: Die Hardwarebeschreibung in VHDL gliedert sich in \emph{Components}. Ein \emph{Component} ist die Instanz einer \emph{Entity} und einer \emph{Architecture}. Eine \emph{Entity} beschreibt die Schnittstelle eines \emph{Component}, und die \emph{Architecture} den inneren Aufbau und Funktion des \emph{Component}. Die \emph{Architecture} Aufbau besteht dabei aus kombinatorischer oder sequentieller Logik, kann aber auch wiederum andere \emph{Components} enthalten.

Typdefinitionen und Componentdeklarationen usw. können zur einfacheren Verwendung in \emph{Packages} zusammengefasst werden. Außerdem können \emph{Entities}, \emph{Architectures} und \emph{Packages} in \emph{Libraries} zusammengefasst werden\footnote{Wenn ein Element nicht aktiv einer Bibliothek hinzugefügt wird, so wird es implizit der \lstinline$work$ Bibliothek hinzugefügt.}. So wurden zum Beispiel die Komponenten des Trackingloop in der \lstinline$gps$ Bibliothek zusammengefasst. Die meisten Teile des MBlite Softcores sind Teil der \lstinline$mblite$ Bibliothek.

In den UML Klassendiagrammen sind die \emph{Entities} als  \emph{Interfaces} modelliert, wobei die VHDL Ports als Attribute mit Sichtbarkeit \emph{public} dargestellt sind. Operationen bietet die Schnittstelle selbst keine. Die \emph{Architectures} sind die Klassen, die eine Interface realisieren. Die in den Architectures enhaltenen Signale sind als Attribute mit Sichtbarkeit \emph{private} modelliert. Wenn eine \emph{Architecture} andere \emph{Components} enthält ist das als UML Komposition modelliert.

 
% Synchronität
\subsection{Schnittstellen}

\subsection{Tracking Loop}
Der Trackingloop ist teils in Hardware und Teils in Software im MBlite Softcore Prozessor implementiert. In diesem Teil wird der Hardware Teil beschrieben. Die Bezihungen der einzelnen Klassen des Tracking Loop sind in \FR{Impl_UML_TL_classdiagram.pdf} dargestellt. Alle dort abgebildeten Komponenten sind in der Bibliothek \lstinline$gps$ zusammengefasst. Nicht dargestellt ist das \emph{Package} \lstinline$tracking_loop_pkg$, welches die Typdefinitionen, einige Konstanten und Component Deklarationen zusammenfasst.

\FGimg[Klassendiagramm des Tracking Loop]{Impl_UML_TL_classdiagram.pdf}{Klassendiagramm des Trackingloop. Die VHDL \emph{Entities} sind als UML \emph{Interfaces} modelliert, und die dazugehörigen VHDL \emph{Architectures} als UML \emph{Klassen}}{0.95\textwidth}

\subsubsection{tracking\_loop}
\FGimg[Entity/Architecture des Carrier NCO]{UML_class_carrier_nco.pdf}{Entity und Architecture welche die Trägerkopie erzeugt.}{10cm}

\FGimg[Entity/Architecture des Code NCO]{UML_class_code_nco.pdf}{Entity und Architecture welche die Takt Steuersignale für den Code Kopie Generator erzeugt.}{7cm}

\FGimg[Entity/Architecture des Code Replika Generators]{UML_class_code_replica_generator.pdf}{Entity und Architecture welche die drei Code Kopien erzeugt.}{9cm}

\FGimg[Entity/Architecture des Integrate and Dump Blocks]{UML_class_IandD.pdf}{Entity und Architecture des Integrate and Dump Blocks.}{6.4cm}

\FGimg[Entity/Architecture des Gold Code LFSR]{UML_class_LFSR.pdf}{Entity und Architecture des \gls{LFSR}, das die Gold Code Sequenz erzeugt.}{10cm}

\FGimg[Entity/Architecture Tracking Loop]{UML_class_tracking_loop.pdf}{Die übergeordnete Entity/Architecture des Tracking Loop}{10cm}

\subsubsection{Gold Code LFSR}
\subsubsection{Code Replika Generator}
\subsubsection{Code NCO}
\subsubsection{Carrier NCO}
\subsubsection{Korrelatoren und Integrate \& Dump}

\subsubsection{FIFOs}

\label{ImplMemoryMapTrackingloop}
\subsubsection{VCO}\label{VCOimplementierung}

\subsection{MBlite Prozessor}
%Instruction und Data Memory. Addressmapping.
\subsection{MBlite Firmware}