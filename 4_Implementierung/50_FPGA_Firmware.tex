\section{FPGA Firmware}
In diesem Abschnitt wird die im Rahmen dieser Arbeit entwickelte Implementierung auf FPGA Seite besprochen. Begonnen wird mit einem Überblick und einer Beschreibung der Top-Level Entity. Danach beschäftigt sich der größte Teil mit der Implementierung des Tracking Loop. Der MBlite Prozessor selbst wird hier nicht im Detail besprochen\footnote{Der MBlite Softcore ist in \cite{MBliteThesis} beschrieben.}, aber der Interrupt Controller der entwickelt wurde und die Schnittstelle zwischen MBlite und den Tracking Loop Instanzen.

Im Folgenden wird von UML Klassendiagrammen Gebrauch gemacht um Beziehungen zwischen den Komponenten darzustellen. Deshalb soll kurz erklärt wie diese im Zusammenhang mit der Programmiersprache VHDL zu interpretieren sind: Die Hardwarebeschreibung in VHDL gliedert sich in \emph{Components}. Ein \emph{Component} ist die Instanz einer Einheit aus \emph{Entity} und  \emph{Architecture}. Eine \emph{Entity} beschreibt die Schnittstelle eines \emph{Component}, und die \emph{Architecture} den inneren Aufbau und Funktion des \emph{Component}. Die \emph{Architecture} Aufbau besteht dabei aus kombinatorischer oder sequentieller Logik, kann aber auch wiederum andere \emph{Components} enthalten.

Typdefinitionen und Componentdeklarationen usw. können zur einfacheren Verwendung in \emph{Packages} zusammengefasst werden. Außerdem können \emph{Entities}, \emph{Architectures} und \emph{Packages} in \emph{Libraries} zusammengefasst werden\footnote{Wenn ein Element nicht aktiv einer Bibliothek hinzugefügt wird, so wird es implizit der \lstinline$work$ Bibliothek hinzugefügt.}. So wurden zum Beispiel die Komponenten des Trackingloop in der \lstinline$gps$ Bibliothek zusammengefasst. Die meisten Teile des MBlite Softcores sind Teil der \lstinline$mblite$ Bibliothek.

In den UML Klassendiagrammen\footnote{In den Diagrammen wird der VHDL \lstinline$std_logic_vector$ Typ abkürzend als \emph{SLV} bezeichnet.} sind die \emph{Entities} als  \emph{Interfaces} modelliert, wobei die VHDL Ports als Attribute mit Sichtbarkeit \emph{public} dargestellt sind. Operationen bietet die Schnittstelle selbst keine. Die \emph{Architectures} sind die Klassen, die eine Interface realisieren. Die in den Architectures enhaltenen Signale sind als Attribute mit Sichtbarkeit \emph{private} modelliert. Wenn eine \emph{Architecture} andere \emph{Components} enthält ist das als UML Komposition modelliert.

\subsection{Top Level Entity}\label{Section_gps_top_entity}
Die Top Level Entity ist, wie der Name sagt, die übergeordnete Komponente die zum Einen Schnittstellen zu DSP, MAX2769, und Clock bereitstellt und zum Anderen in der dazugehörigen Architecture die anderen Komponenten enthält und verbindet.

In \FR{UML_comp_structure_gps_top.pdf} ist der Aufbau der \lstinline$gps_top$ Komponente etwas vereinfacht dargestellt. Nicht dargestellt ist die in Abschnitt \ref{RohdatenFIFOSchnittstelle} beschriebene Rohdaten Schnittstelle zum DSP, die zu Anfang der Entwicklungsphase genutzt wurde um Rohdaten auf den DSP zu übertragen. Diese wird in einem eigenen Abschnitt beschrieben, da sie zu diesem Zeitpunkt aus den in Abschnitt \ref{UserIOSchnittstelle} genannten Gründen noch nicht mit dem derzeitigen Entwicklungsstand zusammengeführt wurde. Aus Gründen der Übersicht ist in \FR{UML_comp_structure_gps_top.pdf} der Clock Synthesizer nicht abgebildet. Der Clock Synthesizer frischt das vom GPS Frontend kommende Taktsignal auf und versorgt mit dem aufgefrischten Signal die anderen Komponenten.

\FGimg[Übersicht FPGA Firmware]{UML_comp_structure_gps_top.pdf}{Vereinfachte Struktur der Top Level Komponente \lstinline$gps_top$ und einige der verwendeten Typen.}{0.95\imgmaxwidth}

Wie bereits in Abschnitt \ref{SoftcoreAuswahl} erwähnt ist der  MBlite Prozessor eine Implementierung des Microblaze und hat somit eine \emph{Harvard Architektur} bei der Data und Instruction Memory über zwei getrennte Busse angesprochen werden.

Für beide Speicher wird in dieser Arbeit der interne Dual Port Block RAM des FPGA genutzt. Dabei gibt es zwei Möglichkeiten die Speicher zu implementieren: Es können entweder zwei getrennte Block RAM Instanzen als Single Port RAM benutzt werden. Oder es wird die Dual Port Funktionalität genutzt und beide Speicher liegen in  einer gemeinsamen Block RAM Instanz. Von der Funktion unterscheiden sich beide Varianten nicht, aber erstere Variante vereinfacht das Erstellen der Hexfiles etwas, weshalb hier zwei Block RAM Instanzen verwendet werden (siehe dazu auch Abschnitt \ref{MBliteFirmware}).

Der MBlite Prozessorkern ist direkt über den ersten Bus mit dem Instruction Memory verbunden. Der zweite Bus geht an einen Address Decoder. Dieser wertet die Adresse aus und entscheidet anhand einer Memory Map (festgelegt in \lstinline$config_Pkg.vhd$) welche der an ihn angeschlossenen Komponenten Zugriff auf den Datenbus erhält.

Die an den Address Decoder angeschlossenen Komponenten sind der Data Memory, die sechs Tracking Loop Kanäle und der Interrupt Controller. Diese Komponenten und der Clock Synthesizer werden in den weiteren Abschnitten einzeln beschrieben.

% Synchronität
\subsection{Schnittstellen}

\subsection{Tracking Loop}
Der Trackingloop ist teils in Hardware und Teils in Software im MBlite Softcore Prozessor implementiert. In diesem Teil wird der Hardware Teil beschrieben. Die Bezihungen der einzelnen Klassen des Tracking Loop sind in \FR{Impl_UML_TL_classdiagram.pdf} dargestellt. Alle dort abgebildeten Komponenten sind in der Bibliothek \lstinline$gps$ zusammengefasst. Nicht dargestellt ist das \emph{Package} \lstinline$tracking_loop_pkg$, welches die Typdefinitionen, einige Konstanten und Component Deklarationen zusammenfasst.

\FGimg[Klassendiagramm des Tracking Loop]{Impl_UML_TL_classdiagram.pdf}{Klassendiagramm des Trackingloop. Die VHDL \emph{Entities} sind als UML \emph{Interfaces} modelliert, und die dazugehörigen VHDL \emph{Architectures} als UML \emph{Klassen}}{0.95\textwidth}

\subsubsection{tracking\_loop}
\FGimg[Entity/Architecture des Carrier NCO]{UML_class_carrier_nco.pdf}{Entity und Architecture welche die Trägerkopie erzeugt.}{10cm}

\FGimg[Entity/Architecture des Code NCO]{UML_class_code_nco.pdf}{Entity und Architecture welche die Takt Steuersignale für den Code Kopie Generator erzeugt.}{7cm}

\FGimg[Entity/Architecture des Code Replika Generators]{UML_class_code_replica_generator.pdf}{Entity und Architecture welche die drei Code Kopien erzeugt.}{9cm}

\FGimg[Entity/Architecture des Integrate and Dump Blocks]{UML_class_IandD.pdf}{Entity und Architecture des Integrate and Dump Blocks.}{6.4cm}

\FGimg[Entity/Architecture des Gold Code LFSR]{UML_class_LFSR.pdf}{Entity und Architecture des \gls{LFSR}, das die Gold Code Sequenz erzeugt.}{10cm}

\FGimg[Entity/Architecture Tracking Loop]{UML_class_tracking_loop.pdf}{Die übergeordnete Entity/Architecture des Tracking Loop}{10cm}

\subsubsection{Gold Code LFSR}
\subsubsection{Code Replika Generator}
\subsubsection{Code NCO}
\subsubsection{Carrier NCO}
\subsubsection{Korrelatoren und Integrate \& Dump}

\subsubsection{FIFOs}



%Instruction und Data Memory. Addressmapping.

\subsection{Interrupt Controller}
\label{InterruptController}
Die sechs Tracking Loop Kanäle können unabhängig voneinander Interrupt Signale liefern. Der MBlite Prozessor hat allerdings nur einen Interrupt Eingang. Daher wurde ein Interrupt Controller entwickelt, der eine beliebige Anzahl an Interrupts entgegennehmen kann und an den MBlite Prozessor weiterleitet. Das besondere an dem hier entwickelten Interrupt Controller ist, dass die Signale in einem FIFO gespeichert werden und so sichergestellt wird, dass die Interrupts in der Eingangsreihenfolge der Ereignisse abgearbeitet werden. Dies ist wichtig um die Echtzeitanforderungen einzuhalten. Die Komponente ist in \lstinline$interrupt_ctrl.vhd$ implementiert. 

\FGimg[Entity/Architecture des Interrupt Controllers]{UML_class_IRQctrl.pdf}{Entity und Architecture des Interrupt Controllers.}{8cm}

\paragraph{Schnittstelle (Entity)}
In \TR{TabIRQCtrl_Entity} ist die Schnittstelle der \lstinline$interrupt_ctrl$ Komponente beschrieben.

Die Anzahl der Interrupt Quellen und die maximal mögliche Länge der Warteschlange kann über die zwei \emph{Generics} \lstinline$n_intr$ bzw. \lstinline$max_pending$ konfiguriert werden. Das Interrupt Signal einer Quelle wird mit einem Element des Vektors \lstinline$i_intr$ verbunden. Der Index des Bits in dem Vektor ist gleichzeitig die Interruptnummer, d.h. dass eine Quelle die mit \lstinline$i_intr(3)$ verbunden ist die Nummer 3 zugewiesen bekommt.

Der Signal \lstinline$o_intr$ wird mit dem Interrupt Eingang des MBlite Prozessors verbunden. 

Ein Interrupt muss von den Quellen durch einen  \lstinline$'1'$ Impuls signalisiert werden, das heißt, dass die Leitung nur für eine Taktperiode auf  \lstinline$'1'$ gehalten werden soll und danach wieder auf  \lstinline$'0'$ gesetzt wird. Der Interrupt Controller wird daraufhin die Nummer des Interrupts in der FIFO Warteschlange speichern. Je Taktperiode kann ein Ereignis in die Warteschlange eingefügt werden. Bei mehreren gleichzeitig auftretenden Interrupt Ereignissen gehen die weiteren Ereignisse jedoch \emph{nicht} verloren, sondern werden während der darauf folgenden Taktperioden eingefügt. So lange sich Interrupts in der Warteschlange befinden, ist \lstinline$o_intr='1'$. 

In der \gls{ISR} muss die CPU dann zuerst die Quelle des Interrupts abfragen. Dazu ist der Ausgang des FIFO über den Address Decoder an den Datenbus der CPU angeschlossen. Die Adresse unter der der Interrupt Controller zu finden ist wird über die Memory Map in \lstinline$config_Pkg.vhd$ festgelegt. Lesen von dieser Adresse liefert die Nummer der Interrupt Quelle. Anschließend muss die CPU den Interrupt bestätigen um ihn aus der Warteschlange zu löschen. Dazu wird auf die selbe Adresse ein Schreibvorgang ausgeführt. Dabei ist es unwichtig welcher Wert geschrieben wird, intern wertet die Interrupt Controller Komponente nur das \emph{Write Enable} Signal aus (\lstinline$i_dmem.we_o$).

\subparagraph{Einschränkungen} Es gibt einige Einschränkungen die bei der Nutzung des Interrupt Controllers zu beachten sind:
\begin{enumerate}
    \item Im Falle von Interrupt Ereignissen die in der selben Taktperiode auftreten, werden die Ereignisse in den darauf folgenden Taktperioden nach ihrer Interrupt Nummer, in aufsteigender Reihenfolge in die Warteschlange eingefügt.
    \item Wenn noch \emph{nicht} alle Ereignisse in die Warteschlange eingefügt wurden (siehe Punkt 1) und neue Ereignisse auftreten, so werden die neuen und alten Ereignisse in aufsteigender Reihenfolge in die Warteschlange eingefügt. Dadurch kann eventuell ein später aufgetretenes Ereignis mit niedriger Interruptnummer vor einem früheren Ereignis mit höherer Nummer in die Warteschlange eingefügt werden.
    \item Bevor ein erneuter Interrupt von der selben Quelle verarbeitet werden kann, muss der frühere Interrupt in die Warteschlange eingefügt worden sein.
\end{enumerate}

In der Praxis sind diese Einschränkungen jedoch kein Problem, da sie lediglich zum tragen kommen wenn die Ereignisse in sehr kurzer Zeitabfolge (wenige Taktperioden) aufeinander folgen. Eine Änderung der Bearbeitungsreihenfolge ist in diesen Fällen nicht gravierend.

\begin{table}[htbp]
    \ttabbox
    {
        \caption[Code Replika Generator Schnittstelle]{Schnittstellenbeschreibung (Entity) der \lstinline$interrupt_ctrl$ Komponente.}
        \label{TabIRQCtrl_Entity}
    }
    {
        \rowcolors{2}{light-gray}{White}
    \begin{tabular}{p{1.5cm} c  p{2cm} p{5.5cm}}
        \toprule
        Name                        & I/O       & Typ                       & Beschreibung \\
        \midrule
        \lstinline$n_intr$          & I         & \lstinline$positive$      & \lstinline$generic$, dass die Anzahl der Interruptquellen festlegt.\\
        \lstinline$max _pending$     & I         & \lstinline$positive$      & \lstinline$generic$, dass die maximale Anzahl von unbearbeiten Interrupts in der Warteschlange festlegt.\\
        \lstinline$i_clk$           & I         & \lstinline$std_logic$     & Taktsignal das auch den MBlite Core antreibt.\\
        \lstinline$i_reset$         & I         & \lstinline$std_logic$     & Asynchrones Reset Signal.\\
        \lstinline$i_intr$          & I         & \lstinline$SLV(n_intr -1 DOWNTO 0)$    & Ein Vektor mit den Interruptsignalen der Interruptquellen. Eine \lstinline$'1'$ signalisiert einen Interrupt. Die Leitung sollte nur für eine Taktperiode auf \lstinline$'1'$ gehalten werden.\\
        \lstinline$i_dmem$          & I         & \lstinline$dmem_out _type$            & Schnittstelle zum MBlite Prozessor. Steuerleitungen und Daten- und Adressbus.\\
        \lstinline$o_dmem$          & O	        & \lstinline$dmem_in _type$	& Datenbus zum MBlite Prozessor. \\
        \lstinline$o_intr$          & O         & \lstinline$std_logic$     & Interruptleitung zum MBlite Prozessor. \lstinline$'1'$ solange Interrupts in der Warteschlange stehen.\\
        \bottomrule
    \end{tabular}
}
\end{table}



\subsection{MBlite Firmware}
\label{MBliteFirmware}
Ein im MBlite Prozessor laufendes Programm realisiert die \emph{TrackingSW} Komponente (\FR{GPS_UML_deployment1.png}). Diese realisiert die in \FR{Trackingloop.png} grün markierten Funktionsblöcke, also die Carrier und Code Loop Diskriminatoren und Filter. Die zeitkritische Neuberechnung der Werte ist vollständig Interrupt gesteuert, wodurch im Hauptprozess andere Aufgaben bearbeitet werden können, die weniger zeitkritisch sind. So würde beispielsweise die Initialisierung eines Tracking Loop Kanals oder die Kommunikation mit dem DSP in der \lstinline[language=C]$main()$ Funktion ablaufen\footnote{Die Kommunikation mit dem DSP ist derzeit noch nicht implementiert.}. Wenn die Loop Parameter eines Kanals neu berechnet werden müssen, wird durch den Interrupt die Ausführung der \lstinline[language=C]$main()$ Funktion unterbrochen, die Loop Parameter neuberechnet, und anschließend die \lstinline[language=C]{main()} Funktion weiter ausgeführt.



\FGimg[MBlite Firmware Übersicht]{UML_class_MBliteFirmware.pdf}{Klassendiagramm MBlite Firmware}{0.9\textwidth}


\begin{table}[htbp]
    \ttabbox
    {
        \caption[Typdefinition Tracking Kanal in MBlite Firmware ]{Der \lstinline$t_channel$ Typ fasst alle Attribute eines Trackingloop Kanals zusammen. Der Typ ist in \lstinline$trackingloop.h$ definiert.}
        \label{Tab_t_channel}
    }
    {
        \rowcolors{2}{light-gray}{White}
    \begin{tabular}{c  p{2cm} p{6cm}}
        \toprule
        Name				            & Typ                       & Beschreibung \\
        \midrule
        \lstinline$hw_regs$             & \lstinline$t_hw_regs*$	& Zeiger auf eine Struktur mit der die Register der \emph{TrackingHW} Komponente angesprochen werden können.\\
        \lstinline$prev_prompt_phi$		& \lstinline$int32_t$       & $\phi_f \cdot z^{-1}$ Carrier Loop Diskriminator Wert aus dem vorherigen Zeitschritt.\\
        \lstinline$prev_carrier_fcw$    & \lstinline$int32_t$       & $\omega_f \cdot z^{-1}$ Carrier Loop \gls{FCW} aus dem vorherigen Zeitschritt.\\
        \lstinline$prev_dll_err$		& \lstinline$int32_t$       & $\phi_g \cdot z^{-1}$ Code Loop Diskriminator Wert aus dem vorherigen Zeitschritt.\\
        \lstinline$prev_code_fcw$		& \lstinline$int32_t$       & $\omega_g \cdot z^{-1}$ Code Loop \gls{FCW} aus dem vorherigen Zeitschritt.\\
        \bottomrule
    \end{tabular}
}
\end{table}


\begin{table}[htbp]
    \ttabbox
    {
        \caption[Typdefinition Hardwareregister in MBlite Firmware ]{Der \lstinline$t_hw_regs$ Typ spiegelt das Memory Layout der Register der \emph{TrackingHW} Komponente wieder (siehe auch \FR{TLMemoryLayout.pdf} und \TR{Tab_tracking_loop_memory_map}). Der Typ ist in \lstinline$trackingloop.h$ definiert.}
        \label{Tab_t_hw_regs}
    }
    {
        \rowcolors{2}{light-gray}{White}
    \begin{tabular}{c  p{2cm} p{6cm}}
        \toprule
        Name				            & Typ                       & Beschreibung \\
        \midrule
        \lstinline$Early$       & \lstinline$t_iq$      & Struktur mit den Registern der Early I\&Q Integratoren.\\
        \lstinline$Prompt$      & \lstinline$t_iq$      & Struktur mit den Registern der Prompt I\&Q Integratoren.\\
        \lstinline$Late$        & \lstinline$t_iq$      & Struktur mit den Registern der Late I\&Q Integratoren.\\
        \lstinline$code_fcw$    & \lstinline$int32_t$   & Code \gls{FCW} Register.\\
        \lstinline$carrier_fcw$	& \lstinline$int32_t$   & Carrier \gls{FCW} Register.\\
        \lstinline$PRN$		    & \lstinline$int32_t$   & \gls{PRN} Register. Legt den zu trackenden Satelliten fest.\\
        \lstinline$codephase$   & \lstinline$int32_t$   & Initiale Codephase die bei der Acquisition bestimmt wurde.\\
        \lstinline$flags$       & \lstinline$int32_t$   & Steuersignale die den \emph{TrackingHW} Kanal kontrollieren. Siehe auch \TR{Tab_TL_flags}.\\
        \bottomrule
    \end{tabular}
}
\end{table}

\begin{table}[htbp]
    \ttabbox
    {
        \caption[Typdefinition IQ Korrelatoren in MBlite Firmware ]{Der \lstinline$t_iq$ Typ fasst I und Q Integrate \& Dump Register des Early, Prompt oder Late Zweigs zusammen (siehe auch \FR{TLMemoryLayout.pdf} und \TR{Tab_tracking_loop_memory_map}). Der Typ ist in \lstinline$trackingloop.h$ definiert.}
        \label{Tab_t_iq}
    }
    {
        \rowcolors{2}{light-gray}{White}
    \begin{tabular}{c  c p{5cm}}
        \toprule
        Name            & Typ                       & Beschreibung \\
        \midrule
        \lstinline$I$   & \lstinline$int16_t const$ & Wert des Integrators im I Zweig. Read-only.\\
        \lstinline$Q$   & \lstinline$int16_t const$ & Wert des Integrators im Q Zweig. Read-only.\\
        \bottomrule
    \end{tabular}
}
\end{table}

\lstinline[language=C]$void cart2polar(t_iq* data, int16_t* kr, int16_t* phi)$
Berechnet Betrag $r$ und/oder Phase $\phi$ der komplexen Zahl, die durch \lstinline$data$ repräsentiert wird, wobei \lstinline$data.I$ der Realteil, und \lstinline$data.Q$ der Imaginärteil ist. Der Betrag wird in \lstinline$kr$ im Festkommaformat $Q15.0$ gespeichert, und die Phase in \lstinline$phi$ im Festkommaformat $Q1.14$. Wenn einer der Werte nicht benötigt wird, kann ein \lstinline$NULL$ Pointer übergeben werden. Die Werte werden mit dem \emph{CORDIC} Algorithmus im Vektormodus berechnet. Wichtig zu beachten ist, dass \lstinline$kr$ der mit dem Faktor $k \approx 1.644$ skalierte Betrag $r$ ist. Je nach Anwendung muss dies berücksichtigt werden. Der Skalierungsfaktor kommt durch den \emph{CORDIC} Algorithmus zustande.

Anschaulich betrachtet funktioniert der \emph{CORDIC} Algorithmus indem die komplexe Zahl als Zeiger betrachtet wird, den es gilt so zu drehen, dass er auf der $x$-Achse liegt. Dies geschieht mit inkrementellen Schritten, wobei die Drehung durch Skalieren der $x$ und $y$ Komponenten des Zeigers realisiert wird. Die Drehwinkel sind dabei so gewählt, dass der Skalierungsfaktor eine Potenz von 2 ist\footnote{Zum besseren Verständnis sei auch auf den Quellcode verwiesen. Außerdem wurde in \lstinline$doc/CORDIC.ggb$ ein interaktives \emph{GeoGebra} Applet erstellt, mit dem die Funktion des \emph{CORDIC} Algorithmus graphisch dargestellt ist.}. Damit kann das Skalieren durch einfache Schiebe- und Additionsoperationen und völlig ohne Multiplikationen oder Divisionen realisiert werden. Dadurch kann der \emph{CORDIC} Algorithmus sehr schnell und mit einer vorab bekannten Anzahl an Schritten, Betrag und Phase berechnen.

\lstinline[language=C]$int16_t dll_discriminator(t_channel* ch_x)$ 
Berechnet den Code Diskriminator Wert für Kanal \lstinline[language=C]$ch_x$. Der Rückgabewert ist eine Festkommazahl im Format $Q1.14$. Die Berechnung erfolgt nach der Formel \ref{EqnCodeDiscr}. Zur Berechnung der Beträge ($\sqrt{\tilde{y}_{I}^2+\tilde{y}_{Q}^2}$) wird der \emph{CORDIC} Algorithmus verwendet der in \lstinline[language=C]$cart2polar()$ implementiert ist. Wie bei der Beschreibung der \lstinline$cart2polar()$ Funktion erwähnt, gibt die Funktion den mit dem Faktor $k \approx 1.644$ skalierten Betrag ($k\cdot \sqrt{\tilde{y}_{I}^2+\tilde{y}_{Q}^2}$) zurück. Dies hat jedoch auf das Ergebnis keinen Einfluss, da wenn man Formel \ref{EqnCodeDiscr} betrachtet der Faktor $k$ in Zähler und Nenner vorkommt und sich somit heraus kürzt.


\lstinline[language=C]$void pll(t_channel* ch_x)$
Berechnet das \gls{FCW} für den Code Loop \gls{NCO} von Kanal \lstinline$ch_x$ neu. Dazu wird zuerst die Funktion \lstinline$cart2polar()$ aufgerufen, um den Carrier Loop Discriminator Wert für Kanal \lstinline$ch_x$ zu bestimmen. Der Rest der \lstinline$pll()$ Funktion implementiert den in Abschnitt \ref{FixedPointCarrierLoopFilterDesign} entworfenen Carrier Loop Filter.


\lstinline[language=C]$void dll(t_channel* ch_x)$
Berechnet das \gls{FCW} für den Code Loop \gls{NCO} von Kanal \lstinline$ch_x$ neu. Dazu wird zuerst die Funktion \lstinline$dll_discriminator()$ aufgerufen, um den Code Loop Discriminator Wert für Kanal \lstinline$ch_x$ zu bestimmen. Der Rest der \lstinline$dll()$ Funktion implementiert den in Abschnitt \ref{FixedPointCodeLoopFilterDesign} entworfenen Code Loop Filter.

\lstinline[language=C]$void isr()$
Diese Funktion ist mit dem speziellen Attribut \lstinline[language=C]$__attribute__ ((interrupt_handler))$ versehen und ist wird damit vom Compiler als Interrupt Service Routine verwendet. Diese führt der Prozessor bei einem Hardware Interrupt aus. Die Funktion an sich ist sehr einfach:

\begin{lstlisting}[language=C]
void isr(){
	int irq_source = *((uint32_t*)0x8000);
	*((uint32_t*)0x8000) = 0x00; // Acknowledge Interrupt

	if ((irq_source>=0) && (irq_source<=(NUM_CHANNELS-1)))
	{
		dll(&ch[irq_source]);
		pll(&ch[irq_source]);
	}
}
\end{lstlisting}

Zuerst wird in Zeile 2 die genaue Quelle des Interrupts aus dem FIFO des Interrupt Controllers gelesen. \lstinline[language=C]$0x8000$ ist dabei die Adresse die für den Interrupt Controller in der Memory Map des Address Decoders in \lstinline$config_Pkg.vhd$ festgelegt wurde. In Zeile 3 wird dann der Interrupt bestätigt, indem ein Schreibvorgang auf die selbe Adresse ausgeführt wird. Welcher Wert geschrieben wird ist dabei nicht wichtig (siehe auch die Beschreibung des Interrupt Controllers in Abschnitt \ref{InterruptController}).

In den weiteren Zeilen wird dann überprüft, ob es sich um eine zulässige Interrupt Nummer handelt, und dann werden die Funktionen \lstinline[language=C]$dll()$ und \lstinline[language=C]$pll()$ für den Kanal, der den Interrupt ausgelöst hat, aufgerufen.

\lstinline[language=C]$int main()$
In der \lstinline[language=C]$main()$ Funktion werden die Tracking Loop Kanäle konfiguriert. Dazu muss die in Abschnitt \ref{TLStartProzedur} beschriebene Startprozedur eingehalten werden. Außerdem ist vorgesehen, aber bisher noch nicht implementiert, dass in der \lstinline[language=C]$main()$ Funktion die Kommunikation mit dem DSP abläuft. 


\subsubsection{Kompilieren der Firmware und erzeugen der Speicherabbilder}
% Adressoffset? Makefile?
Im Unterschied zu einem Anwendungsprogramm für einen \enquote{normalen} Mikrocontroller oder einem Programm für einen PC sind beim Kompilieren und Erzeugen des ausführbaren Programms für den MBlite Softcore einige Besonderheiten zu beachten.

Wie bereits erwähnt ist der MBlite ein Prozessor in Harvard Architektur, wo  Daten und Instruktionen in getrennten Speichern vorliegen. Die Speicher sind als VHDL Arrays realisiert welche mit den Daten initialisiert werden\footnote{Genauer gesagt wird der Instruction Memory mit dem Programmcode und der Data Memory mit den Variablen und Konstanten initialisiert.}. Bei der Synthese der FPGA Firmware instantiieren die VHDL Arrays dann FPGA-internen Block RAM.

Die Schritte zum Erstellen des VHDL Quellcodes von Data- und Instruction Memory sind:
\begin{enumerate}
    \item Kompilieren des Quellcodes zu \emph{Object Files}.
    \item Linken der \emph{Object Files} zu einem \gls{ELF} Binary.
    \item Erstellen eines Speicherauszugs (\emph{Memory Dump}) des \gls{ELF} Binary.
    \item Auftrennen des Memory Dump in Instruction- und Data Memory.
    \item Konvertieren der Memory Dumps in VHDL Arrays und verpacken des Arrays in eine VHDL Architecture und Entity.
\end{enumerate}

All diese Schritte sind in einem \lstinline$Makefile$ zusammengefasst\footnote{Das Makefile für die MBlite Firmware ist in \lstinline$gps_fpga/firmware/$ zu finden.}. Im Folgenden werden die wichtigsten Punkte des Makefile erklärt, wie z.B. besondere Compileroptionen und Kommandos die verwendet werden. Für Grundlegende Informationen zum Kompilieren mit \emph{gcc} und dem Build-Management mit \emph{gnu make} sei hier auf die Dokumentation von \emph{gcc} und \emph{gnu make} verwiesen.

\paragraph{Schritt 1: Kompilieren des Quellcodes}
Weil der MBlite Prozessor über optionale Funktionen  (z.B. Hardware Multiplizierer oder Barrel Shifter) verfügt, muss dem Compiler für Schritt 1 mitgeteilt werden welche Instruktionen in Hardware realisiert werden können und welche mit anderen Instruktionen emuliert werden müssen. Die entsprechenden Optionen sind in dem \lstinline$Makefile$ in der Variable \lstinline$XILFLAGS$ zusammengefasst.

\paragraph{Schritt 2: Linken der Object Files}
In Schritt 2 wird das Memory Layout des Programms festgelegt, wie z.B. bei welcher Adresse die Datensektionen beginnen oder wie groß Stack und Heap sind. Die erste Sektion des Data Memories ist die \lstinline$.ctors$ Sektion. Die Startadresse dieser Sektion wurde auf $0x10000000$ (Variable \lstinline$DMEM_OFFSET$) festgelegt. Dies bedeutet nicht das das Binary \SI{>268}{\mega\byte} groß ist: Die Adressen sind \emph{virtuelle Adressen}. Der Adressdekoder sorgt dafür, dass Zugriffe auf diese virtuellen Adressen dem richtigen Speicher bzw. Peripheriegerät zugeordnet werden\footnote{Die Memory Map ist in der Datei \lstinline$gps_fpga/rtl/vhdl/config_Pkg.vhd$ festgelegt.}.

In dem Programm wird lediglich wenig Stack- bzw. Heap Speicher benötigt. Deshalb wird über Linker Optionen die Heap- und Stack Größe gegenüber dem Standardwert auf je \SI{4}{\kilo\byte} verkleinert. 

\paragraph{Schritt 3\&4:}
Der Compiler erzeugt ein \gls{ELF} Binary welches verschiedene Sektionen enthält, die grob in ausführbare und nicht-ausführbare Sektionen unterteilt werden können. Eine Tabelle mit den Sektionsheadern des \gls{ELF} Binary kann beispielsweise mit dem Kommando\footnote{\label{MBBinutils}Die Werkzeuge sind Teil der \emph{GNU Binutils} für den MBlite Prozessor. Gegebenenfalls muss noch das Präfix \enquote{mb-} angefügt werden, z.B. \lstinline$mb-readelf$} \lstinline$readelf -S ProgrammName$ ausgegeben werden. Ausführbare Sektionen sind in der \emph{Flag} Spalte mit einem \emph{X} gekennzeichnet. Dies sind die Sektionen die im Instruction Memory enthalten sind. Die nicht-ausführbaren Sektionen sind im Data Memory enthalten.

In dem Makefile sind die ausführbaren Sektionen in der Variable \lstinline$EXECUTABLESECTIONS$ aufgelistet. Diese Liste wurde anhand der Ausgabe von \lstinline$readelf$ erstellt.

Mit dem Kommando\footref{MBBinutils} \lstinline$objdump$ werden dann aus dem \gls{ELF} File zwei Speicherabbilder erzeugt: Das erste Abbild (\lstinline$imem.bin$) enthält nur die ausführbaren Sektionen. Das zweite Abbild (\lstinline$dmem4.bin$) enthält alle nicht-ausführbaren Sektionen.

\paragraph{Schritt 5:}
Die Programme\footnote{Die Programme kommen als Teil des MBlite Quellcodes und sind unter \lstinline$gps_fpga/firmware/util$ zu finden.} \lstinline$bin2vhd_4x8$ bzw. \lstinline$bin2vhd_32b$ konvertieren die \lstinline$.bin$ Memory Dumps in VHDL Komponenten die den Speicherinhalt als VHDL Array enthalten. 
Der Grund weshalb zwei verschiedene Programme verwendet werden liegt in den unterschiedlichen Anforderungen für Instruction und Data Memory: Während für den Instruction Memory immer nur auf ganze \SI{32}{\bit} Worte zugegriffen werden muss, soll bei dem Data Memory auch auf \SI{16}{\bit}, \SI{8}{\bit} Worte zugegriffen werden können. 

Als zweiten Parameter erwarten die beiden Programme die Größe des Speichers. Die minimal notwendige Größe lässt sich mit dem Kommando\footref{MBBinutils} \lstinline$size$ bestimmen. Der Instruction Memory muss mindestest so groß sein wie die Sektion \lstinline$.text$. Der Data Memory muss mindestens so groß sein wie die Sektionen \lstinline$.data$ und \lstinline$.bss$ zusammen. Zu beachten ist herbei, dass \lstinline$size$ die Größe in \emph{Bytes} ausgibt, \lstinline$bin2vhd_*$ aber die Anzahl der 32 Bit Speicherworte als Potenz von 2 erwartet. Wenn also als Parameter eine \lstinline$10$ angegeben wird, so wird ein Speicher erzeugt, der $2^{10}\times \SI{4}{\byte}=\SI{4}{\kilo\byte}$ umfasst.

%Eine ELF-Datei kann aus bis zu fünf Teilen bestehen:

%Kopfinformationen (ELF header)
%Programmkopf-Tabelle (program header table)
%Sektionskopf-Tabelle (section header table)
%die Sektionen (ELF sections)
%die Segmente (ELF segment)