%\newglossaryentry{⟨label⟩}{%
%name={⟨name⟩}, 
%description={⟨description⟩}, 
%⟨other options⟩
%}

%\newacronym{⟨label⟩}{⟨short⟩}{⟨long⟩}


\newacronym{DSP}{DSP}{Digitaler Signal Prozessor}
\newacronym{FPGA}{FPGA}{Field Programmable Gate Array}


\newglossaryentry{Softcore}{%
name={Softcore},
description={Vorgefertigter, wiederverwendbarer Funktionsblock der als Quellcode vorliegt und für ein \gls{FPGA} synthetisiert werden kann}
}

\newacronym{EBIU}{EBIU}{\gls{l_EBIU}}
\newglossaryentry{l_EBIU}{
name={External Bus Interface Unit},
description={Ein Speicherinterface des Blackfing \gls{DSP}}
}

\newglossaryentry{l_NCO}{%
name={Numerically Controlled Oscillator}, 
description={Ein digital steuerbarer Funktionsblock der ein digitales Signal mit einstellbarer Frequenz erzeugt. Er wird im \gls{FPGA} zur zur Realisierung des \gls{LO} genutzt}
}
\newacronym{NCO}{NCO}{\gls{l_NCO}}


\newglossaryentry{Chip}{%
name={Chip}, 
description={Bezeichnung für ein Bit des CA Codewortes, um es von den Bits des Informationswortes zu unterscheiden}
}

\newglossaryentry{Codephase}{%
name={Codephase}, 
description={Die zeitliche Differenz zwischen dem Empfangszeitpunkt, und dem nächsten Beginn der Codesequenz. Für die C/A Codes ist der Beginn der Codesequenz der Zeitpunkt an dem die erzeugenden LFSR den Zustand „1111111111“ haben}
}

\newglossaryentry{Firmware}{%
name={Firmware}, 
description={Im Kontext dieser Arbeit die auf dem Softcore Prozessor im FPGA laufende Software}
}

\newacronym{LFSR}{LFSR}{\gls{l_LFSR}}
\newglossaryentry{l_LFSR}{%
name={Linear Feedback Shift Register}, 
description={Linear rückgekoppeltes Schieberegister. Eine Methode zur Erzeugung von pseudozufälligen Bitsequenzen}
}

\newacronym{SV}{SV}{\gls{l_SV}}
\newglossaryentry{l_SV}{%
name={Space Vehicle}, 
description={Im GPS Kontext Bezeichnung für einen GPS Satellit}
}

\newacronym{PRN}{PRN}{\gls{l_PRN}}
\newglossaryentry{l_PRN}{%
name={Pseudo Random Number}, 
description={Im GPS Kontext eine Nummer die eine bestimmte \gls{CA} bzw. \gls{PY} Sequenz identifiziert. Jedem aktiven GPS \gls{SV} ist eine eindeutige PRN zwischen 1 und 32 zugewiesen}
}

\newglossaryentry{l_Acq}{%
name={Acquisiton}, 
description={Die Acquisition bestimmt die Werte der unbekannten Variablen die zur Demodulation benötigt werden}
}

\newacronym{IF}{IF}{\gls{ZF}}
\newglossaryentry{ZF}{%
name={Zwischenfrequenz}, 
description={Die Differenz aus LO Frequenz und der Mittenfrequenz des empfangenen Signals}
}

\newacronym{LEO}{LEO}{\gls{l_LEO}}
\newglossaryentry{l_LEO}{%
name={Low Earth Orbit}, 
description={Orbit mit Höhe \SI{200}{\km} bis \SI{2000}{\km}. Der für Cubesat Missionen am häufigsten gebrauchte Orbit}
}

\newglossaryentry{LO}{%
name={Lokaloszillator}, 
description={Der Oszillator der auf der Empfängerseite eine Kopie des Trägersignals erzeugt}
}

\newacronym{SNR}{SNR}{Signal to Noise Ratio}

\newglossaryentry{CA}{%
name={C/A}, 
description={Ein Code dessen Codewörter die Eigenschaft haben Orthogonal zueinander zu sein (d.h. geringe Kreuzkorrelation der Codewörter). Der C/A Code verwendet sogenannte \emph{Gold Folgen} die durch \gls{LFSR} erzeugt werden. C/A steht hierbei für \emph{Coarse/Acquisition}. Der C/A Code wird zur Spreizung des zivil nutzbaren GPS Signals benutzt. (TODO Referenz GPS Spezifikation)}
}

\newglossaryentry{PY}{%
name={P(Y)}, 
description={Ein Code mit orthogonalen Codeworten, ähnlich wie der \gls{CA}. Jedoch sind die Codeworte im Unterschied zum \gls{CA} deutlich länger. Dadurch ist er weniger anfällig für Interferenzen, und das Codewort lässt sich nicht erraten. Er wird für Anwendungen die hohe Genauigkeit oder geringe Störanfälligkeit erfordern (Landvermessung, Militär) benutzt}
}

\newacronym{MAC}{MAC}{\gls{l_MAC}}
\newglossaryentry{l_MAC}{%
name={Medium Access Control}, 
description={Verfahren die die gemeinsame Nutzung des Übertragungsmediums regeln}
}


\newacronym{GPS}{GPS}{\gls{l_GPS}}
\newglossaryentry{l_GPS}{%
name={Global Positioning System}, 
description={Satellitengestütztes Navigationssystem zur Positionsbestimmung, entwickelt und betrieben durch das US Air Force Space Command}
}

\newacronym{GNSS}{GNSS}{\gls{l_GNSS}}
\newglossaryentry{l_GNSS}{%
name={Global Navigation Satellite System}, 
description={Allgemeine Bezeichnung für Satellitengestützte Navigationssysteme. Beispiele sind \gls{GPS}, Galileo und GLONASS}
}

\newacronym{SDR}{SDR}{\gls{l_SDR}}
\newglossaryentry{l_SDR}{%
name={Software Defined Radio}, 
description={Funkübertragungssysteme bei denen wichtige Teile des Systems wie (De)Modulation, Mischer, Oszillatoren und Filter in Software realisiert werden}
}

\newacronym{BPSK}{BPSK}{\gls{l_BPSK}}
\newglossaryentry{l_BPSK}{%
name={Binary Phase Shift Keying}, 
description={Digitales Modulationsverfahren bei dem '1' und '0' durch die jeweils \SI{180}{\degree} unterschiedliche Phasenlage des Trägers symbolisiert werden}
}

\newacronym{CDMA}{CDMA}{\gls{l_CDMA}}
\newglossaryentry{l_CDMA}{%
	name={Code Division Multiple Access}, 
	description={Multiplex Verfahren bei dem der gleichzeitige alle Sender gleichzeitig auf derselben Frequenz Zugriff auf das Medium haben. CDMA ist ein Spread Spectrum Verfahren, bei dem die Frequenzspreizung durch Mischung des Informationssignals mit orthogonalen Codes realisiert wird}
}

\newacronym{LNA}{LNA}{Low Noise Amplifier}

\newacronym{AGC}{AGC}{Automatic Gain Control}

\newacronym{TCXO}{TCXO}{\gls{l_TCXO}}
\newglossaryentry{l_TCXO}{%
	name={Temperature Controlled Oscillator}, 
	description={Ein Quarzoszillator, der zur verbesserten Frequenzstabilität temperaturkompensiert ist.}
}

\newacronym{PGA}{PGA}{Programmable Gain Amplifier}

\newacronym{zword}{Z-Word}{\gls{l_zword}}
\newglossaryentry{l_zword}{%
	name={Time-of-Week}, 
	description={Anzahl der Sekunden seit dem \emph{zero-time point}. Der \emph{zero-time point} ist der Zeitpunkt um Mitternacht vom 5. Januar 1980 auf den 6. Januar 1980.}
}

\newacronym{TOW}{TOW}{\gls{l_TOW}}
\newglossaryentry{l_TOW}{%
	name={Time-of-Week}, 
	description={Die Time of Week gibt die Anzahl der Sekunden seit dem letzten Sonntag an. Verkürzte Version des \gls{zword}.}
}

\newacronym{HOW}{HOW}{\gls{l_HOW}}
\newglossaryentry{l_HOW}{%
	name={Hand-over-Word}, 
	description={Teil der vom GPS \gls{SV} gesendendeten NAV Daten. Der im HOW übertragene Wert gibt die \gls{TOW} zum Zeitpunkt des Beginn des nächsten Subframe an. HOW ist eine verkürzte Version von \gls{TOW}, da ein Subframe \SI{6}{\second} dauert.}
}

\newacronym{DMA}{DMA}{\gls{l_DMA}}
\newglossaryentry{l_DMA}{%
	name={Direct Memory Access}, 
	description={Technik bei der Daten zwischen Geräten, die an den gemeinsam genutzten Daten- und Adressbus angeschlossen sind, ohne Mitwirken des Prozessors ausgetauscht werden können.}
}

\newacronym{AMC}{AMC}{Asynchronous Memory Controller}

\newacronym{FIFO}{FIFO}{First In - First Out}

\newacronym{FSM}{FSM}{\gls{l_FSM}}
\newglossaryentry{l_FSM}{%
	name={Finite State Machine}, 
	description={Endlicher Zustandsautomat.}
}

\newacronym{FCW}{FCW}{\gls{l_FCW}}
\newglossaryentry{l_FCW}{%
	name={Frequency Control Word}, 
	description={Steuerwort, dass die Ausgangsfrequenz eines \gls{NCO} bestimmt. Dabei ist das Steuerwort nicht notwendigerweise \emph{gleich} der Ausgangsfrequenz, sondern lediglich proportional dazu.}
}

\newacronym{LUT}{LUT}{Look-Up-Table}

\newacronym{ISR}{ISR}{Interrupt Service Routine}